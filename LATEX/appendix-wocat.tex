% ============================================================
% APÊNDICE — Questionário WOCAT sobre Tecnologias de MST
% Tradução para português brasileiro do documento:
%   "Questionnaire on Sustainable Land Management (SLM) Technologies"
%   WOCAT, versão 2019.
% ============================================================

\appendix
\section*{Apêndice~A --- Questionário WOCAT sobre Tecnologias de Manejo Sustentável da Terra (MST)}
\addcontentsline{toc}{section}{Apêndice~A --- Questionário WOCAT--MST}
\label{app:wocat}
\setcounter{table}{0}
\renewcommand{\thetable}{A\arabic{table}}
\setcounter{figure}{0}
\renewcommand{\thefigure}{A\arabic{figure}}

\noindent\textbf{Fonte:} WOCAT --- \textit{World Overview of Conservation Approaches and Technologies}, Centre for Development and Environment (CDE), Universidade de Berna, Suíça. Versão 2019 \citep{Liniger2019}.\\[6pt]
\noindent\textbf{Nota:} Este apêndice apresenta a tradução integral para o português brasileiro do questionário sobre Tecnologias de MST (\textit{SLM Technologies}). Os campos de preenchimento foram preservados como linhas pontilhadas. Círculos ($\bigcirc$) indicam seleção única; quadrados ($\square$) indicam seleção múltipla.

%% ---------------------------------------------------------
\subsection*{Introdução ao questionário}

O WOCAT fornece ferramentas padronizadas, dirigidas pelo usuário, de acesso aberto e uso global para documentação e avaliação de práticas de Manejo Sustentável da Terra (MST). \textbf{MST}, no contexto do WOCAT, define-se como o uso sustentável dos recursos da terra, incluindo solos, água, vegetação e animais. O WOCAT concentra-se em esforços para prevenir e reduzir a degradação da terra e restaurar terras degradadas por meio de \textbf{tecnologias de manejo da terra} e \textbf{abordagens para implementá-las}. Todas as práticas podem ser consideradas, sejam elas indígenas, recentemente introduzidas por projetos ou inovações recentes de usuários da terra.

\medskip\noindent
\textbf{Tecnologia ou Abordagem?} Existem dois questionários separados: um para Tecnologias e outro para Abordagens. Uma \textbf{Tecnologia de MST} é uma prática física que controla a degradação da terra e aumenta a produtividade e/ou outros serviços ecossistêmicos. Uma \textbf{Abordagem de MST} define os meios empregados para implementar uma ou mais Tecnologias de MST.

\medskip\noindent
\textbf{Observações gerais:}
\begin{itemize}[nosep]
  \item Responda a todas as perguntas. Caso dados precisos não estejam disponíveis, forneça a melhor estimativa com base em seu julgamento profissional. Se determinadas perguntas não forem aplicáveis, indique ``n/a''.
  \item Preencha um questionário separado para cada Tecnologia.
\end{itemize}

%% =========================================================
\subsection*{1\quad Informações Gerais}

\subsubsection*{1.1\quad Nome da Tecnologia de MST (doravante referida como ``Tecnologia'')}

Nome: \dotfill\\
Nome localmente utilizado: \dotfill\\
País: \dotfill

\subsubsection*{1.2\quad Dados de contato das pessoas-recurso e instituições envolvidas na avaliação e documentação da Tecnologia}

\noindent\textbf{Compilador(a)}\\
\textit{A pessoa que conduziu as entrevistas, compilou as informações e preencheu o questionário.}\\[4pt]
Sobrenome: \dotfill\quad Nome(s): \dotfill\quad $\bigcirc$~Sra.\quad $\bigcirc$~Sr.\\
Nome da instituição: \dotfill\\
País: \dotfill\quad Telefone: \dotfill\quad E-mail: \dotfill

\medskip\noindent\textbf{Pessoa(s)-recurso principal(is)}\\
\textit{Pessoa(s) que forneceram a maioria das informações documentadas neste questionário. Podem ser usuários da terra, especialistas em MST (p.\,ex.\@ assessores técnicos, pesquisadores) ou outras pessoas.}\\[4pt]
Especifique a pessoa-recurso:\quad $\bigcirc$~Usuário da terra\quad $\bigcirc$~Especialista/assessor técnico\quad $\bigcirc$~Co-compilador\quad $\bigcirc$~Outro: \dotfill

\subsubsection*{1.3\quad Condições relativas ao uso de dados documentados via WOCAT}

O compilador e a(s) pessoa(s)-recurso aceitam as condições relativas ao uso de dados documentados via WOCAT?\\
$\bigcirc$~Sim \qquad $\bigcirc$~Não

\medskip\noindent\textit{Condições:} Os dados capturados serão armazenados no banco de dados online do WOCAT. Os dados são de acesso aberto e disponibilizados sob a licença \textit{Creative Commons Attribution-NonCommercial-ShareAlike 3.0 Unported}.

\subsubsection*{1.4\quad Declaração sobre a sustentabilidade da Tecnologia descrita}

A Tecnologia possui efeitos adversos sobre a degradação da terra, de modo que não pode ser declarada uma tecnologia de manejo \textit{sustentável} da terra?\\
$\bigcirc$~Sim \qquad $\bigcirc$~Não\\
Comentários: \dotfill

\subsubsection*{1.5\quad Referência a Questionário(s) sobre Abordagens de MST (documentados usando WOCAT)}

Nome da Abordagem de MST: \dotfill\quad Compilador: \dotfill

%% =========================================================
\subsection*{2\quad Descrição de uma Tecnologia de MST}

\textit{Uma Tecnologia de MST é uma prática aplicada no campo que controla a degradação da terra e/ou aumenta a produtividade. Este questionário foi projetado para documentar uma única Tecnologia de MST e não pode ser usado para avaliar uma propriedade inteira.}

\subsubsection*{2.1\quad Descrição curta da Tecnologia}
\textit{Resuma a Tecnologia em 1--2 frases, contendo palavras-chave relevantes.}\\[4pt]
\dotfill\\[2pt]\dotfill

\subsubsection*{2.2\quad Descrição detalhada da Tecnologia}
\textit{A descrição detalhada deve apresentar um panorama conciso mas abrangente da Tecnologia. Deve abordar: (1) características/elementos principais (incluindo especificações técnicas); (2) onde é aplicada (ambiente natural e humano); (3) propósitos/funções; (4) principais atividades/insumos para implantação/manutenção; (5) benefícios/impactos; (6) percepção dos usuários. Extensão ideal: 2.500--3.000 caracteres (máximo: 3.500).}\\[4pt]
\dotfill\\[2pt]\dotfill\\[2pt]\dotfill

\subsubsection*{2.3\quad Fotografias da Tecnologia}
\textit{Forneça ao menos dois arquivos digitais (JPG, PNG, GIF) de alta resolução, com legenda explicativa para cada foto. As fotos devem ilustrar a Tecnologia antes/depois ou com/sem medidas de MST quando apropriado.}

\subsubsection*{2.4\quad Vídeos da Tecnologia}
\textit{Caso disponíveis, indique links para plataformas públicas (p.\,ex.\@ Vimeo, YouTube).}

\subsubsection*{2.5\quad País/região/locais onde a Tecnologia foi aplicada e que são cobertos por esta avaliação}

País: \dotfill\quad Região/Estado/Província: \dotfill\\
Especificação adicional do local: \dotfill\\[4pt]
Número de locais considerados/analisados:\\
$\bigcirc$~Local único \quad $\bigcirc$~2--10 locais \quad $\bigcirc$~10--100 locais \quad $\bigcirc$~100--1.000 locais \quad $\bigcirc$~$>$\,1.000 locais\\[4pt]
Coordenadas georreferenciadas (graus decimais): \dotfill\\[4pt]
Distribuição da Tecnologia:\\
$\bigcirc$~Distribuída uniformemente sobre uma área\\
$\bigcirc$~Aplicada em pontos específicos/concentrada em área pequena\\
Área coberta (se distribuída uniformemente, em km\textsuperscript{2}): \dotfill

\subsubsection*{2.6\quad Data de implementação}

Ano de implementação: \dotfill\\
$\bigcirc$~Menos de 10 anos (recente) \quad $\bigcirc$~10--50 anos \quad $\bigcirc$~Mais de 50 anos (tradicional)

\subsubsection*{2.7\quad Introdução da Tecnologia}

Especifique como a Tecnologia foi introduzida:\\
$\square$~Como parte de um sistema tradicional\\
$\square$~Por inovação recente de usuários da terra\\
$\square$~Durante experimentos/pesquisa\\
$\square$~Por projetos/intervenções externas\\
$\square$~Outro (especifique): \dotfill

%% =========================================================
\subsection*{3\quad Classificação da Tecnologia de MST}

\subsubsection*{3.1\quad Propósito(s) principal(is) da Tecnologia}
\textit{Várias respostas possíveis (máximo 5).}\\[4pt]
$\square$~Melhorar a produção (culturas, forragem, madeira/fibra, água, energia)\\
$\square$~Prevenir, reduzir a degradação da terra; restaurar/reabilitar a terra\\
$\square$~Conservar ecossistemas\\
$\square$~Preservar/melhorar a biodiversidade\\
$\square$~Criar impacto econômico benéfico (p.\,ex.\@ aumentar renda/oportunidades de emprego)\\
$\square$~Criar impacto social benéfico (p.\,ex.\@ reduzir conflitos por recursos naturais)\\
$\square$~Reduzir risco de desastres (p.\,ex.\@ secas, enchentes, deslizamentos)\\
$\square$~Adaptar-se a mudanças/extremos climáticos e seus impactos\\
$\square$~Mitigar mudanças climáticas e seus impactos (p.\,ex.\@ sequestro de carbono)\\
$\square$~Outro propósito (especifique): \dotfill

\subsubsection*{3.2\quad Tipo(s) atual(is) de uso da terra onde a Tecnologia é aplicada}

O uso da terra é misto na mesma unidade (conforme definições ICRAF)?~$\bigcirc$~Sim~$\bigcirc$~Não\\
Se sim, especifique o sistema agroflorestal:\\
$\bigcirc$~Agrossilvicultura \quad $\bigcirc$~Agrossilvipastoril \quad $\bigcirc$~Silvipastoril

\medskip\noindent
\textit{Selecione o(s) tipo(s) de uso da terra e subcategorias:}

\begin{table}[htbp]
\centering\small
\caption{Tipos de uso da terra e subcategorias}
\label{tab:app_land_use}
\begin{tabular}{p{3.2cm}p{5.5cm}p{5.5cm}}
\toprule
\textbf{Categoria} & \textbf{Subcategorias} & \textbf{Especificações} \\
\midrule
$\square$ Terra de cultivo &
  $\square$ Cultivo anual;
  $\square$ Cultivo perene;
  $\square$ Cultivo arbóreo/arbustivo;
  $\square$ Outro &
  Culturas: \dotfill \newline
  Estações/ano: $\bigcirc$1 $\bigcirc$2 $\bigcirc$3 \newline
  Rotação? $\bigcirc$Sim $\bigcirc$Não \newline
  Consórcio? $\bigcirc$Sim $\bigcirc$Não \\
\addlinespace
$\square$ Terra de pastagem &
  Extensivo: $\square$ Nomadismo; $\square$ Seminomadismo; $\square$ Transumância; $\square$ Ranching \newline
  Intensivo: $\square$ Corte-e-carrega; $\square$ Pastagem melhorada &
  Tipo de animal: \dotfill \newline
  Integração lavoura-pecuária? $\bigcirc$Sim $\bigcirc$Não \\
\addlinespace
$\square$ Floresta/Mata &
  $\square$ Floresta (semi)natural; $\square$ Plantação florestal &
  Tipo de árvore: \dotfill \newline
  $\bigcirc$Decídua $\bigcirc$Mista $\bigcirc$Perenifólia \\
\addlinespace
$\square$ Assentamentos/infraestrutura &
  $\square$ Edificações; $\square$ Transporte; $\square$ Energia; $\square$ Outro &
  Observações: \dotfill \\
\addlinespace
$\square$ Cursos d'água/corpos hídricos/áreas úmidas &
  $\square$ Drenagens; $\square$ Lagoas/barragens; $\square$ Pântanos; $\square$ Rios e zona ripária; $\square$ Lagos; $\square$ Mar &
  Produtos/serviços: \dotfill \\
\addlinespace
$\square$ Mineração &
  Especifique: \dotfill &
  Produtos: \dotfill \\
\addlinespace
$\square$ Terra improdutiva &
  Especifique: \dotfill &
  Observações: \dotfill \\
\bottomrule
\end{tabular}
\end{table}

\subsubsection*{3.3\quad Uso da terra antes da implementação da Tecnologia}

O uso da terra mudou devido à implementação da Tecnologia?\\
$\bigcirc$~Não (pule para 3.4) \qquad $\bigcirc$~Sim (preencha abaixo com o uso anterior)\\[4pt]
\textit{Se sim, preencha as mesmas categorias da Tabela~\ref{tab:app_land_use} referentes ao uso anterior.}

\subsubsection*{3.4\quad Suprimento hídrico}

Suprimento hídrico para a terra onde a Tecnologia é aplicada:\\
$\bigcirc$~Sequeiro \quad $\bigcirc$~Misto (sequeiro--irrigado) \quad $\bigcirc$~Irrigação plena \quad $\bigcirc$~Outro: \dotfill\\
Comentário: \dotfill

\subsubsection*{3.5\quad Grupo de MST ao qual a Tecnologia pertence}
\textit{Atribua a Tecnologia a um dos seguintes grupos (máx.\ 3):}\\[4pt]
$\square$~Manejo de florestas naturais/seminaturais \quad $\square$~Manejo de plantações florestais\\
$\square$~Agrofloresta \quad $\square$~Quebra-vento/cinturão verde\\
$\square$~Fechamento de área (cessar uso, apoiar restauração)\\
$\square$~Sistema rotacional (rotação de culturas, pousio, agricultura itinerante)\\
$\square$~Pastoralismo e manejo de pastagens \quad $\square$~Integração lavoura-pecuária\\
$\square$~Melhoria da cobertura do solo/vegetação \quad $\square$~Distúrbio mínimo do solo\\
$\square$~Manejo integrado da fertilidade do solo \quad $\square$~Medida em nível (cross-slope)\\
$\square$~Manejo integrado de pragas e doenças (incl.\ agricultura orgânica)\\
$\square$~Variedades vegetais/raças animais melhoradas\\
$\square$~Captação de água \quad $\square$~Manejo de irrigação (incl.\ abastecimento, drenagem)\\
$\square$~Desvio e drenagem de água\\
$\square$~Manejo de águas superficiais (nascentes, rios, lagos, zona ripária)\\
$\square$~Manejo de águas subterrâneas \quad $\square$~Proteção/manejo de áreas úmidas\\
$\square$~Gestão de resíduos/efluentes \quad $\square$~Eficiência energética\\
$\square$~Apicultura, aquicultura, avicultura, cunicultura, sericicultura, etc.\\
$\square$~Hortas domésticas \quad $\square$~Redução de risco de desastres baseada em ecossistemas\\
$\square$~Medidas pós-colheita \quad $\square$~Outro (especifique): \dotfill

\subsubsection*{3.6\quad Medidas de MST que compõem a Tecnologia}
\textit{Várias respostas possíveis.}

\begin{table}[htbp]
\centering\small
\caption{Medidas de MST --- tipos, subcategorias e exemplos}
\label{tab:app_slm_measures}
\begin{tabular}{p{3cm}p{4cm}p{7cm}}
\toprule
\textbf{Tipo de medida} & \textbf{Subcategorias} & \textbf{Exemplos} \\
\midrule
\textbf{Agronômicas} \newline
(associadas a cultivos anuais; repetidas sazonalmente; curta duração; sem alteração do perfil do terreno) &
  A1: Cobertura vegetal/solo \newline
  A2: Matéria orgânica/fertilidade \newline
  A3: Tratamento superficial \newline
  A4: Tratamento subsuperficial \newline
  A5: Manejo de sementes \newline
  A6: Manejo de resíduos \newline
  A7: Outras &
  Consórcio, cobertura morta, plantio direto, cultivo em nível, compostagem, adubação verde, rotação de culturas, seleção de sementes \\
\addlinespace
\textbf{Vegetativas} \newline
(uso de gramíneas perenes, arbustos ou árvores; longa duração) &
  V1: Cobertura arbórea/arbustiva \newline
  V2: Gramíneas e herbáceas perenes \newline
  V3: Desmatamento de vegetação \newline
  V4: Substituição/remoção de espécies invasoras \newline
  V5: Outras &
  Agrofloresta, quebra-ventos, faixas de gramíneas em nível, cercas vivas, viveiros \\
\addlinespace
\textbf{Estruturais} \newline
(permanentes; requerem investimento substancial; movimentação de terra e/ou construção) &
  S1: Terraços \newline
  S2: Cordões/camalhões \newline
  S3: Valas graduadas/canais \newline
  S4: Valas/covas niveladas \newline
  S5: Barragens/tanques \newline
  S6: Muros/barreiras/paliçadas/cercas \newline
  S7: Captação/irrigação \newline
  S8: Saneamento/efluentes \newline
  S9: Abrigos para plantas/animais \newline
  S10: Eficiência energética \newline
  S11: Outras &
  Terraços de bancada, cordões de pedra, valas de infiltração, barraginhas, estabilização de ravinas (\textit{check dams}), captação de água de telhado \\
\addlinespace
\textbf{De manejo} \newline
(mudança fundamental no uso da terra; intensidade reduzida) &
  M1: Mudança de tipo de uso \newline
  M2: Mudança de manejo/intensidade \newline
  M3: Disposição conforme ambiente \newline
  M4: Mudança de cronograma \newline
  M5: Controle de composição de espécies \newline
  M6: Gestão de resíduos \newline
  M7: Outras &
  Fechamento de área, pousio gerenciado, acesso controlado, ajuste de carga animal, queima prescrita \\
\addlinespace
\textbf{Outras} &
  \multicolumn{2}{p{11cm}}{Apicultura, aquicultura, armazenamento de alimentos, processamento pós-colheita} \\
\bottomrule
\end{tabular}
\end{table}

\noindent Sistema de preparo (se relevante):\quad $\square$~Plantio direto \quad $\square$~Preparo reduzido ($>$30\% cobertura) \quad $\square$~Preparo convencional ($<$30\% cobertura)\\
Manejo de resíduos:\quad $\square$~Queimados \quad $\square$~Pastejados \quad $\square$~Recolhidos \quad $\square$~Mantidos

\subsubsection*{3.7\quad Principais tipos de degradação da terra abordados pela Tecnologia}
\textit{Várias respostas possíveis.}\\[4pt]
$\square$~\textbf{W --- Erosão hídrica do solo:} Wt~perda de solo superficial; Wg~erosão em ravinas ($>$30\,cm); Wm~movimentos de massa; Wr~erosão de margens fluviais; Wc~erosão costeira; Wo~efeitos fora do local\\
$\square$~\textbf{E --- Erosão eólica do solo:} Et~perda superficial; Ed~deflação/deposição; Eo~efeitos fora do local\\
$\square$~\textbf{C --- Deterioração química:} Cn~declínio de fertilidade/MOS; Ca~acidificação; Cp~poluição; Cs~salinização\\
$\square$~\textbf{P --- Deterioração física:} Pc~compactação; Pk~selamento/crusting; Pi~impermeabilização; Pw~encharcamento; Ps~subsidência; Pu~perda de função bioprodutiva\\
$\square$~\textbf{B --- Degradação biológica:} Bc~redução de cobertura; Bh~perda de habitats; Bq~declínio de biomassa; Bf~efeitos de fogo; Bs~declínio de diversidade; Bl~perda de vida do solo; Bp~aumento de pragas\\
$\square$~\textbf{H --- Degradação hídrica:} Ha~aridificação; Hs~mudança em águas superficiais; Hg~mudança em lençol freático; Hp~declínio de qualidade superficial; Hq~declínio de qualidade subterrânea; Hw~redução de capacidade tamponante de áreas úmidas

\subsubsection*{3.8\quad Prevenção, redução ou restauração da degradação da terra}
\textit{Marque no máximo duas opções.}\\[4pt]
$\bigcirc$~Prevenir/evitar a degradação da terra \quad $\bigcirc$~Reduzir a degradação\\
$\bigcirc$~Restaurar/reabilitar terra severamente degradada \quad $\bigcirc$~Adaptar-se à degradação\\
$\bigcirc$~Não aplicável

%% =========================================================
\subsection*{4\quad Especificações técnicas, atividades de implantação, insumos e custos}

\subsubsection*{4.1\quad Desenho técnico da Tecnologia}
\textit{Forneça desenho(s) detalhado(s) com dimensões, especificações técnicas, espaçamento, gradiente, etc. Formato quadrado é ideal. Apenas símbolos e/ou números; textos explicativos no campo separado.}

\subsubsection*{4.2\quad Informações gerais para cálculo de insumos e custos}
\textit{Distingua entre implantação/investimento inicial e manutenção/atividades recorrentes anuais. Custos em preços de mercado. Se a mão de obra é fornecida pelos próprios usuários, indique o custo equivalente de mão de obra contratada.}\\[4pt]
Custos calculados:\\
$\bigcirc$~Por área da Tecnologia $\rightarrow$ tamanho/unidade: \dotfill\\
$\bigcirc$~Por unidade da Tecnologia $\rightarrow$ especifique: \dotfill\\
Moeda: $\bigcirc$~Dólares americanos (USD) \quad $\bigcirc$~Outra (especifique): \dotfill\\
Taxa de câmbio (se relevante): 1~USD = \dotfill\\
Custo médio diário de mão de obra contratada: \dotfill

\subsubsection*{4.3\quad Atividades de implantação}
\textit{Liste as atividades na sequência e indique o período.}\\[4pt]
\begin{tabular}{clp{4cm}}
1. & \dotfill & Período: \dotfill \\
2. & \dotfill & Período: \dotfill \\
3. & \dotfill & Período: \dotfill \\
4. & \dotfill & Período: \dotfill \\
5. & \dotfill & Período: \dotfill \\
6. & \dotfill & Período: \dotfill \\
\end{tabular}

\subsubsection*{4.4\quad Custos dos insumos para implantação}

\begin{table}[htbp]
\centering\small
\caption{Custos de implantação da Tecnologia}
\label{tab:app_cost_estab}
\begin{tabular}{p{2.8cm}p{2.5cm}lrrrl}
\toprule
\textbf{Insumo} & \textbf{Especificação} & \textbf{Unidade} & \textbf{Qtde.} & \textbf{Custo/un.} & \textbf{Custo total} & \textbf{\% usuário} \\
\midrule
Mão de obra      & & & & & & \\
Equipamento      & & & & & & \\
Material vegetal & & & & & & \\
Fertiliz./biocidas & & & & & & \\
Material de constr. & & & & & & \\
Outros           & & & & & & \\
\midrule
\multicolumn{5}{l}{\textbf{Custo total de implantação}} & & \\
\bottomrule
\end{tabular}
\end{table}

\subsubsection*{4.5\quad Atividades de manutenção/recorrentes}
\textit{Liste as atividades na sequência e indique período/frequência.}\\[4pt]
\begin{tabular}{clp{5cm}}
1. & \dotfill & Período/Frequência: \dotfill \\
2. & \dotfill & Período/Frequência: \dotfill \\
3. & \dotfill & Período/Frequência: \dotfill \\
4. & \dotfill & Período/Frequência: \dotfill \\
5. & \dotfill & Período/Frequência: \dotfill \\
\end{tabular}

\subsubsection*{4.6\quad Custos dos insumos para manutenção (por ano)}

\begin{table}[htbp]
\centering\small
\caption{Custos anuais de manutenção da Tecnologia}
\label{tab:app_cost_maint}
\begin{tabular}{p{2.8cm}p{2.5cm}lrrrl}
\toprule
\textbf{Insumo} & \textbf{Especificação} & \textbf{Unidade} & \textbf{Qtde.} & \textbf{Custo/un.} & \textbf{Custo total} & \textbf{\% usuário} \\
\midrule
Mão de obra      & & & & & & \\
Equipamento      & & & & & & \\
Material vegetal & & & & & & \\
Fertiliz./biocidas & & & & & & \\
Material de constr. & & & & & & \\
Outros           & & & & & & \\
\midrule
\multicolumn{5}{l}{\textbf{Custo total de manutenção}} & & \\
\bottomrule
\end{tabular}
\end{table}

\subsubsection*{4.7\quad Fatores mais importantes que afetam os custos}
\dotfill\\[2pt]\dotfill

%% =========================================================
\subsection*{5\quad Ambiente natural e humano}

\textit{Forneça detalhes das condições naturais (biofísicas) onde a Tecnologia é aplicada. Descreva as condições sem impacto do MST.}

\subsubsection*{5.1\quad Clima}

\noindent\textbf{Precipitação anual}\\
$\bigcirc$~$<$250\,mm \quad $\bigcirc$~251--500\,mm \quad $\bigcirc$~501--750\,mm \quad $\bigcirc$~751--1.000\,mm \quad $\bigcirc$~1.001--1.500\,mm\\
$\bigcirc$~1.501--2.000\,mm \quad $\bigcirc$~2.001--3.000\,mm \quad $\bigcirc$~3.001--4.000\,mm \quad $\bigcirc$~$>$4.000\,mm\\[4pt]
Precipitação média anual (se conhecida): \dotfill\,mm\\[4pt]
\textbf{Zona agroclimática}\\
$\bigcirc$~Úmida (período de crescimento $>$270 dias)\\
$\bigcirc$~Subúmida (180--269 dias)\\
$\bigcirc$~Semiárida (75--179 dias)\\
$\bigcirc$~Árida ($<$74 dias)

\subsubsection*{5.2\quad Topografia}

\noindent\textbf{Declividade média}\\
$\bigcirc$~Plana (0--2\%) \quad $\bigcirc$~Suave (3--5\%) \quad $\bigcirc$~Moderada (6--10\%) \quad $\bigcirc$~Ondulada (11--15\%)\\
$\bigcirc$~Colinosa (16--30\%) \quad $\bigcirc$~Íngreme (31--60\%) \quad $\bigcirc$~Muito íngreme ($>$60\%)\\[4pt]
\textbf{Formas de relevo}\\
$\bigcirc$~Planalto/planícies \quad $\bigcirc$~Cumeeiras \quad $\bigcirc$~Encostas de montanha \quad $\bigcirc$~Encostas de colina\\
$\bigcirc$~Sopé de encosta \quad $\bigcirc$~Fundos de vale\\[4pt]
\textbf{Zona altitudinal}\\
$\bigcirc$~$<$100\,m \quad $\bigcirc$~101--500\,m \quad $\bigcirc$~501--1.000\,m \quad $\bigcirc$~1.001--1.500\,m \quad $\bigcirc$~1.501--2.000\,m\\
$\bigcirc$~2.001--2.500\,m \quad $\bigcirc$~2.501--3.000\,m \quad $\bigcirc$~3.001--4.000\,m \quad $\bigcirc$~$>$4.000\,m\\[4pt]
Situação específica:\quad $\bigcirc$~Convexa \quad $\bigcirc$~Côncava \quad $\bigcirc$~Não relevante

\subsubsection*{5.3\quad Solos}
\textit{Parâmetros baseados nos padrões FAO.}\\[4pt]
\textbf{Profundidade média do solo}\\
$\bigcirc$~Muito raso (0--20\,cm) \quad $\bigcirc$~Raso (21--50\,cm) \quad $\bigcirc$~Moderadamente profundo (51--80\,cm)\\
$\bigcirc$~Profundo (81--120\,cm) \quad $\bigcirc$~Muito profundo ($>$120\,cm)\\[4pt]
\textbf{Textura do solo (superficial)}\\
$\bigcirc$~Grossa/leve (arenosa) \quad $\bigcirc$~Média (franca, siltosa) \quad $\bigcirc$~Fina/pesada (argilosa)\\[4pt]
\textbf{Matéria orgânica no horizonte superficial}\\
$\bigcirc$~Alta ($>$3\%) \quad $\bigcirc$~Média (1--3\%) \quad $\bigcirc$~Baixa ($<$1\%)

\subsubsection*{5.4\quad Disponibilidade e qualidade da água}

\noindent\textbf{Nível do lençol freático}\quad $\bigcirc$~Superficial \quad $\bigcirc$~$<$5\,m \quad $\bigcirc$~5--50\,m \quad $\bigcirc$~$>$50\,m\\[4pt]
\textbf{Disponibilidade de águas superficiais}\\
$\bigcirc$~Excesso \quad $\bigcirc$~Boa (disponível o ano todo) \quad $\bigcirc$~Média (não disponível o ano todo) \quad $\bigcirc$~Pouca/nenhuma\\[4pt]
\textbf{Qualidade da água (não tratada)}\\
$\bigcirc$~Boa (potável) \quad $\bigcirc$~Pobre (requer tratamento) \quad $\bigcirc$~Apenas para uso agrícola \quad $\bigcirc$~Inutilizável\\[4pt]
Salinidade é um problema? $\bigcirc$~Sim \quad $\bigcirc$~Não\\
Inundação da área ocorre? $\bigcirc$~Sim ($\bigcirc$ frequente / $\bigcirc$ episódica) \quad $\bigcirc$~Não

\subsubsection*{5.5\quad Biodiversidade}

\textbf{Diversidade de espécies}\quad $\bigcirc$~Alta \quad $\bigcirc$~Média \quad $\bigcirc$~Baixa\\
\textbf{Diversidade de habitats}\quad $\bigcirc$~Alta \quad $\bigcirc$~Média \quad $\bigcirc$~Baixa

\subsubsection*{5.6\quad Características dos usuários da terra que aplicam a Tecnologia}

\noindent\textbf{Sedentário ou nômade}\quad $\bigcirc$~Sedentário \quad $\bigcirc$~Seminômade \quad $\bigcirc$~Nômade \quad $\bigcirc$~Outro\\[4pt]
\textbf{Orientação de mercado}\quad $\bigcirc$~Subsistência \quad $\bigcirc$~Misto \quad $\bigcirc$~Comercial\\[4pt]
\textbf{Nível relativo de riqueza}\quad $\bigcirc$~Muito pobre \quad $\bigcirc$~Pobre \quad $\bigcirc$~Médio \quad $\bigcirc$~Rico \quad $\bigcirc$~Muito rico\\[4pt]
\textbf{Individual ou grupo}\quad $\bigcirc$~Individual/domicílio \quad $\bigcirc$~Grupos/comunidade \quad $\bigcirc$~Cooperativa \quad $\bigcirc$~Empregado\\[4pt]
\textbf{Gênero}\quad $\square$~Mulheres \quad $\square$~Homens\\[4pt]
\textbf{Faixa etária}\quad $\square$~Crianças \quad $\square$~Jovens \quad $\square$~Meia-idade \quad $\square$~Idosos\\[4pt]
\textbf{Renda extra-propriedade}\quad $\bigcirc$~$<$10\% \quad $\bigcirc$~10--50\% \quad $\bigcirc$~$>$50\% da renda total\\[4pt]
\textbf{Mecanização}\quad $\bigcirc$~Manual \quad $\bigcirc$~Tração animal \quad $\bigcirc$~Mecanizada/motorizada

\subsubsection*{5.7\quad Área média de terra possuída/arrendada/utilizada pelos usuários}

$\bigcirc$~$<$0,5\,ha \quad $\bigcirc$~0,5--1\,ha \quad $\bigcirc$~1--2\,ha \quad $\bigcirc$~2--5\,ha \quad $\bigcirc$~5--15\,ha \quad $\bigcirc$~15--50\,ha\\
$\bigcirc$~50--100\,ha \quad $\bigcirc$~100--500\,ha \quad $\bigcirc$~500--1.000\,ha \quad $\bigcirc$~1.000--10.000\,ha \quad $\bigcirc$~$>$10.000\,ha\\[4pt]
Escala local:\quad $\bigcirc$~Pequena \quad $\bigcirc$~Média \quad $\bigcirc$~Grande

\subsubsection*{5.8\quad Propriedade da terra, direitos de uso da terra e direitos de uso da água}

\noindent\textbf{Propriedade da terra}\\
$\bigcirc$~Estado \quad $\bigcirc$~Empresa \quad $\bigcirc$~Comunal/vilarejo \quad $\bigcirc$~Grupo \quad $\bigcirc$~Individual sem título \quad $\bigcirc$~Individual com título \quad $\bigcirc$~Outro\\[4pt]
\textbf{Direitos de uso da terra}\\
$\bigcirc$~Acesso livre \quad $\bigcirc$~Comunal (organizado) \quad $\bigcirc$~Arrendado \quad $\bigcirc$~Individual \quad $\bigcirc$~Outro\\[4pt]
\textbf{Direitos de uso da água}\\
$\bigcirc$~Acesso livre \quad $\bigcirc$~Comunal (organizado) \quad $\bigcirc$~Arrendado \quad $\bigcirc$~Individual \quad $\bigcirc$~Outro\\[4pt]
Direitos de uso baseados em sistema jurídico tradicional? $\bigcirc$~Sim (especifique: \dotfill) $\bigcirc$~Não

\subsubsection*{5.9\quad Acesso a serviços e infraestrutura}

\begin{table}[htbp]
\centering\small
\caption{Acesso a serviços e infraestrutura}
\label{tab:app_services}
\begin{tabular}{lccc}
\toprule
\textbf{Serviço} & \textbf{Precário} & \textbf{Moderado} & \textbf{Bom} \\
\midrule
Saúde                         & $\bigcirc$ & $\bigcirc$ & $\bigcirc$ \\
Educação                      & $\bigcirc$ & $\bigcirc$ & $\bigcirc$ \\
Assistência técnica            & $\bigcirc$ & $\bigcirc$ & $\bigcirc$ \\
Emprego (p.\,ex.\@ extra-propriedade) & $\bigcirc$ & $\bigcirc$ & $\bigcirc$ \\
Mercados                      & $\bigcirc$ & $\bigcirc$ & $\bigcirc$ \\
Energia                       & $\bigcirc$ & $\bigcirc$ & $\bigcirc$ \\
Estradas e transporte          & $\bigcirc$ & $\bigcirc$ & $\bigcirc$ \\
Água potável e saneamento      & $\bigcirc$ & $\bigcirc$ & $\bigcirc$ \\
Serviços financeiros           & $\bigcirc$ & $\bigcirc$ & $\bigcirc$ \\
\bottomrule
\end{tabular}
\end{table}

%% =========================================================
\subsection*{6\quad Impactos e declarações conclusivas}

\textit{Avalie os impactos relevantes. Se dados medidos não estiverem disponíveis, forneça sua melhor estimativa. Utilize as colunas ``Quantifique antes/depois do MST'' e ``Comentários'' para demonstrar evidências.}

\subsubsection*{6.1\quad Impactos no local (\textit{on-site})}

\noindent\textbf{Impactos socioeconômicos}\\[2pt]
\textit{Produção:} produção agrícola, qualidade das culturas, produção de forragem, produção animal, produção de madeira, qualidade florestal, produtos florestais não madeireiros, risco de falha da produção, diversidade de produtos, área de produção, manejo da terra, geração de energia.\\[4pt]
\textit{Disponibilidade e qualidade da água:} disponibilidade/qualidade de água potável, água para pecuária, água para irrigação, demanda de irrigação.\\[4pt]
\textit{Renda e custos:} despesas com insumos agrícolas, renda, diversidade de fontes de renda, disparidades econômicas, carga de trabalho.\\[6pt]
\noindent\textbf{Impactos socioculturais}\\[2pt]
Segurança alimentar/autossuficiência, situação de saúde, direitos de uso da terra/água, oportunidades culturais (espirituais, religiosas, estéticas), oportunidades de lazer, instituições comunitárias, instituições nacionais, conhecimento em MST/degradação, mitigação de conflitos, situação de grupos socioeconomicamente desfavorecidos.\\[6pt]
\noindent\textbf{Impactos ecológicos}\\[2pt]
\textit{Ciclo hidrológico/escoamento:} quantidade e qualidade da água, captação/coleta de água, escoamento superficial, drenagem, nível do lençol freático, evaporação.\\[4pt]
\textit{Solo:} umidade do solo, cobertura do solo, perda de solo, acúmulo de solo, selamento/crusting, compactação, ciclagem de nutrientes, salinidade, matéria orgânica/carbono subsuperficial, acidez.\\[4pt]
\textit{Biodiversidade:} cobertura vegetal, biomassa/carbono aéreo, diversidade vegetal, espécies invasoras, diversidade animal, espécies benéficas, espécies nocivas, diversidade de habitats, pragas/doenças.\\[4pt]
\textit{Mudanças climáticas e redução de risco de desastres:} impactos de enchentes, deslizamentos, secas, ciclones/tempestades, emissões de GEE, risco de incêndio, velocidade do vento, microclima.

\medskip\noindent
\textit{Para cada impacto selecionado, classifique a magnitude em escala de 7 pontos (muito negativo a muito positivo) e quantifique antes/depois do MST quando possível.}

\subsubsection*{6.2\quad Impactos fora do local (\textit{off-site})}

Disponibilidade hídrica (lençol freático, nascentes), estabilidade de vazões, enchentes a jusante, assoreamento a jusante, poluição de águas subterrâneas/rios, capacidade de filtragem/tamponamento (solo, vegetação, áreas úmidas), sedimentos transportados pelo vento, danos a campos vizinhos, danos a infraestrutura pública/privada, impacto de gases de efeito estufa.

\subsubsection*{6.3\quad Exposição e sensibilidade da Tecnologia a mudanças climáticas graduais e extremos/desastres climáticos}

\textit{Indique mudanças graduais no clima e extremos/desastres observados pelos usuários nos últimos 10 anos.}

\medskip\noindent
\textbf{Mudanças climáticas graduais:} temperatura anual, temperatura sazonal, precipitação anual, precipitação sazonal.\\[4pt]
\textbf{Extremos climáticos (desastres):}\\
\textit{Meteorológicos:} tempestade tropical, ciclone extratropical, chuva intensa local, trovoada, granizo, tempestade de neve, tempestade de areia/poeira, vendaval, tornado.\\
\textit{Climatológicos:} onda de calor, onda de frio, condições extremas de inverno, seca, incêndio florestal, incêndio de campo.\\
\textit{Hidrológicos:} enchente fluvial, enxurrada, maré de tempestade, deslizamento/fluxo de detritos, avalanche.\\
\textit{Biológicos:} epidemias, infestações de insetos/vermes.\\[4pt]
\textit{Para cada evento, avalie a capacidade da Tecnologia de lidar com ele (muito precariamente a muito bem).}

\subsubsection*{6.4\quad Análise custo-benefício}

\textbf{Como os benefícios se comparam aos custos de implantação (perspectiva do usuário)?}\\
Retorno de curto prazo (1--3 anos):\quad $\bigcirc$ Muito negativo \quad $\bigcirc$ Negativo \quad $\bigcirc$ Levemente negativo \quad $\bigcirc$ Neutro \quad $\bigcirc$ Levemente positivo \quad $\bigcirc$ Positivo \quad $\bigcirc$ Muito positivo\\
Retorno de longo prazo ($>$10 anos):\quad $\bigcirc$ Muito negativo \quad $\bigcirc$ Negativo \quad $\bigcirc$ Levemente negativo \quad $\bigcirc$ Neutro \quad $\bigcirc$ Levemente positivo \quad $\bigcirc$ Positivo \quad $\bigcirc$ Muito positivo

\medskip\noindent
\textbf{Como os benefícios se comparam aos custos de manutenção (perspectiva do usuário)?}\\
Retorno de curto prazo:\quad $\bigcirc$ Muito negativo \quad ... \quad $\bigcirc$ Muito positivo\\
Retorno de longo prazo:\quad $\bigcirc$ Muito negativo \quad ... \quad $\bigcirc$ Muito positivo

\subsubsection*{6.5\quad Adoção da Tecnologia}

Quantos usuários da terra na área adotaram/implementaram a Tecnologia?\\
$\bigcirc$~Casos isolados/experimentais \quad $\bigcirc$~1--10\% \quad $\bigcirc$~10--50\% \quad $\bigcirc$~$>$50\%\\[4pt]
Se disponível, quantifique (nº de domicílios e/ou área coberta): \dotfill\\[4pt]
Dos que adotaram, quantos o fizeram espontaneamente (sem incentivos)?\\
$\bigcirc$~0--10\% \quad $\bigcirc$~10--50\% \quad $\bigcirc$~50--90\% \quad $\bigcirc$~90--100\%

\subsubsection*{6.6\quad Adaptação}

A Tecnologia foi modificada recentemente para adaptar-se a condições em mudança? $\bigcirc$~Sim \quad $\bigcirc$~Não\\[4pt]
Se sim, adaptada a:\\
$\bigcirc$~Mudanças/extremos climáticos \quad $\bigcirc$~Mercados em transformação \quad $\bigcirc$~Disponibilidade de mão de obra \quad $\bigcirc$~Outro: \dotfill\\
Especifique a adaptação: \dotfill

\subsubsection*{6.7\quad Pontos fortes/vantagens/oportunidades da Tecnologia}

\textit{Diferencie entre perspectivas de usuários e de pessoas-recurso.}\\[4pt]
\textbf{Perspectiva do usuário da terra:}\\
1)\,\dotfill \quad 2)\,\dotfill \quad 3)\,\dotfill \quad 4)\,\dotfill\\[4pt]
\textbf{Perspectiva do compilador/pessoa-recurso:}\\
1)\,\dotfill \quad 2)\,\dotfill \quad 3)\,\dotfill \quad 4)\,\dotfill

\subsubsection*{6.8\quad Pontos fracos/desvantagens/riscos e formas de superá-los}

\textbf{Perspectiva do usuário da terra:}\\
\begin{tabular}{p{7cm}p{7cm}}
Fraqueza/risco & Como superar? \\
1)\,\dotfill & 1)\,\dotfill \\
2)\,\dotfill & 2)\,\dotfill \\
3)\,\dotfill & 3)\,\dotfill \\
\end{tabular}

\medskip\noindent
\textbf{Perspectiva do compilador/pessoa-recurso:}\\
\begin{tabular}{p{7cm}p{7cm}}
Fraqueza/risco & Como superar? \\
1)\,\dotfill & 1)\,\dotfill \\
2)\,\dotfill & 2)\,\dotfill \\
3)\,\dotfill & 3)\,\dotfill \\
\end{tabular}

%% =========================================================
\subsection*{7\quad Referências e links}

\subsubsection*{7.1\quad Métodos/fontes de informação}
\textit{Várias respostas possíveis.}\\[4pt]
$\square$~Visitas/levantamentos de campo \quad Nº de informantes: \dotfill\\
$\square$~Entrevistas com usuários da terra \quad Nº de informantes: \dotfill\\
$\square$~Entrevistas com especialistas em MST \quad Nº de informantes: \dotfill\\
$\square$~Compilação de relatórios e documentação existente\\
$\square$~Outro (especifique): \dotfill\\[4pt]
Data da coleta de dados (em campo): \dotfill

\subsubsection*{7.2\quad Referências a publicações disponíveis}
\textit{Liste publicações relevantes. Envie cópias digitais ao banco de dados.}\\[4pt]
Título, autor, ano, ISBN: \dotfill\\
Disponível em: \dotfill\quad Custo: \dotfill

\subsubsection*{7.3\quad Links para informações relevantes disponíveis online}
Título/descrição: \dotfill\quad URL: \dotfill

\subsubsection*{7.4\quad Comentários gerais}
\dotfill\\[2pt]\dotfill

%% =========================================================
\subsection*{8\quad Anexo --- Listas de referência}

\textit{As listas abaixo são empregadas em campos de seleção do questionário (seções 3.2 e 3.3). Apresenta-se versão condensada.}

\begin{table}[htbp]
\centering\small
\caption{Culturas anuais (lista parcial WOCAT--IPCC)}
\label{tab:app_annual_crops}
\begin{tabular}{ll}
\toprule
\textbf{Grupo} & \textbf{Exemplos} \\
\midrule
Cereais & milho, arroz (várzea/sequeiro), sorgo, milheto, trigo, aveia, centeio, quinoa \\
Leguminosas e oleaginosas & feijão, lentilha, soja, ervilha, amendoim, girassol, mamona \\
Raízes e tubérculos & mandioca, batata, batata-doce, inhame, beterraba \\
Hortaliças & tomate, cebola, abóbora, berinjela, folhosas, cenoura \\
Fibras e flores & algodão, linho, cânhamo, roseiras \\
Forrageiras & alfafa, trevo, gramíneas \\
Medicinais/aromáticas & diversas \\
\bottomrule
\end{tabular}
\end{table}

\begin{table}[htbp]
\centering\small
\caption{Culturas perenes e arbóreo-arbustivas (lista parcial)}
\label{tab:app_perennial_crops}
\begin{tabular}{ll}
\toprule
\textbf{Grupo} & \textbf{Exemplos} \\
\midrule
Perenes não lenhosas & cana-de-açúcar, banana, abacaxi, sisal, capins forrageiros \\
Arbóreo-arbustivas & café, cacau, coco, dendê, manga, abacate, citros, castanhas \\
& oliveira, seringueira, teca, mogno, goiaba, maracujá \\
\bottomrule
\end{tabular}
\end{table}

\begin{table}[htbp]
\centering\small
\caption{Pecuária e produtos de pastagem}
\label{tab:app_livestock}
\begin{tabular}{ll}
\toprule
\textbf{Tipo} & \textbf{Produtos/serviços} \\
\midrule
Bovinos (leite, corte, trabalho) & carne, leite, couro, tração, esterco \\
Bubalinos, equinos, muares & transporte, tração \\
Suínos & carne \\
Caprinos, ovinos & carne, leite, lã, couro \\
Aves, coelhos & carne, ovos \\
Apicultura & mel, cera, pólen \\
Piscicultura & peixes \\
Fauna silvestre & herbívoros grandes/pequenos \\
\bottomrule
\end{tabular}
\end{table}
