
% Template for Elsevier CRC journal article
% version 1.2 dated 09 May 2011

% This file (c) 2009-2011 Elsevier Ltd.  Modifications may be freely made,
% provided the edited file is saved under a different name

% This file contains modifications for Procedia Computer Science
% but may easily be adapted to other journals

% Changes since version 1.1
% - added "procedia" option compliant with ecrc.sty version 1.2a
%   (makes the layout approximately the same as the Word CRC template)
% - added example for generating copyright line in abstract

%-----------------------------------------------------------------------------------

%% This template uses the elsarticle.cls document class and the extension package ecrc.sty
%% For full documentation on usage of elsarticle.cls, consult the documentation "elsdoc.pdf"
%% Further resources available at http://www.elsevier.com/latex

%-----------------------------------------------------------------------------------

%%%%%%%%%%%%%%%%%%%%%%%%%%%%%%%%%%%%%%%%%%%%%%%%%%%%%%%%%%%%%%
%%%%%%%%%%%%%%%%%%%%%%%%%%%%%%%%%%%%%%%%%%%%%%%%%%%%%%%%%%%%%%
%%                                                          %%
%% Important note on usage                                  %%
%% -----------------------                                  %%
%% This file should normally be compiled with PDFLaTeX      %%
%% Using standard LaTeX should work but may produce clashes %%
%%                                                          %%
%%%%%%%%%%%%%%%%%%%%%%%%%%%%%%%%%%%%%%%%%%%%%%%%%%%%%%%%%%%%%%
%%%%%%%%%%%%%%%%%%%%%%%%%%%%%%%%%%%%%%%%%%%%%%%%%%%%%%%%%%%%%%

%% The '3p' and 'times' class options of elsarticle are used for Elsevier CRC
%% Add the 'procedia' option to approximate to the Word template
%\documentclass[3p,times,procedia]{elsarticle}
\documentclass[3p,times]{elsarticle}

%% The `ecrc' package must be called to make the CRC functionality available
\usepackage{ecrc}
\usepackage[utf8]{inputenc}
\usepackage[T1]{fontenc}
\usepackage[brazil]{babel}
\usepackage{amsmath}
\usepackage{booktabs}
\usepackage{multirow}
\usepackage[breaklinks]{hyperref}
\usepackage{enumitem}
\usepackage{xcolor}
\usepackage{tikz}
\usetikzlibrary{shapes.geometric, arrows.meta, positioning, fit, backgrounds, calc}

%% The ecrc package defines commands needed for running heads and logos.
%% For running heads, you can set the journal name, the volume, the starting page and the authors

%% set the volume if you know. Otherwise `00'
\volume{00}

%% set the starting page if not 1
\firstpage{1}

%% Give the name of the journal
\journalname{Procedia Computer Science}

%% Give the author list to appear in the running head
%% Example \runauth{C.V. Radhakrishnan et al.}
\runauth{}

%% The choice of journal logo is determined by the \jid and \jnltitlelogo commands.
%% A user-supplied logo with the name <\jid>logo.pdf will be inserted if present.
%% e.g. if \jid{yspmi} the system will look for a file yspmilogo.pdf
%% Otherwise the content of \jnltitlelogo will be set between horizontal lines as a default logo

%% Give the abbreviation of the Journal.  Contact the journal editorial office if in any doubt
\jid{procs}

%% Give a short journal name for the dummy logo (if needed)
\jnltitlelogo{Procedia Computer Science}

%% Provide the copyright line to appear in the abstract
%% Usage:
%   \CopyrightLine[<text-before-year>]{<year>}{<restt-of-the-copyright-text>}
%   \CopyrightLine[Crown copyright]{2011}{Published by Elsevier Ltd.}
%   \CopyrightLine{2011}{Elsevier Ltd. All rights reserved}
\CopyrightLine{2011}{Published by Elsevier Ltd.}

%% Hereafter the template follows `elsarticle'.
%% For more details see the existing template files elsarticle-template-harv.tex and elsarticle-template-num.tex.

%% Elsevier CRC generally uses a numbered reference style
%% For this, the conventions of elsarticle-template-num.tex should be followed (included below)
%% If using BibTeX, use the style file elsarticle-num.bst

%% End of ecrc-specific commands
%%%%%%%%%%%%%%%%%%%%%%%%%%%%%%%%%%%%%%%%%%%%%%%%%%%%%%%%%%%%%%%%%%%%%%%%%%

%% The amssymb package provides various useful mathematical symbols
\usepackage{amssymb}
%% The amsthm package provides extended theorem environments
%% \usepackage{amsthm}

%% The lineno packages adds line numbers. Start line numbering with
%% \begin{linenumbers}, end it with \end{linenumbers}. Or switch it on
%% for the whole article with \linenumbers after \end{frontmatter}.
%% \usepackage{lineno}

%% natbib.sty is loaded by default. However, natbib options can be
%% provided with \biboptions{...} command. Following options are
%% valid:

%%   round  -  round parentheses are used (default)
%%   square -  square brackets are used   [option]
%%   curly  -  curly braces are used      {option}
%%   angle  -  angle brackets are used    <option>
%%   semicolon  -  multiple citations separated by semi-colon
%%   colon  - same as semicolon, an earlier confusion
%%   comma  -  separated by comma
%%   numbers-  selects numerical citations
%%   super  -  numerical citations as superscripts
%%   sort   -  sorts multiple citations according to order in ref. list
%%   sort&compress   -  like sort, but also compresses numerical citations
%%   compress - compresses without sorting
%%
%% \biboptions{comma,round}

% \biboptions{}

% if you have landscape tables
\usepackage[figuresright]{rotating}

% put your own definitions here:
%   \newcommand{\cZ}{\cal{Z}}
%   \newtheorem{def}{Definition}[section]
%   ...

% add words to TeX's hyphenation exception list
%\hyphenation{author another created financial paper re-commend-ed Post-Script}

% declarations for front matter

\begin{document}

\begin{frontmatter}

%% Title, authors and addresses

%% use the tnoteref command within \title for footnotes;
%% use the tnotetext command for the associated footnote;
%% use the fnref command within \author or \address for footnotes;
%% use the fntext command for the associated footnote;
%% use the corref command within \author for corresponding author footnotes;
%% use the cortext command for the associated footnote;
%% use the ead command for the email address,
%% and the form \ead[url] for the home page:
%%
%% \title{Title\tnoteref{label1}}
%% \tnotetext[label1]{}
%% \author{Name\corref{cor1}\fnref{label2}}
%% \ead{email address}
%% \ead[url]{home page}
%% \fntext[label2]{}
%% \cortext[cor1]{}
%% \address{Address\fnref{label3}}
%% \fntext[label3]{}

\dochead{}

\title{Adaptação Transcultural do Questionário WOCAT-SLM e Elicitação Estruturada de Saberes Agroecológicos Tradicionais em Comunidades Quilombolas do Semiárido Nordeste~II}

\author[a]{Catuxe Varjão de Santana Oliveira\corref{cor1}}
\ead{catuxe@academico.ufs.br}
\author[b]{Luiz Diego Vidal Santos}
\ead{ldvsantos@uefs.br}
\author[a]{XXXXXXX}

\cortext[cor1]{Autor correspondente.}
\address[a]{Programa de Pós-Graduação em Ciência da Propriedade Intelectual (PPGPI), Universidade Federal de Sergipe (UFS), São Cristóvão, SE, Brasil}
\address[b]{Universidade Estadual de Feira de Santana (UEFS), Feira de Santana, BA, Brasil}

\begin{abstract}
A adaptação transcultural do questionário WOCAT-SLM para comunidades quilombolas do Território Semiárido Nordeste~II foi concluída mediante protocolo ITC em seis etapas documentadas. Sessenta e oito itens distribuídos nas seções 2, 3, 5 e 6 foram traduzidos por dois especialistas bilíngues, reconciliados, retrotraduzidos e validados por comitê multidisciplinar com dez membros (três mestres de saberes quilombolas, três pesquisadores em agroecologia, dois psicometristas e dois gestores de PI). O Índice de Validade de Conteúdo (IVC) médio atingiu 0,93, o kappa de Fleiss permaneceu em 0,78 e a taxa de compreensão observada no pré-teste com 30 agricultores alcançou 87\%, consolidando a versão WOCAT-SLM-QBR e incorporando oito itens suplementares associados às dimensões espiritual-ritual, transmissão intergeracional e governança coletiva. A fase subsequente estruturou um painel Delphi com 21 participantes heterogêneos, conduzindo três rodadas com feedback controlado e triangulação com 18 entrevistas semiestruturadas. O processo estabilizou 26 variáveis linguísticas nas seis dimensões derivadas do WOCAT, com mediana $\geq 4$, $IQR \leq 1$, CVC de 0,84 e coeficiente de Kendall $W = 0,74$. As entrevistas confirmaram aderência ecológica das variáveis e indicaram correlação positiva ($r = 0,68$) entre consenso técnico e relevância percebida pelas comunidades. O encadeamento adaptação-Delphi estabeleceu cadeia rastreável de evidências para alimentar o Índice de Valoração Bioeconômica (IVB) e demonstra que plataformas globais de SLM podem ser recalibradas para salvaguardar ativos intangíveis de alta criticidade cultural.
\end{abstract}

\begin{keyword}
Adaptação transcultural \sep Método Delphi \sep WOCAT \sep Gestão sustentável da terra \sep Elicitação de variáveis \sep Comunidades quilombolas \sep Bioeconomia \sep Propriedade intelectual coletiva
\end{keyword}

\end{frontmatter}

% \linenumbers


\section{Introdução}
\label{sec:intro}

A valoração de Saberes e Sistemas Agrícolas Tradicionais (SSAT) em contextos de bioeconomia demanda cadeia metodológica articulada em duas etapas estruturantes que se complementam funcionalmente. A primeira requer instrumento de avaliação culturalmente equivalente, capaz de capturar as múltiplas dimensões dos sistemas produtivos tradicionais, enquanto a segunda exige a conversão de conhecimentos tácitos em variáveis mensuráveis, consensuadas e auditáveis, passíveis de alimentar modelos de valoração e sistemas de apoio à decisão \cite{Nonaka1995,CohenLevinthal1990}. O presente estudo integra essas duas etapas em protocolo sequencial cujo produto final sustenta a modelagem difusa do Índice de Valoração Bioeconômica (IVB), ferramenta de suporte à decisão projetada para mitigar a ambiguidade inerente à quantificação de ativos intangíveis em cenários de alta entropia informacional.

O questionário WOCAT (\textit{World Overview of Conservation Approaches and Technologies}), desenvolvido pelo Centre for Development and Environment (CDE) da Universidade de Berna e adotado como ferramenta oficial pela FAO, UNCCD e rede global de parceiros \cite{Liniger2019,Schwilch2012}, representa o framework mais consolidado para documentação de tecnologias SLM. Com aplicações em mais de 120 países e banco de dados contendo mais de 2.000 tecnologias catalogadas, o WOCAT oferece arquitetura padronizada de sete seções que abrangem desde a classificação técnica da tecnologia até a avaliação de impactos socioeconômicos, ecológicos e adaptativos, passando por análise de custos, ambiente biofísico e governança dos usuários da terra.

Contudo, a aplicação direta desse instrumento a comunidades tradicionais brasileiras, particularmente comunidades quilombolas do semiárido nordestino, esbarra em barreiras de equivalência que operam simultaneamente nos planos semântico-idiomático, experiencial e conceitual. No primeiro plano, o questionário original em inglês contém terminologia técnica agronômica sem paralelo direto no português vernacular de agricultores quilombolas, cujo vocabulário opera sob categorias etnotaxonômicas próprias \cite{Rist2006,Quave2014}, incompatibilidade que se amplifica no plano experiencial porque categorias de resposta pressupõem contextos fundiários formalizados, ao passo que comunidades quilombolas operam sob regimes coletivos de uso com reconhecimento jurídico precário \cite{Almeida2011}. De forma concomitante, o plano conceitual expõe a organização do WOCAT sob racionalidade agronômica ocidental que não contempla dimensões espirituais, rituais e simbólicas constitutivas da lógica de manejo quilombola \cite{Santos2007,Toledo2008}.

Essas barreiras não são meramente acadêmicas. A literatura sobre equivalência transcultural \cite{Herdman1999,Guillemin1993} demonstra que a aplicação de instrumentos sem adaptação formal produz viés sistemático de mensuração. No contexto desta pesquisa, esse risco assume magnitude crítica, pois o WOCAT-SLM servirá como template de referência para a elicitação Delphi e, indiretamente, para a construção do Índice de Valoração Bioeconômica (IVB) via lógica fuzzy. Um instrumento mal adaptado propagaria erros sistêmicos por toda a cadeia.

Simultaneamente, a economia do conhecimento contemporânea enfrenta paradoxo estrutural, pois, enquanto ativos intangíveis constituem o principal motor de competitividade organizacional \cite{Edvinsson1997,Pulic2000}, os SSAT (que atendem rigorosamente aos critérios VRIN de \cite{Barney1991}) permanecem excluídos dos circuitos formais de valoração econômica. Essa invisibilidade decorre de lacuna metodológica fundamental, qual seja, a ausência de protocolos validados para converter saberes tácitos, oralmente transmitidos e corporalmente incorporados \cite{Polanyi1966}, em variáveis mensuráveis e auditáveis.

Nesse cenário, o método Delphi \cite{Linstone1975,Rowe1999} emerge como técnica de elicitação particularmente adequada, dado que a operação sob anonimato e feedback controlado minimiza efeitos de dominância social enquanto a iteração estruturada produz convergência mensurável mediante coeficientes estatísticos \cite{Diamond2014}, configuração que assegura validade ecológica sobretudo quando painéis heterogêneos incluem detentores de saberes tradicionais \cite{Santos2007}. Aplicado sobre o instrumento previamente adaptado transculturalmente, o Delphi produz variáveis com dupla legitimidade, simultaneamente técnica (consenso estatístico interavaliadores) e cultural (ancoragem no WOCAT-SLM-QBR validado junto às comunidades).

A questão norteadora, que integra ambas as fases, indaga \textit{em que medida o processo combinado de adaptação transcultural e elicitação Delphi produz instrumento culturalmente equivalente e conjunto validado de variáveis mensuráveis, aumentando consenso e comparabilidade entre avaliadores?} A hipótese postula que a versão adaptada (WOCAT-SLM-QBR) atingirá IVC~$\geq$~0,80, kappa de Fleiss~$\geq$~0,70 e taxa de compreensão~$\geq$~80\%, gerando itens suplementares culturalmente específicos, e que a aplicação do método Delphi sobre as dimensões do instrumento adaptado produzirá índice de consenso estatístico ($W \geq 0,70$; $CVC \geq 0,80$) superior ao obtido por métodos não estruturados.


\section{Referencial Teórico}
\label{sec:theory}

\subsection{Teoria da Equivalência Transcultural de Instrumentos}

Transpor instrumentos entre contextos culturais exige procedimento que vai além da tradução linguística. O modelo hierárquico de \cite{Herdman1999} formaliza essa exigência ao estratificar a equivalência transcultural em seis níveis progressivos, da equivalência conceitual e de itens até a equivalência funcional, perpassando as dimensões semântica, operacional e de mensuração.

No cerne dessa estratificação situa-se a tensão entre abordagens \textit{etic} (universalista) e \textit{emic} (culturalmente específica) descrita por \cite{Pike1967}. Ao adotar perspectiva predominantemente \textit{etic}, o WOCAT pressupõe categorias universalmente aplicáveis, viabilizando comparabilidade internacional à custa de obscurecer categorias \textit{emic} significativas. Agricultores quilombolas, por exemplo, classificam terras por atributos espirituais ou memória social, categorias invisíveis ao instrumento original \cite{Toledo2008}. Preservar a dimensão \textit{etic} que confere comparabilidade e, concomitantemente, incorporar dimensões \textit{emic} que conferem validade ecológica define o duplo objetivo da adaptação transcultural.

Operacionalmente, o protocolo formalizado por \cite{Guillemin1993} e refinado por \cite{Beaton2000} exige tradução independente por múltiplos profissionais seguida de retrotradução cega, revisão por comitê multidisciplinar e multicultural e pré-teste com amostra da população-alvo.

\subsection{Conhecimento Tácito, Espiral SECI e Conversão de Saberes}

Desde \cite{Polanyi1966}, a fronteira entre conhecimento tácito e explícito orienta investigações sobre codificação de saberes. Em sistemas agroecológicos tradicionais, essa fronteira torna-se especialmente opaca, dado que manejo fenológico, leitura de sinais climáticos, seleção de variedades adaptadas e práticas de conservação de solos são transmitidos oralmente e pela prática cotidiana, resistindo à decomposição em categorias discretas \cite{Berkes2017,Toledo2008}.

Para modelar essa conversão progressiva, a espiral SECI de \cite{Nonaka1995} (Socialização, Externalização, Combinação e Internalização) oferece arcabouço conceitual robusto. No contexto deste estudo, a adaptação transcultural opera na transição entre Socialização e Externalização (traduzindo práticas locais em linguagem do instrumento), enquanto o Delphi opera na fase de \textit{externalização} avançada, onde julgamentos qualitativos são traduzidos em variáveis linguísticas calibradas e escalas padronizadas. Diferentemente da externalização espontânea, essa variante estruturada submete cada resultado a critérios de consenso verificáveis, conferindo auditabilidade ao processo de conversão.

\subsection{Fundamentos e Evolução do Método Delphi}

Desenvolvido pela RAND Corporation \cite{Linstone1975}, o Delphi evoluiu de exercício prospectivo militar para ferramenta consolidada de construção de consenso em domínios onde dados empíricos são escassos. Segundo \cite{Rowe1999}, quatro pilares sustentam essa capacidade, pois o anonimato elimina pressões de conformidade social, a iteração com feedback controlado induz convergência progressiva, a agregação estatística traduz opiniões em métricas e a heterogeneidade do painel amplia o espaço amostral de perspectivas.

Quando aplicado a conhecimentos tradicionais, o anonimato adquire função adicional ao neutralizar assimetrias de poder entre especialistas acadêmicos e detentores de saberes locais \cite{Hasson2000}. Em termos quantitativos, o consenso mobiliza conjunto integrado de indicadores, onde o coeficiente de concordância de Kendall ($W \geq 0,70$) sinaliza consenso forte \cite{Schmidt2014}, o intervalo interquartil ($IQR \leq 1$ em escala de 5 pontos) opera como limiar de estabilidade \cite{VonDerGracht2012} e o Coeficiente de Validade de Conteúdo ($CVC \geq 0,80$) certifica a pertinência das variáveis retidas \cite{Hernandez-Nieto2002}.

Articular adaptação transcultural e Delphi em cadeia sequencial representa contribuição metodológica original, visto que o instrumento adaptado provê template estruturado de dimensões validadas culturalmente, sobre o qual o Delphi opera para elicitar, refinar e consensuar variáveis específicas. Tal sequência evita tanto a imposição de categorias \textit{ad hoc} quanto a dispersão inerente a processos de elicitação sem ancoragem instrumental.

\subsection{O Framework WOCAT e Operacionalização das Dimensões}

O questionário WOCAT para Tecnologias SLM apresenta arquitetura modular em sete seções \cite{Liniger2019} que percorrem a cadeia completa de documentação, desde a identificação e localização da tecnologia (\S1) até o registro de fontes e instituições envolvidas (\S7). O núcleo avaliativo concentra-se nas seções intermediárias, onde a descrição técnica e classificação de medidas (\S2) alimenta a tipificação de uso da terra, degradação e função protetora (\S3), enquanto insumos e custos de estabelecimento e manutenção (\S4) são contextualizados pelo perfil biofísico e socioeconômico dos usuários (\S5). A análise convergente de impactos e custo-benefício (\S6) fecha o circuito avaliativo, consolidando evidências para tomada de decisão.

Para a adaptação transcultural (Fase~1), o foco recai sobre as seções 2, 3, 5 e 6, que contêm os construtos avaliativos diretamente relevantes. Para a elicitação Delphi (Fase~2), as seções WOCAT informam seis dimensões pré-estruturadas do painel, conforme mapeamento na Tabela~\ref{tab:wocat_delphi}.

\begin{table}[htbp]
\centering
\caption{Mapeamento entre seções do questionário WOCAT e dimensões pré-estruturadas do painel Delphi.}
\label{tab:wocat_delphi}
\small
\begin{tabular}{p{3.0cm}p{3.5cm}p{4.5cm}}
\toprule
\textbf{Dimensão Delphi} & \textbf{Seções WOCAT} & \textbf{Variáveis-chave deriváveis} \\
\midrule
Cultural-simbólica & §2 Descrição; §6.1 Impactos socioculturais & Autenticidade, significado ritual, transmissão intergeracional \\
Biofísica-ambiental & §3 Classificação; §5 Ambiente natural; §6.1 Impactos ecológicos & Agrobiodiversidade, resiliência edáfica, cobertura vegetal \\
Econômica-mercadológica & §4 Insumos e custos; §6.1 Impactos socioeconômicos & Custo de reposição, diversificação de renda, potencial de mercado \\
Institucional-governança & §5.6 Características; §5.8 Propriedade; §6.5 Adoção & Regime fundiário, organização comunitária, acesso a serviços \\
Adaptativa-resiliência & §3.8 Prevenção; §6.3 Exposição climática & Capacidade adaptativa, resposta a secas, estabilidade \\
Social-organizacional & §5.9 Infraestrutura; §6.1 Instituições comunitárias & Redes de cooperação, capital social, equidade de gênero \\
\bottomrule
\end{tabular}
\end{table}

\subsection{Comunidades Quilombolas e Especificidades Ontológicas dos SSAT}

Sob lógica estruturalmente distinta da racionalidade agronômica convencional, os SSAT quilombolas são governados pelo complexo Conhecimento-Prática-Crença (K-P-B) descrito por \cite{Toledo2008}, onde crenças funcionam como regulador ético-cosmológico do manejo. De forma complementar, \cite{Berkes2017} demonstra que esses sistemas exemplificam manejo adaptativo em que experimentação empírica é modulada por instituições sociais.

No contexto quilombola do semiárido, o WOCAT original falha em capturar dimensões constitutivas do sistema produtivo local. A esfera espiritual-ritual, que integra bênçãos sobre sementes, rituais de plantio sincronizados com ciclos lunares e proibições em datas sagradas, permanece invisível às categorias do instrumento, lacuna que se estende à transmissão intergeracional via oralidade, visto que o WOCAT documenta a tecnologia como produto acabado sem registrar o processo de transmissão oral que opera como mecanismo central de inovação dos SSAT \cite{Polanyi1966}. Adicionalmente, a lógica coletiva-comunitária (mutirões, trocas de sementes, manejo comunitário de áreas de uso comum) opera sob governança de bens comuns \cite{Ostrom1990} e escapa à arquitetura do instrumento, desenhada para documentar práticas individualizadas.

Com 18 municípios, o Território de Identidade Semiárido Nordeste~II (Bahia) abriga em Jeremoabo 11 comunidades quilombolas certificadas pela Fundação Cultural Palmares, mantenedoras de sistemas agroflorestais e práticas de manejo adaptados ao semiárido \cite{Altieri1995}. Essas comunidades enfrentam duplo desafio, pois a erosão acelerada dos saberes sob pressão de monoculturas coexiste com a inexistência de instrumentos formais para traduzir a sofisticação de seus sistemas em linguagem acessível a mercados e marcos regulatórios.

\subsection{Gestão do Conhecimento}

Converter saberes tácitos em variáveis mensuráveis pressupõe governança do conhecimento que assegure rastreabilidade, retenção e transferibilidade. A ISO~30401:2018 (Sistemas de Gestão do Conhecimento) reconhece que o valor do conhecimento depende de cultura, processos e aprendizagem, não apenas de codificação estática \cite{ISO30401}. Em comunidades quilombolas onde o saber tácito se concentra em poucos mestres, documentar via instrumento adaptado e protocolo Delphi assume caráter urgente de preservação.

A integração entre adaptação transcultural, Delphi e ISO~30401 materializa-se em planos complementares que asseguram rastreabilidade fim a fim, de modo que cada variável validada é acompanhada de definição operacional com origem documentável, o dossiê de adaptação registra integralmente o processo de externalização e a devolutiva às comunidades fecha o ciclo de internalização previsto no modelo SECI.

Sob a ótica da \textit{capacidade absortiva reversa}, são as instituições formais que precisam adaptar suas ferramentas para absorver o conhecimento das comunidades \cite{CohenLevinthal1990}. O instrumento adaptado materializa essa inversão, uma vez que não são os quilombolas que se ajustam ao WOCAT, mas o WOCAT que se reconfigura diante da realidade quilombola \cite{Santos2007}.


\section{Materiais e Métodos}
\label{sec:methods}

\subsection{Delineamento Geral}

Esta investigação configura-se como estudo metodológico de métodos mistos \cite{Creswell2018}, organizado em duas fases sequenciais integradas, sendo a Fase~1 dedicada à Adaptação Transcultural (pesquisa metodológica, protocolo de \cite{Beaton2000}) e a Fase~2 à Elicitação Delphi (estudo exploratório-descritivo com triangulação qualitativa). A integração segue delineamento sequencial, de modo que o instrumento adaptado na Fase~1 serve como template de referência para as dimensões avaliativas da Fase~2. A Figura~\ref{fig:flowchart} apresenta o fluxo geral do estudo.

\begin{figure*}[!htbp]
\centering
\begin{tikzpicture}[
  node distance=0.4cm and 0.2cm,
  every node/.style={font=\footnotesize},
  phase/.style={rectangle, rounded corners=3pt, draw=black!70, fill=black!8,
    minimum height=0.55cm, text width=2.1cm, align=center, font=\footnotesize\bfseries},
  stepbox/.style={rectangle, rounded corners=2pt, draw=black!60, fill=white,
    minimum height=0.7cm, text width=1.85cm, align=center, font=\scriptsize},
  output/.style={rectangle, rounded corners=2pt, draw=black!80, fill=black!12,
    minimum height=0.7cm, text width=2.0cm, align=center,
    font=\scriptsize\bfseries},
  arr/.style={-{Stealth[length=2.5pt]}, thick, black!65},
  dashline/.style={-{Stealth[length=2.5pt]}, thick, black!50, dashed},
  phaselabel/.style={font=\scriptsize\itshape, text=black!60}
]
% ---- FASE 1 ----
\node[phase] (f1) {Fase 1\\Adapta\c{c}\~ao Transcultural};
\node[stepbox, right=0.5cm of f1]  (e1) {Etapa 1\\Tradu\c{c}\~ao\\(T1 + T2)};
\node[stepbox, right=of e1]        (e2) {Etapa 2\\S\'intese\\(T-12)};
\node[stepbox, right=of e2]        (e3) {Etapa 3\\Retrotradu\c{c}\~ao\\(BT1 + BT2)};
\draw[arr] (f1) -- (e1);
\draw[arr] (e1) -- (e2);
\draw[arr] (e2) -- (e3);
% segunda linha fase 1
\node[stepbox, below=0.55cm of e1]  (e4) {Etapa 4\\Comit\^e\\(10 membros)};
\node[stepbox, right=of e4]        (e5) {Etapa 5\\Pr\'e-teste\\($n=30$)};
\node[output, right=of e5]      (e6) {Etapa 6\\WOCAT-SLM-QBR};
\draw[arr] (e3.south) -- ++(0,-0.275) -| (e4.north);
\draw[arr] (e4) -- (e5);
\draw[arr] (e5) -- (e6);
% ---- FASE 2 ----
\node[phase, below=1.0cm of f1 |- e4.south] (f2) {Fase 2\\Elicita\c{c}\~ao Delphi};
\node[stepbox, right=0.5cm of f2]  (r1) {Rodada 1\\Diverg\^encia\\(41 proposi\c{c}\~oes)};
\node[stepbox, right=of r1]        (r2) {Rodada 2\\Converg\^encia\\($Md\geq 4$)};
\node[stepbox, right=of r2]        (r3) {Rodada 3\\Consenso\\($W=0{,}74$)};
\draw[arr] (f2) -- (r1);
\draw[arr] (r1) -- (r2);
\draw[arr] (r2) -- (r3);
% segunda linha fase 2
\node[stepbox, below=0.55cm of r1]  (tr) {Triangula\c{c}\~ao\\Entrevistas\\($n=18$)};
\node[output, right=of tr]      (iv) {26 Vari\'aveis\\validadas $\rightarrow$ IVB};
\draw[arr] (r3.south) -- ++(0,-0.275) -| (tr.north);
\draw[arr] (tr) -- (iv);
% conector entre fases
\draw[dashline] (e6.south) -- ++(0,-0.35) -| (f2.north);
% backgrounds
\begin{scope}[on background layer]
  \node[draw=black!25, fill=blue!3, rounded corners=4pt,
    fit=(f1)(e1)(e2)(e3)(e4)(e5)(e6),
    inner xsep=4pt, inner ysep=6pt] {};
  \node[draw=black!25, fill=orange!3, rounded corners=4pt,
    fit=(f2)(r1)(r2)(r3)(tr)(iv),
    inner xsep=4pt, inner ysep=6pt] {};
\end{scope}
\end{tikzpicture}
\caption{Fluxograma do processo integrado de adapta\c{c}\~ao transcultural (Fase~1) e elicita\c{c}\~ao Delphi (Fase~2) do question\'ario WOCAT-SLM.}
\label{fig:flowchart}
\end{figure*}

% ============================================================
% FASE 1: ADAPTAÇÃO TRANSCULTURAL
% ============================================================
\subsection{Fase 1. Adaptação Transcultural do WOCAT-SLM}

\subsubsection{Etapa 1. Tradução Direta (Inglês $\rightarrow$ Português)}

Dois tradutores independentes realizaram a tradução integral do questionário WOCAT-SLM (seções 2, 3, 5 e 6) do inglês para o português brasileiro. O Tradutor~1 (T1), profissional com formação em ciências agrárias, bilíngue e ciente dos objetivos do estudo, priorizou equivalência técnica e terminológica. O Tradutor~2 (T2), profissional sem formação técnica na área, bilíngue e não informado dos objetivos, preservou linguagem coloquial e acessibilidade.

A divergência intencional entre perfis maximizou a detecção de ambiguidades \cite{Beaton2000}. Cada tradutor produziu versão independente (T1 e T2) acompanhada de relatório de decisões.

\subsubsection{Etapa 2. Síntese das Traduções (T-12)}

Os tradutores e um mediador produziram versão sintetizada (T-12). Discrepâncias foram resolvidas mediante negociação documentada, com itens não resolvidos encaminhados ao comitê de especialistas.

\subsubsection{Etapa 3. Retrotradução (Português $\rightarrow$ Inglês)}

Dois retrotradutores independentes, nativos de língua inglesa ou com proficiência C2, sem conhecimento do original, traduziram a T-12 de volta para o inglês (BT1 e BT2). As retrotraduções foram comparadas com o instrumento original item a item.

\subsubsection{Etapa 4. Comitê de Especialistas}

O comitê multidisciplinar reuniu dez membros cuja composição heterogênea garantiu avaliação multidimensional, com três mestres de saberes quilombolas (experiência mínima de 25 anos em sistemas agroecológicos tradicionais) responsáveis pela equivalência experiencial, três pesquisadores doutores em agroecologia e etnoecologia com trajetória participativa encarregados da pertinência científica, dois especialistas em psicometria e adaptação transcultural para assegurar o rigor do protocolo e dois gestores de PI e extensionistas voltados à perspectiva operacional e institucional.

Fundamentada no princípio de soberania epistêmica \cite{Santos2007} e nas diretrizes ITC \cite{ITC2017}, a presença de mestres de saberes como membros plenos do comitê rompeu a assimetria avaliativa convencional.

O comitê avaliou cada item em quatro dimensões de equivalência (escala de 4 pontos) compreendendo as facetas semântica, idiomática, experiencial e conceitual. A robustez da concordância foi quantificada pelo Índice de Validade de Conteúdo, definido na Equação~\ref{eq:cvi} como a razão entre avaliadores que atribuíram pontuação 3 ou 4 e o total de avaliadores.

\begin{equation}
IVC_{item} = \frac{\text{nº de avaliadores que atribuíram 3 ou 4}}{\text{nº total de avaliadores}}
\label{eq:cvi}
\end{equation}

Itens com $IVC \geq 0,80$ foram aceitos sem ajustes, itens no intervalo $0,60 \leq IVC < 0,80$ foram revisados conforme sugestões do comitê e itens com $IVC < 0,60$ foram reformulados ou excluídos. Dos 68 itens analisados, 55 permaneceram inalterados, 5 foram reescritos e 8 constituíram acréscimos culturalmente específicos. A concordância interavaliadores aferida por kappa de Fleiss \cite{Fleiss1971} atingiu 0,78. O comitê também identificou lacunas culturais e propôs os itens suplementares que migraram para a etapa de pré-teste.

\subsubsection{Etapa 5. Pré-Teste}

Versão pré-final foi aplicada a 30 agricultores quilombolas de Jeremoabo (BA), selecionados por amostragem intencional com variabilidade em idade, gênero, escolaridade e sistema produtivo. A aplicação ocorreu em formato de entrevista assistida e, após cada seção, conduziu-se \textbf{debriefing cognitivo} \cite{Willis2005} com perguntas padronizadas de compreensão, alternativas linguísticas e pertinência experiencial.

Os indicadores quantitativos do pré-teste registraram taxa de compreensão média de 87\%, taxa de não-resposta de 11\%, tempo médio de aplicação de 53 minutos e ausência de efeitos teto ou piso relevantes.

\subsubsection{Etapa 6. Consolidação}

A versão final WOCAT-SLM-QBR foi consolidada com dossiê completo de adaptação, compreendendo as versões T1, T2, T-12, BT1, BT2, atas do comitê, dados do pré-teste, manual de aplicação e a versão aprovada, com subsequente encaminhamento ao WOCAT Secretariat.

% ============================================================
% FASE 2: ELICITAÇÃO DELPHI
% ============================================================
\subsection{Fase 2. Elicitação Estruturada via Protocolo Delphi}

\subsubsection{Composição do Painel}

O painel reuniu 21 participantes selecionados por amostragem intencional \cite{Patton2015}, cuja heterogeneidade controlada combinou cinco mestres de saberes quilombolas (experiência média de 27 anos) para ancoragem ênica, seis pesquisadores doutores em agroecologia, etnoecologia, PI ou gestão da inovação para rigor analítico, cinco técnicos extensionistas com experiência mínima de 8 anos em assessoria a comunidades tradicionais para perspectiva operacional e cinco gestores de PI e bioeconomia vinculados a NITs, SEBRAE, INPI e secretarias territoriais para composição institucional.

Transversalmente, os critérios de elegibilidade demandaram experiência mínima de 5 anos, reconhecimento pela comunidade epistêmica ou territorial e disponibilidade para três rodadas em quatro meses. Incluir mestres de saberes como especialistas de pleno direito ancorou-se em \cite{Santos2007,Beaton2000}.

\subsubsection{Estrutura das Rodadas}

Na primeira rodada, dedicada à divergência e exploração, questionário aberto solicitou a enumeração de variáveis relevantes para valoração de ativos tradicionais, organizadas nas seis dimensões derivadas do WOCAT-SLM-QBR (Tabela~\ref{tab:wocat_delphi}). As contribuições orais dos mestres de saberes foram transcritas por facilitadores e a consolidação foi conduzida mediante análise de conteúdo \cite{Braun2006}, resultando em 41 proposições iniciais.

A segunda rodada operou na dimensão da convergência, com questionário estruturado avaliando cada proposição em escala Likert de 5 pontos para relevância, clareza e operacionalidade. Calcularam-se mediana ($Md$), intervalo interquartil ($IQR$), coeficiente de variação ($CV$) e frequência de respostas extremas, tendo 32 proposições atingido $Md \geq 4$ e $IQR \leq 1,5$.

A terceira rodada consolidou o consenso mediante reenvio com feedback agregado (medianas, distribuição, posicionamento individual anonimizado), permitindo o ajuste final. O consenso operacional adotou $IQR \leq 1,0$ e $Md \geq 4,0$. Vinte e seis variáveis cumpriram simultaneamente os critérios quantitativos e qualitativos, enquanto seis foram encaminhadas para deliberação qualitativa complementar.

\subsubsection{Análise Estatística do Consenso}

A convergência foi aferida por mediana e intervalo interquartil (IQR) por variável e por rodada, observando-se redução média de 42\% no IQR entre as rodadas 1 e 3. O coeficiente de concordância de Kendall ($W$) alcançou 0,74, classificando o consenso como forte \cite{Schmidt2014}. O Coeficiente de Validade de Conteúdo \cite{Hernandez-Nieto2002} permaneceu acima do limiar $CVC \geq 0,80$, com média de 0,84. A taxa de estabilidade entre rodadas 2 e 3 indicou que 81\% dos painelistas ajustaram suas respostas em no máximo $\pm 1$ ponto. Teste de Friedman ($\alpha = 0,05$) seguido de Dunn confirmou diferenças significativas entre as distribuições das rodadas 1 e 2, inexistindo diferenças entre as rodadas 2 e 3. 

Todas as análises foram conduzidas em ambiente R versão 4.5.1 \cite{RCore2024}, empregando o pacote \texttt{irr} para cômputo de $W$ de Kendall e coeficientes de concordância, o pacote \texttt{PMCMRplus} para o teste de Friedman com comparações \textit{post hoc} de Dunn ajustadas por Bonferroni e funções nativas do pacote \texttt{stats} para estatísticas descritivas, enquanto o $CVC$ foi calculado via rotina própria implementada conforme algoritmo de \cite{Hernandez-Nieto2002}.

Para testar diretamente a hipótese de superioridade do método estruturado, os índices de consenso Delphi foram comparados com levantamento qualitativo não estruturado (grupo focal com 9 especialistas do mesmo universo). A diferença observada em termos de variância residual foi significativa ($p < 0,01$) após 5.000 permutações, corroborando a eficiência do protocolo estruturado.

\subsection{Correspondência entre Variáveis Linguísticas e Conjuntos Fuzzy}

As 26 variáveis linguísticas estabilizadas pelo Delphi constituem os termos primários do sistema de inferência Mamdani que operacionaliza o IVB. A equivalência funcional entre o domínio empírico (escalas Likert consensuadas) e o domínio fuzzy (é estabelecida mediante mapeamento biunívoco, onde cada nível da escala (1 a 5) corresponde a um conjunto nebuloso (Muito Baixo, Baixo, Moderado, Alto, Muito Alto) com funções de pertinência triangulares sobrepostas em 25\% nos limites adjacentes. Essa sobreposição garante transição suave entre classes e preserva a granularidade das avaliações dos painelistas.

A calibração dos parâmetros de pertinência ($a$, $m$, $b$) para cada variável baseou-se nas distribuições observadas nas rodadas Delphi, de modo que o centroide de cada função triangular coincide com a mediana do painel e a abertura lateral reflete o intervalo interquartil. Essa conexão direta entre consenso especializado e topologia dos conjuntos nebulosos confere rastreabilidade ao modelo fuzzy e assegura que as regras SE-ENTÃO do IVB herdam a validade de conteúdo certificada no processo Delphi.

\subsection{Triangulação via Entrevistas Semiestruturadas}

Dezoito entrevistas com agricultores quilombolas de Jeremoabo, selecionados por saturação teórica \cite{Glaser1967}, complementaram os dados quantitativos. O roteiro abordou percepção sobre variáveis do Delphi, dimensões não contempladas, adequação da linguagem e hierarquização espontânea de prioridades.

As entrevistas foram gravadas em áudio, transcritas integralmente e submetidas a análise temática \cite{Braun2006} em cinco fases (familiarização, codificação aberta, busca por temas, revisão e redação). A codificação foi conduzida por dois pesquisadores independentes (kappa de Cohen = 0,72), e a triangulação foi operacionalizada via matriz de correspondência entre variáveis validadas e categorias temáticas emergentes.

\subsection{Aspectos Éticos}

O projeto foi aprovado pelo Comitê de Ética em Pesquisa da UFS (Resoluções CNS nº~466/2012 e nº~510/2016). Todos os participantes assinaram consentimento livre, prévio e informado. Os mestres de saberes integrantes do comitê foram reconhecidos como coautores do instrumento. Dados sensíveis foram tratados conforme Protocolo de Nagoia e Lei nº~13.123/2015, com devolutiva das sínteses às comunidades.


\section{Resultados e Discussão}
\label{sec:results_discussion}

\subsection{Adaptação Transcultural e Validação Psicométrica}

Como produto primário, obteve-se a versão WOCAT-SLM-QBR, instrumento adaptado transculturalmente para o contexto quilombola brasileiro contendo 68 itens traduzidos e oito itens suplementares culturalmente específicos. O IVC global atingiu 0,93, o kappa de Fleiss registrou 0,78 e a taxa de compreensão aferida no pré-teste permaneceu em 87\%. O dossiê de adaptação, com 142 páginas de rastreabilidade (relatórios T1/T2, retrotraduções, atas do comitê, planilhas do pré-teste e manual de aplicação), tornou-se referência replicável para outros contextos de comunidades tradicionais brasileiras.

Quanto às lacunas culturais, o mapeamento confirmou que as dimensões espiritual-ritual, transmissão intergeracional via oralidade e coletividade associada a bens comuns não são contempladas pelo WOCAT original. Os itens suplementares relativos a essas dimensões obtiveram $IVC = 0,91$, $kappa = 0,76$ e estabilidade semântica após o pré-teste. Os oito itens emergentes reforçam a limitação inerente a abordagens puramente \textit{etic}, evidenciando que frameworks universalistas carregam pressupostos culturais que operam como ``pontos cegos'' quando transplantados para ontologias distintas \cite{Herdman1999}.

\subsection{Elicitação Delphi e Convergência Estatística}

O protocolo Delphi estabilizou 26 variáveis linguísticas distribuídas nas seis dimensões derivadas do WOCAT, com definições operacionais consensuadas e escalas padronizadas. Cada variável apresenta ficha técnica contendo estatísticas ($Md$, $IQR$, $W$, $CVC$) e mapeamento para os indicadores do Índice de Valoração Bioeconômica (IVB).

Frente ao grupo focal não estruturado, o método Delphi alcançou coeficientes de concordância significativamente mais elevados ($W_{Delphi}=0,74$ versus $W_{GF}=0,41$) e redução de 36\% na variância das respostas, corroborando a eficiência do protocolo iterativo com feedback controlado. Essa integração sequencial entre adaptação transcultural e elicitação Delphi responde a lacuna identificada tanto na gestão da inovação \cite{Tidd2005} quanto na economia de ativos intangíveis \cite{Lev2001}, dado que inexistia cadeia metodológica conectando rigor instrumental com consenso auditável sem sacrificar legitimidade cultural junto aos detentores dos saberes. Pelo prisma da Teoria dos Recursos da Firma \cite{Barney1991}, as variáveis elicitadas operacionalizam os atributos VRIN em dimensões mensuráveis, viabilizando que comunidades quilombolas demonstrem o valor estratégico de seus ativos intangíveis, pré-requisito para negociações de repartição de benefícios, certificação de produtos e proteção jurídica via indicações geográficas ou marcas coletivas \cite{Belletti2015}.

\subsection{Triangulação e Validade Ecológica}

Cruzando consenso técnico (Delphi) e percepção comunitária (entrevistas), a matriz de correspondência evidenciou correlação de Pearson $r=0,68$ ($p<0,01$), confirmando validade ecológica e indicando que o consenso especializado preserva coerência com as prioridades percebidas pelas comunidades. Ter incorporado mestres de saberes quilombolas como membros plenos tanto do comitê de adaptação quanto do painel Delphi configura inovação metodológica alinhada ao paradigma da soberania epistêmica \cite{Santos2007}, na medida em que os detentores de saberes tradicionais passam a coautores do instrumento e do consenso, exercendo agência sobre como sua realidade é representada e mensurada.

\subsection{Implicações para Governança do Conhecimento e Modelagem Fuzzy}

A articulação entre protocolo de adaptação e princípios ISO~30401 (Gestão do Conhecimento) oferece contribuição teórica à literatura de gestão da propriedade intelectual em contextos comunitários. Ao documentar cada decisão com rastreabilidade, o dossiê de adaptação cria infraestrutura de metadados que atende simultaneamente requisitos de governança do conhecimento \cite{ISO30401} e demandas de proteção de conhecimentos tradicionais associados à biodiversidade, funcionalidade dual que posiciona o estudo na interface entre psicometria transcultural e gestão estratégica de PI \cite{Teece1986,ISO56005}.

Na arquitetura mais ampla do programa de pesquisa, o presente estudo opera como fundação metodológica, uma vez que o instrumento culturalmente calibrado e as variáveis consensuadas alimentarão diretamente as funções de pertinência e regras SE-ENTÃO do sistema fuzzy Mamdani. Cada variável do IVB terá origem documentada em adaptação transcultural e consenso especializado, conferindo rastreabilidade ao modelo \cite{Costanza1997} e atendendo à cadeia de evidências que se estende do WOCAT original ao WOCAT-SLM-QBR, deste ao consenso Delphi e, finalmente, ao modelo fuzzy. O manual de operacionalização do protocolo integrado adaptação-Delphi documenta cada decisão crítica com granularidade suficiente para replicação independente.


\section{Considerações Finais}
\label{sec:conclusion}

Este estudo concluiu a adaptação transcultural sistemática do questionário WOCAT-SLM para comunidades quilombolas brasileiras mediante protocolo de seis etapas complementado por diretrizes ITC e princípios de pesquisa-ação participativa, seguida de elicitação estruturada via Delphi e triangulação qualitativa. O WOCAT-SLM-QBR foi disponibilizado com métricas psicométricas robustas (IVC = 0,93, kappa = 0,78, compreensão = 87\%) e oito itens suplementares que preservam comparabilidade internacional sem suprimir especificidades quilombolas, acompanhado de portfólio contendo 26 variáveis linguísticas consensuadas ($W = 0,74$, $CVC = 0,84$) destinadas ao IVB. O protocolo integrado adaptação-Delphi, documentado com checklists, templates de feedback e scripts estatísticos, oferece referência replicável às demais comunidades tradicionais brasileiras, enquanto a evidência empírica dos ``pontos cegos'' culturais de frameworks universalistas de SLM operacionaliza o conceito de soberania epistêmica em instrumentos de mensuração. Esse arcabouço estabelece a camada fundacional de um sistema de governança bioeconômica onde mensuração, elicitação estruturada e modelagem computacional compartilham origem comum culturalmente validada.

A versão WOCAT-SLM-QBR e seu dossiê completo de adaptação foram submetidos ao Secretariado do WOCAT para incorporação à rede global de adaptações regionais, contribuindo para a internacionalização dos saberes agroecológicos quilombolas brasileiros em framework que garanta simultaneamente rigor científico e soberania epistêmica.


%% References with BibTeX database:

\bibliographystyle{elsarticle-num}
\bibliography{references}

\end{document}