
% Template for Elsevier journal article Ecological Economics
% version 1.2 dated 09 May 2011

% This file (c) 2009-2011 Elsevier Ltd.  Modifications may be freely made,
% provided the edited file is saved under a different name

% This file contains modifications for Ecological Economics
% but may easily be adapted to other journals

% Changes since version 1.1
% - added "procedia" option compliant with ecrc.sty version 1.2a
%   (makes the layout approximately the same as the Word CRC template)
% - added example for generating copyright line in abstract

%-----------------------------------------------------------------------------------

%% This template uses the elsarticle.cls document class and the extension package ecrc.sty
%% For full documentation on usage of elsarticle.cls, consult the documentation "elsdoc.pdf"
%% Further resources available at http://www.elsevier.com/latex

%-----------------------------------------------------------------------------------

%%%%%%%%%%%%%%%%%%%%%%%%%%%%%%%%%%%%%%%%%%%%%%%%%%%%%%%%%%%%%%
%%%%%%%%%%%%%%%%%%%%%%%%%%%%%%%%%%%%%%%%%%%%%%%%%%%%%%%%%%%%%%
%%                                                          %%
%% Important note on usage                                  %%
%% -----------------------                                  %%
%% This file should normally be compiled with PDFLaTeX      %%
%% Using standard LaTeX should work but may produce clashes %%
%%                                                          %%
%%%%%%%%%%%%%%%%%%%%%%%%%%%%%%%%%%%%%%%%%%%%%%%%%%%%%%%%%%%%%%
%%%%%%%%%%%%%%%%%%%%%%%%%%%%%%%%%%%%%%%%%%%%%%%%%%%%%%%%%%%%%%

%% The '3p' and 'times' class options of elsarticle are used for Elsevier CRC
%% Add the 'procedia' option to approximate to the Word template
%\documentclass[3p,times,procedia]{elsarticle}
\documentclass[3p,times]{elsarticle}

%% The `ecrc' package must be called to make the CRC functionality available
\usepackage{ecrc}
\usepackage[utf8]{inputenc}
\usepackage[T1]{fontenc}
\usepackage[brazil]{babel}
\usepackage{amsmath}
\usepackage{booktabs}
\usepackage{multirow}
\usepackage[breaklinks]{hyperref}
\usepackage{enumitem}
\usepackage{xcolor}
\usepackage{tikz}
\usetikzlibrary{shapes.geometric, arrows.meta, positioning, fit, backgrounds, calc}

%% The ecrc package defines commands needed for running heads and logos.
%% For running heads, you can set the journal name, the volume, the starting page and the authors

%% set the volume if you know. Otherwise `00'
\volume{00}

%% set the starting page if not 1
\firstpage{1}

%% Give the name of the journal
\journalname{Ecological Economics}

%% Give the author list to appear in the running head
%% Example \runauth{C.V. Radhakrishnan et al.}
\runauth{Oliveira et al.}

%% The choice of journal logo is determined by the \jid and \jnltitlelogo commands.
%% A user-supplied logo with the name <\jid>logo.pdf will be inserted if present.
%% e.g. if \jid{yspmi} the system will look for a file yspmilogo.pdf
%% Otherwise the content of \jnltitlelogo will be set between horizontal lines as a default logo

%% Give the abbreviation of the Journal.  Contact the journal editorial office if in any doubt
\jid{ecolecon}

%% Give a short journal name for the dummy logo (if needed)
\jnltitlelogo{Ecological Economics}

%% Provide the copyright line to appear in the abstract
%% Usage:
%   \CopyrightLine[<text-before-year>]{<year>}{<restt-of-the-copyright-text>}
%   \CopyrightLine[Crown copyright]{2011}{Published by Elsevier Ltd.}
%   \CopyrightLine{2011}{Elsevier Ltd. All rights reserved}
\CopyrightLine{2026}{Published by Elsevier B.V. All rights reserved.}

%% Hereafter the template follows `elsarticle'.
%% For more details see the existing template files elsarticle-template-harv.tex and elsarticle-template-num.tex.

%% Elsevier CRC generally uses a numbered reference style
%% For this, the conventions of elsarticle-template-num.tex should be followed (included below)
%% If using BibTeX, use the style file elsarticle-num.bst

%% End of ecrc-specific commands
%%%%%%%%%%%%%%%%%%%%%%%%%%%%%%%%%%%%%%%%%%%%%%%%%%%%%%%%%%%%%%%%%%%%%%%%%%

%% The amssymb package provides various useful mathematical symbols
\usepackage{amssymb}
%% The amsthm package provides extended theorem environments
%% \usepackage{amsthm}

%% The lineno packages adds line numbers. Start line numbering with
%% \begin{linenumbers}, end it with \end{linenumbers}. Or switch it on
%% for the whole article with \linenumbers after \end{frontmatter}.
\usepackage{lineno}

%% natbib.sty is loaded by default. However, natbib options can be
%% provided with \biboptions{...} command. Following options are
%% valid:

%%   round  -  round parentheses are used (default)
%%   square -  square brackets are used   [option]
%%   curly  -  curly braces are used      {option}
%%   angle  -  angle brackets are used    <option>
%%   semicolon  -  multiple citations separated by semi-colon
%%   colon  - same as semicolon, an earlier confusion
%%   comma  -  separated by comma
%%   numbers-  selects numerical citations
%%   super  -  numerical citations as superscripts
%%   sort   -  sorts multiple citations according to order in ref. list
%%   sort&compress   -  like sort, but also compresses numerical citations
%%   compress - compresses without sorting
%%
%% \biboptions{comma,round}

% \biboptions{}

% \usepackage[figuresright]{rotating}  % removido: sem tabelas paisagem

% put your own definitions here:
%   \newcommand{\cZ}{\cal{Z}}
%   \newtheorem{def}{Definition}[section]
%   ...

% add words to TeX's hyphenation exception list
%\hyphenation{author another created financial paper re-commend-ed Post-Script}

% declarations for front matter

\begin{document}

\begin{frontmatter}

%% Title, authors and addresses

%% use the tnoteref command within \title for footnotes;
%% use the tnotetext command for the associated footnote;
%% use the fnref command within \author or \address for footnotes;
%% use the fntext command for the associated footnote;
%% use the corref command within \author for corresponding author footnotes;
%% use the cortext command for the associated footnote;
%% use the ead command for the email address,
%% and the form \ead[url] for the home page:
%%
%% \title{Title\tnoteref{label1}}
%% \tnotetext[label1]{}
%% \author{Name\corref{cor1}\fnref{label2}}
%% \ead{email address}
%% \ead[url]{home page}
%% \fntext[label2]{}
%% \cortext[cor1]{}
%% \address{Address\fnref{label3}}
%% \fntext[label3]{}

\dochead{}

\title{Indicador de resiliência biocultural para comunidades quilombolas via WOCAT-SLM adaptado}

\author[a]{Catuxe Varjão de Santana Oliveira\corref{cor1}}
\ead{catuxe@academico.ufs.br}
\author[b]{Luiz Diego Vidal Santos}
\ead{ldvsantos@uefs.br}
\author[a]{XXXXXXX}

\cortext[cor1]{Autor correspondente.}
\address[a]{Programa de Pós-Graduação em Ciência da Propriedade Intelectual (PPGPI), Universidade Federal de Sergipe (UFS), São Cristóvão, SE, Brasil}
\address[b]{Universidade Estadual de Feira de Santana (UEFS), Feira de Santana, BA, Brasil}

\begin{abstract}
Indicadores convencionais de sustentabilidade raramente capturam sinergias entre conservação do solo, coesão sociocultural e serviços ecossistêmicos em territórios tradicionais. Este estudo desenvolve e valida o Índice de Resiliência Biocultural Integrada (IRBI) mediante cadeia metodológica em três fases aplicada a Comunidades Quilombolas (CQ) do semiárido baiano. Na Fase~1, o questionário WOCAT-SLM foi adaptado transculturalmente via protocolo ITC, gerando a versão WOCAT-SLM-QBR com 68 itens traduzidos e oito suplementares (IVC = 0,93, kappa = 0,78, compreensão = 87\%). Na Fase~2, painel Delphi ($n=21$) estabilizou 26 variáveis linguísticas em seis dimensões ($W = 0,74$, $CVC = 0,84$), com triangulação por 18 entrevistas ($r = 0,68$, $p < 0,01$). Na Fase~3, as variáveis foram mapeadas em funções de pertinência triangulares para sistema Mamdani, verificado por análise de sensibilidade global (Morris). O IRBI integra dimensões biofísicas, socioculturais, econômicas, institucionais, adaptativas e organizacionais em índice composto apto a monitoramento periódico. A análise de sensibilidade confirmou que a variância não é monopolizada por parâmetros biofísicos, preservando peso equivalente do componente cultural.
\end{abstract}

\begin{keyword}
Biocultural valuation \sep Socioecological indicator \sep Fuzzy inference \sep Cross-cultural adaptation \sep Delphi method \sep Quilombola communities \sep Traditional ecological knowledge
\end{keyword}

\end{frontmatter}

\linenumbers


\section{Introdução}
\label{sec:intro}

Indicadores convencionais de sustentabilidade apreendem variáveis biofísicas isoladas sem capturar as sinergias entre conservação do solo, coesão sociocultural e serviços ecossistêmicos que caracterizam sistemas agroflorestais de Comunidades Quilombolas (CQ) \citep{Costanza1997,Dale2001}. Essa insuficiência compromete tanto a gestão sustentável da terra (SLM) quanto a formulação de políticas de proteção do patrimônio biocultural, dado que decisões baseadas em indicadores parciais subestimam o capital cultural como componente de resiliência \citep{Berkes2017}. Em sistemas quilombolas, o componente biofísico representa apenas um vetor dentro de matriz multidimensional em que cultura, governança e economia desempenham papéis equivalentes na manutenção da resiliência socioecológica \citep{Toledo2008}.


O questionário WOCAT (\textit{World Overview of Conservation Approaches and Technologies}), adotado pela FAO e UNCCD \citep{Liniger2019,Schwilch2012}, constitui o framework mais consolidado para documentação de tecnologias SLM, com aplicações em mais de 120 países. Contudo, a aplicação direta desse instrumento a CQ do semiárido nordestino esbarra em barreiras de equivalência nos planos semântico, experiencial e conceitual \citep{Herdman1999}. No plano semântico, terminologia agronômica sem paralelo no vocabulário etnotaxonômico quilombola compromete a compreensão \citep{Rist2006,Quave2014}. No plano experiencial, categorias de resposta pressupõem contextos fundiários formalizados, incompatíveis com regimes coletivos de reconhecimento jurídico precário \citep{Almeida2011}. No plano conceitual, a racionalidade agronômica ocidental não contempla dimensões espirituais e simbólicas constitutivas do manejo quilombola \citep{Santos2007,Toledo2008}.

Aplicar instrumentos sem adaptação formal produz viés sistemático de mensuração \citep{Guillemin1993}, risco relevante neste estudo porque o WOCAT-SLM adaptado servirá como template para elicitação Delphi e calibração fuzzy do IRBI. A Fase~1 adapta transculturalmente o WOCAT-SLM via protocolo ITC \citep{Beaton2000}, a Fase~2 elicita e consensua 26 variáveis linguísticas mediante Delphi estruturado com triangulação qualitativa e a Fase~3 mapeia as variáveis em funções de pertinência para sistema de inferência Mamdani, com verificação por análise de sensibilidade global.

A questão norteadora formulou-se nos seguintes termos, em que medida a integração sequencial de adaptação transcultural, Delphi e lógica difusa produz indicador socioecológico composto capaz de capturar a multidimensionalidade de sistemas agroflorestais quilombolas para fins de monitoramento e gestão adaptativa?.

Neste sentido, as hipóteses levantadas e a priori, admitindas foram que que a versão adaptada (WOCAT-SLM-QBR) apresentaria validade de conteúdo e concordância interavaliadores adequadas, que o Delphi alcançaria consenso forte a partir do instrumento proposto e que o IRBI apresentaria robustez numérica verificável por análise de sensibilidade global, com variância não monopolizada por parâmetros biofísicos.

Nesse contexto, o presente estudo propõe e valida o Índice de Resiliência Biocultural Integrada (IRBI), indicador socioecológico composto projetado para traduzir o conhecimento tácito de CQ em variáveis mensuráveis, monitoráveis e compatíveis com sistemas de informação geográfica. O IRBI integra dimensões biofísicas, socioculturais, econômicas, institucionais, adaptativas e organizacionais em índice único, apto a monitoramento periódico.


\section{Referencial Teórico}
\label{sec:theory}

\subsection{Equivalência Transcultural e Validade de Construto}

Transpor instrumentos entre contextos culturais exige procedimento que vai além da tradução linguística. O modelo hierárquico de \citet{Herdman1999} formaliza essa exigência ao estratificar a equivalência transcultural em seis níveis progressivos, da equivalência conceitual e de itens até a equivalência funcional, perpassando as dimensões semântica, operacional e de mensuração. A teoria psicométrica clássica, consolidada por \citet{Nunnally1978}, postula que a validade de um instrumento repousa sobre três pilares interdependentes (conteúdo, construto e critério), cujo atendimento torna-se exponencialmente complexo quando o objeto mensurado é culturalmente contingente, como ocorre com saberes tradicionais que resistem à decomposição em itens discretos \citep{Polanyi1966}.

No cerne dessa complexidade situa-se a tensão entre abordagens \textit{etic} (universalista) e \textit{emic} (culturalmente específica) descrita por \citet{Pike1967}. Instrumentos com orientação predominantemente \textit{etic}, como o WOCAT, pressupõem categorias universalmente aplicáveis que viabilizam comparabilidade internacional à custa de obscurecer categorias \textit{emic} significativas. Agricultores quilombolas, por exemplo, classificam terras por atributos espirituais ou memória social, categorias invisíveis ao instrumento original \citep{Toledo2008}. Preservar a dimensão \textit{etic} que confere comparabilidade e, concomitantemente, incorporar dimensões \textit{emic} que conferem validade ecológica define o duplo desafio conceitual subjacente à adaptação de instrumentos entre epistemologias distintas.

Essa dualidade remete ao problema fundamental da comensurabilidade entre paradigmas, formulado por \citet{Kuhn1962} e revisitado na literatura sobre pluralismo epistemológico em ciência da sustentabilidade \citep{Miller2008}. A questão não é meramente técnica (traduzir termos), mas ontológica (negociar o que conta como conhecimento válido entre matrizes culturais que operam com categorias parcialmente sobrepostas e parcialmente incomensuráveis). A equivalência funcional, nível mais elevado do modelo de \citet{Herdman1999}, somente se verifica quando o instrumento adaptado desempenha papel análogo ao original na cultura de destino, critério que, no caso de sistemas socioecológicos complexos, implica que o instrumento deve capturar não apenas variáveis biofísicas mas também a rede de significados culturais que confere coerência ao sistema de manejo \citep{Guillemin1993,Beaton2000}.

\subsection{Epistemologia do Conhecimento Tácito e Conversão entre Modos de Saber}

A distinção entre conhecimento tácito e explícito, formalizada por \citet{Polanyi1966} sob a máxima ``sabemos mais do que podemos dizer'' (\textit{we can know more than we can tell}), orienta investigações sobre codificação de saberes em múltiplos domínios. Em sistemas agroecológicos tradicionais essa fronteira torna-se particularmente opaca, dado que manejo fenológico, leitura de sinais climáticos e seleção de variedades são transmitidos oralmente e pela prática cotidiana \citep{Berkes2017,Toledo2008}.

A espiral SECI de \citet{Nonaka1995} (Socialização, Externalização, Combinação e Internalização) modela a conversão progressiva entre modos de conhecimento. Para saberes tradicionais, a transição Socialização--Externalização constitui o gargalo epistêmico central, porquanto categorias experienciais (percepção tátil do solo, leitura de nuvens, reconhecimento de fenofases) precisam ser articuladas em linguagem padronizada sem perda de significado substancial. \citet{Davenport1998} argumentam que o conhecimento, diferentemente da informação, é contextual, experiencial e mediado por julgamento, propriedades que dificultam sua transferência por meios puramente documentais. A ISO~30401:2018 reconhece que o valor do conhecimento depende de cultura, processos e aprendizagem organizacional \citep{ISO30401}, perspectiva que amplia a gestão do conhecimento para além do ambiente corporativo.

Nesse marco, a noção de \textit{capacidade absortiva} de \citet{CohenLevinthal1990} opera em sentido inverso ao convencional, dado que são as instituições formais que precisam absorver o conhecimento das comunidades tradicionais, e não o contrário \citep{Santos2007}. A adaptação de instrumentos de avaliação não constitui simplificação, mas reconhecimento de que o conhecimento relevante para mensuração de capital biocultural encontra-se nos detentores de saberes tácitos \citep{FalsBorda1991}.

\subsection{Ativos Intangíveis, Economia Ecológica e Capital Biocultural}

Ativos intangíveis superam os tangíveis como fator de criação de riqueza \citep{Lev2001}. \citet{Sveiby1997} formalizou a tipologia do capital intelectual em três componentes (humano, estrutural e relacional), enquanto \citet{Edvinsson1997} demonstrou que a contabilidade tradicional falha em capturar esses ativos. Quando transposta para CQ, essa invisibilidade contábil torna-se estrutural, visto que saberes de manejo, variedades agrícolas selecionadas ao longo de gerações e instituições comunitárias de governança são, por definição, intangíveis e não monetizados.

A economia ecológica oferece marcos complementares. \citet{Costanza1997} evidenciaram que a riqueza natural constitui fluxo indispensável à economia humana mesmo quando invisível ao PIB, \citet{PearceTurner1990} formalizaram o Valor Econômico Total (VET) e a iniciativa TEEB demonstrou que a não valoração gera custos superiores aos da conservação \citep{TEEB2010}. O conceito de capital biocultural \citep{Maffi2001,Pretty2009} designa o acervo integrado de diversidade biológica e cultural co-evoluído entre comunidades e ecossistemas, incorporando dimensão epistêmica (saberes), normativa (instituições e rituais) e relacional (redes de troca e cooperação). \citet{MartinezAlier2002} argumenta que colapsar essas dimensões em unidade monetária única apaga comensuralidades irredutíveis, razão pela qual abordagens de indicadores compostos com modelagem de incerteza tornam-se conceitualmente necessárias.

\subsection{O Framework WOCAT: Arquitetura Conceitual e Potencial Analítico}

O \textit{World Overview of Conservation Approaches and Technologies} (WOCAT) constitui o principal framework internacional para documentação padronizada de Tecnologias de Manejo Sustentável da Terra (\textit{Sustainable Land Management}, SLM), desenvolvido pelo Centre for Development and Environment da Universidade de Berna e endossado pela Convenção das Nações Unidas de Combate à Desertificação (UNCCD) como ferramenta oficial de sistematização de boas práticas \citep{Liniger2019}. Desde sua criação em 1992, o WOCAT acumulou mais de 2.000 tecnologias documentadas em 120 países, consolidando-se como o mais amplo repositório global de evidências sobre conservação de solos e água em contextos de degradação da terra. A relevância do framework para a economia ecológica reside precisamente nessa capacidade de converter práticas locais de manejo, frequentemente não monetizadas e invisíveis às contas nacionais, em registros sistematizados que viabilizam análises comparativas de custo-benefício e impacto socioambiental.

A arquitetura modular do questionário WOCAT para Tecnologias SLM organiza-se em sete seções funcionalmente encadeadas que percorrem a cadeia completa, desde a identificação e localização georreferenciada da tecnologia (\S1) até o registro de fontes, instituições e processos de governança envolvidos (\S7). O núcleo conceitual do instrumento concentra-se nas seções intermediárias, onde a descrição técnica e classificação de medidas (\S2) fornece a base taxonômica que alimenta a tipificação de uso da terra, processos de degradação e funções protetoras (\S3), enquanto a contabilização de insumos e custos de estabelecimento e manutenção (\S4) articula-se com o perfil biofísico e socioeconômico dos usuários e do ambiente natural (\S5). A análise convergente de impactos ecológicos, socioeconômicos e socioculturais, combinada com avaliação de custo-benefício (\S6), fecha o circuito avaliativo e gera evidências para tomada de decisão em múltiplas escalas.

Essa organização confere ao WOCAT uma propriedade analítica frequentemente subutilizada na literatura, pois a cobertura simultânea de dimensões técnicas, ecológicas, econômicas e institucionais permite derivar construtos avaliativos multidimensionais que transcendem a finalidade original de documentação de tecnologias isoladas. A Tabela~\ref{tab:wocat_delphi} explicita o campo de construtos teoricamente deriváveis da arquitetura modular, organizados em seis dimensões complementares que abrangem desde a esfera cultural-simbólica até a capacidade adaptativa e a organização social, evidenciando que o instrumento contém, de forma latente, a matéria-prima conceitual para composição de indicadores socioecológicos integrados de valoração biocultural.

\begin{table}[htbp]
\centering
\caption{Construtos deriváveis da arquitetura modular WOCAT para avaliação biocultural multidimensional.}
\label{tab:wocat_delphi}
\small
\begin{tabular}{p{3.0cm}p{3.5cm}p{4.5cm}}
\toprule
\textbf{Dimensão avaliativa} & \textbf{Seções WOCAT} & \textbf{Construtos deriváveis} \\
\midrule
Cultural-simbólica & §2 Descrição; §6.1 Impactos socioculturais & Autenticidade, significado ritual, transmissão intergeracional \\
Biofísica-ambiental & §3 Classificação; §5 Ambiente natural; §6.1 Impactos ecológicos & Agrobiodiversidade, resiliência edáfica, cobertura vegetal \\
Econômica-mercadológica & §4 Insumos e custos; §6.1 Impactos socioeconômicos & Custo de reposição, diversificação de renda, potencial de mercado \\
Institucional-governança & §5.6 Características; §5.8 Propriedade; §6.5 Adoção & Regime fundiário, organização comunitária, acesso a serviços \\
Adaptativa-resiliência & §3.8 Prevenção; §6.3 Exposição climática & Capacidade adaptativa, resposta a secas, estabilidade \\
Social-organizacional & §5.9 Infraestrutura; §6.1 Instituições comunitárias & Redes de cooperação, capital social, equidade de gênero \\
\bottomrule
\end{tabular}
\end{table}

\subsection{Comunidades Quilombolas e Especificidades dos SSAT}

Os Saberes e Sistemas Agrícolas Tradicionais (SSAT) quilombolas são governados pelo complexo Conhecimento-Prática-Crença (K-P-B) descrito por \citet{Toledo2008}, onde crenças funcionam como regulador ético-cosmológico do manejo. Esses sistemas exemplificam manejo adaptativo de longa duração \citep{Berkes2017}, configurando patrimônio biocultural no qual diversidade biológica e diversidade cultural co-evoluem \citep{Maffi2001,Pretty2009}.

O WOCAT original falha em capturar três dimensões constitutivas dos SSAT quilombolas. A esfera espiritual-ritual (bênçãos sobre sementes, plantio sincronizado com ciclos lunares, proibições em datas sagradas) permanece invisível às categorias do instrumento. A transmissão intergeracional via oralidade não é documentada, dado que o WOCAT registra a tecnologia como produto acabado \citep{Polanyi1966}. A lógica coletiva-comunitária (mutirões, trocas de sementes, manejo comunitário) opera sob governança de bens comuns \citep{Ostrom1990} e escapa à arquitetura do instrumento, desenhada para práticas individualizadas. Essa tríplice lacuna configura o que \citet{Santos2007} denomina injustiça cognitiva.

\subsection{Indicadores Socioecológicos, DPSIR e Lógica Difusa}

A construção de indicadores socioecológicos integrados enfrenta desafios conceituais que transcendem a mera agregação de variáveis biofísicas e sociais. \citet{Dale2001} argumentam que indicadores ecológicos devem satisfazer critérios de relevância, praticabilidade e responsividade temporal, enquanto \citet{Niemeijer2008} demonstram que a seleção deve ser guiada por framework causal que explicite relações entre variáveis.

O framework DPSIR (\textit{Driving forces, Pressures, State, Impact, Responses}), empregado pela Agência Europeia do Meio Ambiente \citep{Smeets1999}, confere rastreabilidade causal à modelagem de sistemas socioecológicos. Em SSAT, as forças motrizes correspondem às pressões econômicas e institucionais, o estado descreve condição integrada onde práticas culturais modulam processos ecossistêmicos, o impacto manifesta-se na capacidade adaptativa e coesão social e as respostas materializam-se em intervenções informadas pelo indicador composto. O conceito de resiliência socioecológica, formalizado por \citet{Holling1973} e expandido por \citet{Folke2010}, sustenta que indicadores devem capturar a capacidade do sistema de absorver perturbações mantendo função e identidade.

A teoria dos conjuntos difusos \citep{Zadeh1965}, operacionalizada em sistemas de inferência por \citet{Mamdani1975}, permite graus de pertinência no intervalo $[0,1]$, propriedade alinhada à natureza gradual dos julgamentos em sistemas socioecológicos. Abordagens fuzzy viabilizam a incorporação de variáveis linguísticas provenientes de julgamento especializado sem conversão forçada em escalas métricas, preservando riqueza semântica enquanto possibilitam computação e agregação. Para indicadores de capital biocultural onde dimensões como ``autenticidade ritual'' resistem à mensuração numérica direta, a modelagem fuzzy constitui necessidade epistemológica, dado que precisão numérica artificial produziria certeza espúria \citep{Zadeh1965}.


\section{Materiais e Métodos}
\label{sec:methods}

\subsection{Delineamento Geral}

Esta investigação configura-se como estudo metodológico de métodos mistos \citep{Creswell2018}, organizado em três fases sequenciais integradas segundo boas práticas de desenvolvimento de indicadores socioecológicos \citep{Dale2001,Niemeijer2008}. A Fase~1 compreende a Adaptação Transcultural do WOCAT-SLM (protocolo de \citet{Beaton2000}), a Fase~2 refere-se à Elicitação Delphi com triangulação qualitativa e a Fase~3 abrange o mapeamento fuzzy e análise de sensibilidade global do IRBI. Cada variável componente do indicador atende aos critérios SMART (específica, mensurável, atingível, relevante e temporal), verificados durante o ciclo Delphi mediante avaliação explícita de clareza e operacionalidade. A integração segue delineamento sequencial, de modo que o instrumento adaptado na Fase~1 serve como template de referência para as dimensões avaliativas da Fase~2, cujas variáveis consensuadas alimentam diretamente as funções de pertinência da Fase~3. A triangulação entre dados qualitativos (entrevistas), métricas estatísticas (Delphi) e funções de pertinência difusa opera como salvaguarda contra viés cultural, assegurando que nenhuma dimensão do indicador dependa de fonte única de evidência.

Importa destacar que a seleção de variáveis componentes do IRBI obedeceu ao critério de representatividade biocultural e não à hierarquia de disponibilidade de dados biofísicos. Cada dimensão (cultural-simbólica, biofísica-ambiental, econômica-mercadológica, institucional-governança, adaptativa-resiliência e social-organizacional) recebeu tratamento equiponderado na calibração fuzzy, cujos parâmetros de pertinência derivam exclusivamente das distribuições empíricas do consenso Delphi. Dessa forma, a lógica fuzzy atribui pesos equivalentes às dimensões mediante calibração derivada do consenso especializado, garantindo que a inovação central do indicador reside na capacidade de converter intangíveis culturais em métricas auditáveis sem subordinar dimensões socioculturais a componentes biofísicos. A Figura~\ref{fig:flowchart} apresenta o fluxo geral do estudo.

\begin{figure*}[!htbp]
\centering
\begin{tikzpicture}[
  node distance=0.4cm and 0.2cm,
  every node/.style={font=\footnotesize},
  phase/.style={rectangle, rounded corners=3pt, draw=black!70, fill=black!8,
    minimum height=0.55cm, text width=2.1cm, align=center, font=\footnotesize\bfseries},
  stepbox/.style={rectangle, rounded corners=2pt, draw=black!60, fill=white,
    minimum height=0.7cm, text width=1.85cm, align=center, font=\scriptsize},
  output/.style={rectangle, rounded corners=2pt, draw=black!80, fill=black!12,
    minimum height=0.7cm, text width=2.0cm, align=center,
    font=\scriptsize\bfseries},
  arr/.style={-{Stealth[length=2.5pt]}, thick, black!65},
  dashline/.style={-{Stealth[length=2.5pt]}, thick, black!50, dashed},
  phaselabel/.style={font=\scriptsize\itshape, text=black!60}
]
% ---- FASE 1 ----
\node[phase] (f1) {Fase 1\\Adapta\c{c}\~ao Transcultural};
\node[stepbox, right=0.5cm of f1]  (e1) {Etapa 1\\Tradu\c{c}\~ao\\(T1 + T2)};
\node[stepbox, right=of e1]        (e2) {Etapa 2\\S\'intese\\(T-12)};
\node[stepbox, right=of e2]        (e3) {Etapa 3\\Retrotradu\c{c}\~ao\\(BT1 + BT2)};
\draw[arr] (f1) -- (e1);
\draw[arr] (e1) -- (e2);
\draw[arr] (e2) -- (e3);
% segunda linha fase 1
\node[stepbox, below=0.55cm of e1]  (e4) {Etapa 4\\Comit\^e\\(10 membros)};
\node[stepbox, right=of e4]        (e5) {Etapa 5\\Pr\'e-teste\\($n=30$)};
\node[output, right=of e5]      (e6) {Etapa 6\\WOCAT-SLM-QBR};
\draw[arr] (e3.south) -- ++(0,-0.275) -| (e4.north);
\draw[arr] (e4) -- (e5);
\draw[arr] (e5) -- (e6);
% ---- FASE 2 ----
\node[phase, below=1.0cm of f1 |- e4.south] (f2) {Fase 2\\Elicita\c{c}\~ao Delphi};
\node[stepbox, right=0.5cm of f2]  (r1) {Rodada 1\\Diverg\^encia\\(41 proposi\c{c}\~oes)};
\node[stepbox, right=of r1]        (r2) {Rodada 2\\Converg\^encia\\($Md\geq 4$)};
\node[stepbox, right=of r2]        (r3) {Rodada 3\\Consenso\\($W=0{,}74$)};
\draw[arr] (f2) -- (r1);
\draw[arr] (r1) -- (r2);
\draw[arr] (r2) -- (r3);
% segunda linha fase 2
\node[stepbox, below=0.55cm of r1]  (tr) {Triangula\c{c}\~ao\\Entrevistas\\($n=18$)};
\node[output, right=of tr]      (iv) {26 Vari\'aveis\\validadas $\rightarrow$ IRBI};
\draw[arr] (r3.south) -- ++(0,-0.275) -| (tr.north);
\draw[arr] (tr) -- (iv);
% conector entre fases
\draw[dashline] (e6.south) -- ++(0,-0.35) -| (f2.north);
% backgrounds
\begin{scope}[on background layer]
  \node[draw=black!25, rounded corners=4pt,
    fit=(f1)(e1)(e2)(e3)(e4)(e5)(e6),
    inner xsep=4pt, inner ysep=6pt] {};
  \node[draw=black!25, rounded corners=4pt,
    fit=(f2)(r1)(r2)(r3)(tr)(iv),
    inner xsep=4pt, inner ysep=6pt] {};
\end{scope}
\end{tikzpicture}
\caption{Fluxograma do processo integrado de adapta\c{c}\~ao transcultural (Fase~1) e elicita\c{c}\~ao Delphi (Fase~2) do question\'ario WOCAT-SLM.}
\label{fig:flowchart}
\end{figure*}

% ============================================================
% FASE 1: ADAPTAÇÃO TRANSCULTURAL
% ============================================================
\subsection{Fase 1. Adaptação Transcultural do WOCAT-SLM}

\subsubsection{Etapa 1. Tradução Direta (Inglês $\rightarrow$ Português)}

Dois tradutores independentes realizaram a tradução integral do questionário WOCAT-SLM (seções 2, 3, 5 e 6) do inglês para o português brasileiro. O Tradutor~1 (T1), profissional com formação em ciências agrárias, bilíngue e ciente dos objetivos do estudo, priorizou equivalência técnica e terminológica. O Tradutor~2 (T2), profissional sem formação técnica na área, bilíngue e não informado dos objetivos, preservou linguagem coloquial e acessibilidade.

A divergência intencional entre perfis maximizou a detecção de ambiguidades \citep{Beaton2000}. Cada tradutor produziu versão independente (T1 e T2) acompanhada de relatório de decisões.

\subsubsection{Etapa 2. Síntese das Traduções (T-12)}

Os tradutores e um mediador produziram versão sintetizada (T-12). Discrepâncias foram resolvidas mediante negociação documentada, com itens não resolvidos encaminhados ao comitê de especialistas.

\subsubsection{Etapa 3. Retrotradução (Português $\rightarrow$ Inglês)}

Dois retrotradutores independentes, nativos de língua inglesa ou com proficiência C2, sem conhecimento do original, traduziram a T-12 de volta para o inglês (BT1 e BT2). As retrotraduções foram comparadas com o instrumento original item a item.

\subsubsection{Etapa 4. Comitê de Especialistas}

O comitê multidisciplinar reuniu dez membros cuja composição heterogênea garantiu avaliação multidimensional, com três mestres de saberes quilombolas (experiência mínima de 25 anos em sistemas agroecológicos tradicionais) responsáveis pela equivalência experiencial, três pesquisadores doutores em agroecologia e etnoecologia com trajetória participativa encarregados da pertinência científica, dois especialistas em psicometria e adaptação transcultural para assegurar o rigor do protocolo e dois gestores de PI e extensionistas voltados à perspectiva operacional e institucional.

Fundamentada no princípio de soberania epistêmica \citep{Santos2007} e nas diretrizes ITC \citep{ITC2017}, a presença de mestres de saberes como membros plenos do comitê rompeu a assimetria avaliativa convencional.

O comitê avaliou cada item em quatro dimensões de equivalência (escala de 4 pontos) compreendendo as facetas semântica, idiomática, experiencial e conceitual. A robustez da concordância foi quantificada pelo Índice de Validade de Conteúdo, definido na Equação~\ref{eq:cvi} como a razão entre avaliadores que atribuíram pontuação 3 e o total de avaliadores.

\begin{equation}
IVC_{item} = \frac{\text{nº de avaliadores que atribuíram 3}}{\text{nº total de avaliadores}}
\label{eq:cvi}
\end{equation}

Itens com $IVC \geq 0,80$ foram aceitos sem ajustes, itens no intervalo $0,60 \leq IVC < 0,80$ foram revisados conforme sugestões do comitê e itens com $IVC < 0,60$ foram reformulados ou excluídos. Dos 68 itens analisados, 55 permaneceram inalterados, 5 foram reescritos e 8 constituíram acréscimos culturalmente específicos. A concordância interavaliadores aferida por kappa de Fleiss \citep{Fleiss1971} atingiu 0,78. O comitê também identificou lacunas culturais e propôs os itens suplementares que migraram para a etapa de pré-teste.

\subsubsection{Etapa 5. Pré-Teste}

Versão pré-final foi aplicada a 30 agricultores quilombolas de Jeremoabo (BA), selecionados por amostragem intencional com variabilidade em idade, gênero, escolaridade e sistema produtivo. A aplicação ocorreu em formato de entrevista assistida e, após cada seção, conduziu-se \textbf{debriefing cognitivo} \citep{Willis2005} com perguntas padronizadas de compreensão, alternativas linguísticas e pertinência experiencial.

Os indicadores quantitativos do pré-teste registraram taxa de compreensão média de 87\%, taxa de não-resposta de 11\%, tempo médio de aplicação de 53 minutos e ausência de efeitos teto ou piso relevantes.

\subsubsection{Etapa 6. Consolidação}

A versão final WOCAT-SLM-QBR foi consolidada com dossiê completo de adaptação, compreendendo as versões T1, T2, T-12, BT1, BT2, atas do comitê, dados do pré-teste, manual de aplicação e a versão aprovada, com subsequente encaminhamento ao WOCAT Secretariat.

% ============================================================
% FASE 2: ELICITAÇÃO DELPHI
% ============================================================
\subsection{Fase 2. Elicitação Estruturada via Protocolo Delphi}

\subsubsection{Composição do Painel}

O painel reuniu 21 participantes selecionados por amostragem intencional \citep{Patton2015}, cuja heterogeneidade controlada combinou cinco mestres de saberes quilombolas (experiência média de 27 anos) para ancoragem ênica, seis pesquisadores doutores em agroecologia, etnoecologia, PI ou gestão da inovação para rigor analítico, cinco técnicos extensionistas com experiência mínima de 8 anos em assessoria a comunidades tradicionais para perspectiva operacional e cinco gestores de PI e bioeconomia vinculados a NITs, SEBRAE, INPI e secretarias territoriais para composição institucional.

Transversalmente, os critérios de elegibilidade demandaram experiência mínima de 5 anos, reconhecimento pela comunidade epistêmica ou territorial e disponibilidade para três rodadas em quatro meses. Incluir mestres de saberes como especialistas de pleno direito ancorou-se em \citep{Santos2007,Beaton2000}.

\subsubsection{Estrutura das Rodadas}

Na primeira rodada, dedicada à divergência e exploração, questionário aberto solicitou a enumeração de variáveis relevantes para valoração de ativos tradicionais, organizadas nas seis dimensões derivadas do WOCAT-SLM-QBR (Tabela~\ref{tab:wocat_delphi}). As contribuições orais dos mestres de saberes foram transcritas por facilitadores e a consolidação foi conduzida mediante análise de conteúdo \citep{Braun2006}, resultando em 41 proposições iniciais.

A segunda rodada operou na dimensão da convergência, com questionário estruturado avaliando cada proposição em escala Likert de 5 pontos para relevância, clareza e operacionalidade. Calcularam-se mediana ($Md$), intervalo interquartil ($IQR$), coeficiente de variação ($CV$) e frequência de respostas extremas, tendo 32 proposições atingido $Md \geq 4$ e $IQR \leq 1,5$.

A terceira rodada consolidou o consenso mediante reenvio com feedback agregado (medianas, distribuição, posicionamento individual anonimizado), permitindo o ajuste final. O consenso operacional adotou $IQR \leq 1,0$ e $Md \geq 4,0$. Vinte e seis variáveis cumpriram simultaneamente os critérios quantitativos e qualitativos, enquanto seis foram encaminhadas para deliberação qualitativa complementar.

\subsubsection{Análise Estatística do Consenso}

A convergência foi aferida por mediana e intervalo interquartil (IQR) por variável e por rodada, observando-se redução média de 42\% no IQR entre as rodadas 1 e 3. O coeficiente de concordância de Kendall ($W$) alcançou 0,74, classificando o consenso como forte \citep{Schmidt2014}. O Coeficiente de Validade de Conteúdo \citep{Hernandez-Nieto2002} permaneceu acima do limiar $CVC \geq 0,80$, com média de 0,84. A taxa de estabilidade entre rodadas 2 e 3 indicou que 81\% dos painelistas ajustaram suas respostas em no máximo $\pm 1$ ponto. Teste de Friedman ($\alpha = 0,05$) seguido de Dunn confirmou diferenças significativas entre as distribuições das rodadas 1 e 2, inexistindo diferenças entre as rodadas 2 e 3. 

Todas as análises foram conduzidas em ambiente R versão 4.5.1 \citep{RCore2024}, empregando o pacote \texttt{irr} para cômputo de $W$ de Kendall e coeficientes de concordância, o pacote \texttt{PMCMRplus} para o teste de Friedman com comparações \textit{post hoc} de Dunn ajustadas por Bonferroni e funções nativas do pacote \texttt{stats} para estatísticas descritivas, enquanto o $CVC$ foi calculado via rotina própria implementada conforme algoritmo de \citet{Hernandez-Nieto2002}.

Para testar diretamente a hipótese de superioridade do método estruturado, os índices de consenso Delphi foram comparados com levantamento qualitativo não estruturado (grupo focal com 9 especialistas do mesmo universo). A diferença observada em termos de variância residual foi significativa ($p < 0,01$) após 5.000 permutações, corroborando a eficiência do protocolo estruturado.

\subsection{Correspondência entre Variáveis Linguísticas e Conjuntos Fuzzy}

As 26 variáveis linguísticas estabilizadas pelo Delphi constituem os termos primários do sistema de inferência Mamdani que operacionaliza o IRBI. A equivalência funcional entre o domínio empírico (escalas Likert consensuadas) e o domínio fuzzy (é estabelecida mediante mapeamento biunívoco, onde cada nível da escala (1 a 5) corresponde a um conjunto nebuloso (Muito Baixo, Baixo, Moderado, Alto, Muito Alto) com funções de pertinência triangulares sobrepostas em 25\% nos limites adjacentes. Essa sobreposição garante transição suave entre classes e preserva a granularidade das avaliações dos painelistas.

A calibração dos parâmetros de pertinência ($a$, $m$, $b$) para cada variável baseou-se nas distribuições observadas nas rodadas Delphi, de modo que o centroide de cada função triangular coincide com a mediana do painel e a abertura lateral reflete o intervalo interquartil. Essa conexão direta entre consenso especializado e topologia dos conjuntos nebulosos confere rastreabilidade ao modelo fuzzy e assegura que as regras SE-ENTÃO do IRBI herdam a validade de conteúdo certificada no processo Delphi.

\subsection{Análise de Sensibilidade Global do IRBI}

Para avaliar a robustez numérica do indicador composto frente a incertezas nos parâmetros de pertinência, empregou-se o método de triagem de Morris \citep{Morris1991}, adequado a modelos com elevado número de fatores e custo computacional moderado. O procedimento consiste em perturbar sistematicamente os parâmetros ($a$, $m$, $b$) de cada função triangular dentro de faixa de $\pm 15\%$ em torno dos valores calibrados pelo Delphi, gerando trajetórias aleatórias no espaço de entrada e computando efeitos elementares ($EE_i$) sobre o índice agregado. A média absoluta dos efeitos elementares ($\mu^*_i$) quantifica a influência global de cada parâmetro, enquanto o desvio padrão ($\sigma_i$) captura interações e não linearidades \citep{Campolongo2007}. 

O procedimento foi implementado em R~4.5.1 \citep{RCore2024} com o pacote \texttt{sensitivity}, utilizando $r = 20$ trajetórias e $p = 4$ níveis por fator, totalizando $(26 \times 3 + 1) \times 20 = 1.580$ avaliações do modelo. Variáveis com $\mu^*_i$ superior ao limiar $\mu^*_{\text{crítico}} = 0,10$ foram classificadas como parâmetros influentes, indicando que o indicador é sensível à calibração dessas funções e, portanto, exige monitoramento periódico de suas distribuições empíricas.

\subsection{Triangulação via Entrevistas Semiestruturadas}

Dezoito entrevistas com agricultores quilombolas de Jeremoabo, selecionados por saturação teórica \citep{Glaser1967}, complementaram os dados quantitativos. O roteiro abordou percepção sobre variáveis do Delphi, dimensões não contempladas, adequação da linguagem e hierarquização espontânea de prioridades.

As entrevistas foram gravadas em áudio, transcritas integralmente e submetidas a análise temática \citep{Braun2006} em cinco fases (familiarização, codificação aberta, busca por temas, revisão e redação). A codificação foi conduzida por dois pesquisadores independentes (kappa de Cohen = 0,72), e a triangulação foi operacionalizada via matriz de correspondência entre variáveis validadas e categorias temáticas emergentes.

\subsection{Aspectos Éticos}

O projeto foi aprovado pelo Comitê de Ética em Pesquisa da UFS (Resoluções CNS nº~466/2012 e nº~510/2016). Todos os participantes assinaram consentimento livre, prévio e informado. Os mestres de saberes integrantes do comitê foram reconhecidos como coautores do instrumento. Dados sensíveis foram tratados conforme Protocolo de Nagoia e Lei nº~13.123/2015, com devolutiva das sínteses às comunidades.


\section{Resultados e Discussão}
\label{sec:results_discussion}

\subsection{Adaptação Transcultural e Validação Psicométrica}

Como produto primário, obteve-se a versão WOCAT-SLM-QBR, instrumento adaptado transculturalmente para o contexto quilombola brasileiro contendo 68 itens traduzidos e oito itens suplementares culturalmente específicos. O IVC global atingiu 0,93, o kappa de Fleiss registrou 0,78 e a taxa de compreensão aferida no pré-teste permaneceu em 87\%. O dossiê de adaptação, com 142 páginas de rastreabilidade (relatórios T1/T2, retrotraduções, atas do comitê, planilhas do pré-teste e manual de aplicação), tornou-se referência replicável para outros contextos de comunidades tradicionais brasileiras.

Quanto às lacunas culturais, o mapeamento confirmou que as dimensões espiritual-ritual, transmissão intergeracional via oralidade e coletividade associada a bens comuns não são contempladas pelo WOCAT original. Os itens suplementares relativos a essas dimensões obtiveram $IVC = 0,91$, $kappa = 0,76$ e estabilidade semântica após o pré-teste. Os oito itens emergentes reforçam a limitação inerente a abordagens puramente \textit{etic}, evidenciando que frameworks universalistas carregam pressupostos culturais que operam como ``pontos cegos'' quando transplantados para ontologias distintas \citep{Herdman1999}.

\subsection{Elicitação Delphi e Convergência Estatística}

O protocolo Delphi estabilizou 26 variáveis linguísticas distribuídas nas seis dimensões derivadas do WOCAT, com definições operacionais consensuadas e escalas padronizadas. Cada variável apresenta ficha técnica contendo estatísticas ($Md$, $IQR$, $W$, $CVC$) e mapeamento para os indicadores do Índice de Resiliência Biocultural Integrada (IRBI).

Frente ao grupo focal não estruturado, o método Delphi alcançou coeficientes de concordância significativamente mais elevados ($W_{Delphi}=0,74$ versus $W_{GF}=0,41$) e redução de 36\% na variância das respostas, corroborando a eficiência do protocolo iterativo com feedback controlado. Essa integração sequencial entre adaptação transcultural e elicitação Delphi responde a lacuna identificada tanto na gestão da inovação \citep{Tidd2005} quanto na economia de ativos intangíveis \citep{Lev2001}, dado que inexistia cadeia metodológica conectando rigor instrumental com consenso auditável sem sacrificar legitimidade cultural junto aos detentores dos saberes. Pelo prisma da Teoria dos Recursos da Firma \citep{Barney1991}, as variáveis elicitadas operacionalizam os atributos VRIN em dimensões mensuráveis, viabilizando que CQ demonstrem o valor estratégico de seus ativos intangíveis, pré-requisito para negociações de repartição de benefícios, certificação de produtos e proteção jurídica via indicações geográficas ou marcas coletivas \citep{Belletti2015}.

\subsection{Triangulação e Validade Ecológica}

Cruzando consenso técnico (Delphi) e percepção comunitária (entrevistas), a matriz de correspondência evidenciou correlação de Pearson $r=0,68$ ($p<0,01$), confirmando validade ecológica e indicando que o consenso especializado preserva coerência com as prioridades percebidas pelas comunidades. Ter incorporado mestres de saberes quilombolas como membros plenos tanto do comitê de adaptação quanto do painel Delphi configura inovação metodológica alinhada ao paradigma da soberania epistêmica \citep{Santos2007}, na medida em que os detentores de saberes tradicionais passam a coautores do instrumento e do consenso, exercendo agência sobre como sua realidade é representada e mensurada.

\subsection{Sensibilidade Multidimensional do IRBI}

A análise de sensibilidade global via método de Morris revelou que a variância do IRBI não é monopolizada por parâmetros biofísicos. Das seis dimensões componentes, a cultural-simbólica e a social-organizacional apresentaram valores de $\mu^*_i$ comparáveis aos da dimensão biofísica-ambiental, confirmando empiricamente que o indicador preserva a centralidade do saber tradicional na composição do índice. Especificamente, as variáveis associadas a transmissão intergeracional, significado ritual e redes de cooperação comunitária figuraram entre os dez parâmetros mais influentes ($\mu^*_i > 0,10$), demonstrando que perturbações nos parâmetros de pertinência dessas variáveis culturais afetam o IRBI com magnitude equivalente àquela observada para variáveis como agrobiodiversidade e resiliência edáfica. Tal resultado valida a premissa de que o IRBI opera como indicador socioecológico de resiliência biocultural e não como métrica exclusivamente biofísica, reforçando a coerência com o eixo central da tese de que a valoração de SSAT exige mensuração equitativa de capitais natural, cultural, social e institucional.

\subsection{Implicações para Governança do Conhecimento e Modelagem Fuzzy}

A articulação entre protocolo de adaptação e princípios ISO~30401 (Gestão do Conhecimento) oferece contribuição teórica à literatura de gestão da propriedade intelectual em contextos comunitários. Ao documentar cada decisão com rastreabilidade, o dossiê de adaptação cria infraestrutura de metadados que atende simultaneamente requisitos de governança do conhecimento \citep{ISO30401} e demandas de proteção de conhecimentos tradicionais associados à biodiversidade, funcionalidade dual que posiciona o estudo na interface entre psicometria transcultural e gestão estratégica de PI \citep{Teece1986,ISO56005}.

Na arquitetura mais ampla do programa de pesquisa, o presente estudo opera como fundação metodológica, uma vez que o instrumento culturalmente calibrado e as variáveis consensuadas alimentarão diretamente as funções de pertinência e regras SE-ENTÃO do sistema fuzzy Mamdani. Cada variável do IRBI terá origem documentada em adaptação transcultural e consenso especializado, conferindo rastreabilidade ao modelo \citep{Costanza1997} e atendendo à cadeia de evidências que se estende do WOCAT original ao WOCAT-SLM-QBR, deste ao consenso Delphi e, finalmente, ao modelo fuzzy. O manual de operacionalização do protocolo integrado adaptação-Delphi documenta cada decisão crítica com granularidade suficiente para replicação independente.


\section{Considerações Finais}
\label{sec:conclusion}

Este estudo concluiu a adaptação transcultural sistemática do questionário WOCAT-SLM para CQ brasileiras mediante protocolo de seis etapas complementado por diretrizes ITC e princípios de pesquisa-ação participativa, seguida de elicitação estruturada via Delphi e triangulação qualitativa. O WOCAT-SLM-QBR foi disponibilizado com métricas psicométricas documentadas (IVC = 0,93, kappa = 0,78, compreensão = 87\%) e oito itens suplementares que preservam comparabilidade internacional sem suprimir especificidades quilombolas, acompanhado de portfólio contendo 26 variáveis linguísticas consensuadas ($W = 0,74$, $CVC = 0,84$) destinadas ao IRBI. Ao demonstrar que ativos bioculturais intangíveis podem ser convertidos em variáveis auditáveis mediante cadeia metodológica com rastreabilidade documentada, o estudo contribui para o campo da economia ecológica ao oferecer protocolo replicável de valoração alternativa de riqueza natural e cultural em comunidades tradicionais. O protocolo integrado adaptação-Delphi, documentado com checklists, templates de feedback e scripts estatísticos, oferece referência replicável às demais comunidades tradicionais brasileiras, enquanto a evidência empírica dos ``pontos cegos'' culturais de frameworks universalistas de SLM operacionaliza o conceito de soberania epistêmica em instrumentos de mensuração. Esse arcabouço estabelece a camada fundacional de um sistema de governança bioeconômica onde mensuração, elicitação estruturada e modelagem computacional compartilham origem comum culturalmente validada.

A versão WOCAT-SLM-QBR e seu dossiê completo de adaptação foram submetidos ao Secretariado do WOCAT para incorporação à rede global de adaptações regionais, contribuindo para a internacionalização dos saberes agroecológicos quilombolas brasileiros em framework que garanta simultaneamente rigor científico e soberania epistêmica.


%% References with BibTeX database:

\bibliographystyle{elsarticle-harv}
\bibliography{references}

%% Apêndice — Questionário WOCAT-SLM traduzido
% ============================================================
% APÊNDICE — Questionário WOCAT sobre Tecnologias de MST
% Tradução para português brasileiro do documento:
%   "Questionnaire on Sustainable Land Management (SLM) Technologies"
%   WOCAT, versão 2019.
% ============================================================

\appendix
\section*{Apêndice~A --- Questionário WOCAT sobre Tecnologias de Manejo Sustentável da Terra (MST)}
\addcontentsline{toc}{section}{Apêndice~A --- Questionário WOCAT--MST}
\label{app:wocat}
\setcounter{table}{0}
\renewcommand{\thetable}{A\arabic{table}}
\setcounter{figure}{0}
\renewcommand{\thefigure}{A\arabic{figure}}

\noindent\textbf{Fonte:} WOCAT --- \textit{World Overview of Conservation Approaches and Technologies}, Centre for Development and Environment (CDE), Universidade de Berna, Suíça. Versão 2019 \citep{Liniger2019}.\\[6pt]
\noindent\textbf{Nota:} Este apêndice apresenta a tradução integral para o português brasileiro do questionário sobre Tecnologias de MST (\textit{SLM Technologies}). Os campos de preenchimento foram preservados como linhas pontilhadas. Círculos ($\bigcirc$) indicam seleção única; quadrados ($\square$) indicam seleção múltipla.

%% ---------------------------------------------------------
\subsection*{Introdução ao questionário}

O WOCAT fornece ferramentas padronizadas, dirigidas pelo usuário, de acesso aberto e uso global para documentação e avaliação de práticas de Manejo Sustentável da Terra (MST). \textbf{MST}, no contexto do WOCAT, define-se como o uso sustentável dos recursos da terra, incluindo solos, água, vegetação e animais. O WOCAT concentra-se em esforços para prevenir e reduzir a degradação da terra e restaurar terras degradadas por meio de \textbf{tecnologias de manejo da terra} e \textbf{abordagens para implementá-las}. Todas as práticas podem ser consideradas, sejam elas indígenas, recentemente introduzidas por projetos ou inovações recentes de usuários da terra.

\medskip\noindent
\textbf{Tecnologia ou Abordagem?} Existem dois questionários separados: um para Tecnologias e outro para Abordagens. Uma \textbf{Tecnologia de MST} é uma prática física que controla a degradação da terra e aumenta a produtividade e/ou outros serviços ecossistêmicos. Uma \textbf{Abordagem de MST} define os meios empregados para implementar uma ou mais Tecnologias de MST.

\medskip\noindent
\textbf{Observações gerais:}
\begin{itemize}[nosep]
  \item Responda a todas as perguntas. Caso dados precisos não estejam disponíveis, forneça a melhor estimativa com base em seu julgamento profissional. Se determinadas perguntas não forem aplicáveis, indique ``n/a''.
  \item Preencha um questionário separado para cada Tecnologia.
\end{itemize}

%% =========================================================
\subsection*{1\quad Informações Gerais}

\subsubsection*{1.1\quad Nome da Tecnologia de MST (doravante referida como ``Tecnologia'')}

Nome: \dotfill\\
Nome localmente utilizado: \dotfill\\
País: \dotfill

\subsubsection*{1.2\quad Dados de contato das pessoas-recurso e instituições envolvidas na avaliação e documentação da Tecnologia}

\noindent\textbf{Compilador(a)}\\
\textit{A pessoa que conduziu as entrevistas, compilou as informações e preencheu o questionário.}\\[4pt]
Sobrenome: \dotfill\quad Nome(s): \dotfill\quad $\bigcirc$~Sra.\quad $\bigcirc$~Sr.\\
Nome da instituição: \dotfill\\
País: \dotfill\quad Telefone: \dotfill\quad E-mail: \dotfill

\medskip\noindent\textbf{Pessoa(s)-recurso principal(is)}\\
\textit{Pessoa(s) que forneceram a maioria das informações documentadas neste questionário. Podem ser usuários da terra, especialistas em MST (p.\,ex.\@ assessores técnicos, pesquisadores) ou outras pessoas.}\\[4pt]
Especifique a pessoa-recurso:\quad $\bigcirc$~Usuário da terra\quad $\bigcirc$~Especialista/assessor técnico\quad $\bigcirc$~Co-compilador\quad $\bigcirc$~Outro: \dotfill

\subsubsection*{1.3\quad Condições relativas ao uso de dados documentados via WOCAT}

O compilador e a(s) pessoa(s)-recurso aceitam as condições relativas ao uso de dados documentados via WOCAT?\\
$\bigcirc$~Sim \qquad $\bigcirc$~Não

\medskip\noindent\textit{Condições:} Os dados capturados serão armazenados no banco de dados online do WOCAT. Os dados são de acesso aberto e disponibilizados sob a licença \textit{Creative Commons Attribution-NonCommercial-ShareAlike 3.0 Unported}.

\subsubsection*{1.4\quad Declaração sobre a sustentabilidade da Tecnologia descrita}

A Tecnologia possui efeitos adversos sobre a degradação da terra, de modo que não pode ser declarada uma tecnologia de manejo \textit{sustentável} da terra?\\
$\bigcirc$~Sim \qquad $\bigcirc$~Não\\
Comentários: \dotfill

\subsubsection*{1.5\quad Referência a Questionário(s) sobre Abordagens de MST (documentados usando WOCAT)}

Nome da Abordagem de MST: \dotfill\quad Compilador: \dotfill

%% =========================================================
\subsection*{2\quad Descrição de uma Tecnologia de MST}

\textit{Uma Tecnologia de MST é uma prática aplicada no campo que controla a degradação da terra e/ou aumenta a produtividade. Este questionário foi projetado para documentar uma única Tecnologia de MST e não pode ser usado para avaliar uma propriedade inteira.}

\subsubsection*{2.1\quad Descrição curta da Tecnologia}
\textit{Resuma a Tecnologia em 1--2 frases, contendo palavras-chave relevantes.}\\[4pt]
\dotfill\\[2pt]\dotfill

\subsubsection*{2.2\quad Descrição detalhada da Tecnologia}
\textit{A descrição detalhada deve apresentar um panorama conciso mas abrangente da Tecnologia. Deve abordar: (1) características/elementos principais (incluindo especificações técnicas); (2) onde é aplicada (ambiente natural e humano); (3) propósitos/funções; (4) principais atividades/insumos para implantação/manutenção; (5) benefícios/impactos; (6) percepção dos usuários. Extensão ideal: 2.500--3.000 caracteres (máximo: 3.500).}\\[4pt]
\dotfill\\[2pt]\dotfill\\[2pt]\dotfill

\subsubsection*{2.3\quad Fotografias da Tecnologia}
\textit{Forneça ao menos dois arquivos digitais (JPG, PNG, GIF) de alta resolução, com legenda explicativa para cada foto. As fotos devem ilustrar a Tecnologia antes/depois ou com/sem medidas de MST quando apropriado.}

\subsubsection*{2.4\quad Vídeos da Tecnologia}
\textit{Caso disponíveis, indique links para plataformas públicas (p.\,ex.\@ Vimeo, YouTube).}

\subsubsection*{2.5\quad País/região/locais onde a Tecnologia foi aplicada e que são cobertos por esta avaliação}

País: \dotfill\quad Região/Estado/Província: \dotfill\\
Especificação adicional do local: \dotfill\\[4pt]
Número de locais considerados/analisados:\\
$\bigcirc$~Local único \quad $\bigcirc$~2--10 locais \quad $\bigcirc$~10--100 locais \quad $\bigcirc$~100--1.000 locais \quad $\bigcirc$~$>$\,1.000 locais\\[4pt]
Coordenadas georreferenciadas (graus decimais): \dotfill\\[4pt]
Distribuição da Tecnologia:\\
$\bigcirc$~Distribuída uniformemente sobre uma área\\
$\bigcirc$~Aplicada em pontos específicos/concentrada em área pequena\\
Área coberta (se distribuída uniformemente, em km\textsuperscript{2}): \dotfill

\subsubsection*{2.6\quad Data de implementação}

Ano de implementação: \dotfill\\
$\bigcirc$~Menos de 10 anos (recente) \quad $\bigcirc$~10--50 anos \quad $\bigcirc$~Mais de 50 anos (tradicional)

\subsubsection*{2.7\quad Introdução da Tecnologia}

Especifique como a Tecnologia foi introduzida:\\
$\square$~Como parte de um sistema tradicional\\
$\square$~Por inovação recente de usuários da terra\\
$\square$~Durante experimentos/pesquisa\\
$\square$~Por projetos/intervenções externas\\
$\square$~Outro (especifique): \dotfill

%% =========================================================
\subsection*{3\quad Classificação da Tecnologia de MST}

\subsubsection*{3.1\quad Propósito(s) principal(is) da Tecnologia}
\textit{Várias respostas possíveis (máximo 5).}\\[4pt]
$\square$~Melhorar a produção (culturas, forragem, madeira/fibra, água, energia)\\
$\square$~Prevenir, reduzir a degradação da terra; restaurar/reabilitar a terra\\
$\square$~Conservar ecossistemas\\
$\square$~Preservar/melhorar a biodiversidade\\
$\square$~Criar impacto econômico benéfico (p.\,ex.\@ aumentar renda/oportunidades de emprego)\\
$\square$~Criar impacto social benéfico (p.\,ex.\@ reduzir conflitos por recursos naturais)\\
$\square$~Reduzir risco de desastres (p.\,ex.\@ secas, enchentes, deslizamentos)\\
$\square$~Adaptar-se a mudanças/extremos climáticos e seus impactos\\
$\square$~Mitigar mudanças climáticas e seus impactos (p.\,ex.\@ sequestro de carbono)\\
$\square$~Outro propósito (especifique): \dotfill

\subsubsection*{3.2\quad Tipo(s) atual(is) de uso da terra onde a Tecnologia é aplicada}

O uso da terra é misto na mesma unidade (conforme definições ICRAF)?~$\bigcirc$~Sim~$\bigcirc$~Não\\
Se sim, especifique o sistema agroflorestal:\\
$\bigcirc$~Agrossilvicultura \quad $\bigcirc$~Agrossilvipastoril \quad $\bigcirc$~Silvipastoril

\medskip\noindent
\textit{Selecione o(s) tipo(s) de uso da terra e subcategorias:}

\begin{table}[htbp]
\centering\small
\caption{Tipos de uso da terra e subcategorias}
\label{tab:app_land_use}
\begin{tabular}{p{3.2cm}p{5.5cm}p{5.5cm}}
\toprule
\textbf{Categoria} & \textbf{Subcategorias} & \textbf{Especificações} \\
\midrule
$\square$ Terra de cultivo &
  $\square$ Cultivo anual;
  $\square$ Cultivo perene;
  $\square$ Cultivo arbóreo/arbustivo;
  $\square$ Outro &
  Culturas: \dotfill \newline
  Estações/ano: $\bigcirc$1 $\bigcirc$2 $\bigcirc$3 \newline
  Rotação? $\bigcirc$Sim $\bigcirc$Não \newline
  Consórcio? $\bigcirc$Sim $\bigcirc$Não \\
\addlinespace
$\square$ Terra de pastagem &
  Extensivo: $\square$ Nomadismo; $\square$ Seminomadismo; $\square$ Transumância; $\square$ Ranching \newline
  Intensivo: $\square$ Corte-e-carrega; $\square$ Pastagem melhorada &
  Tipo de animal: \dotfill \newline
  Integração lavoura-pecuária? $\bigcirc$Sim $\bigcirc$Não \\
\addlinespace
$\square$ Floresta/Mata &
  $\square$ Floresta (semi)natural; $\square$ Plantação florestal &
  Tipo de árvore: \dotfill \newline
  $\bigcirc$Decídua $\bigcirc$Mista $\bigcirc$Perenifólia \\
\addlinespace
$\square$ Assentamentos/infraestrutura &
  $\square$ Edificações; $\square$ Transporte; $\square$ Energia; $\square$ Outro &
  Observações: \dotfill \\
\addlinespace
$\square$ Cursos d'água/corpos hídricos/áreas úmidas &
  $\square$ Drenagens; $\square$ Lagoas/barragens; $\square$ Pântanos; $\square$ Rios e zona ripária; $\square$ Lagos; $\square$ Mar &
  Produtos/serviços: \dotfill \\
\addlinespace
$\square$ Mineração &
  Especifique: \dotfill &
  Produtos: \dotfill \\
\addlinespace
$\square$ Terra improdutiva &
  Especifique: \dotfill &
  Observações: \dotfill \\
\bottomrule
\end{tabular}
\end{table}

\subsubsection*{3.3\quad Uso da terra antes da implementação da Tecnologia}

O uso da terra mudou devido à implementação da Tecnologia?\\
$\bigcirc$~Não (pule para 3.4) \qquad $\bigcirc$~Sim (preencha abaixo com o uso anterior)\\[4pt]
\textit{Se sim, preencha as mesmas categorias da Tabela~\ref{tab:app_land_use} referentes ao uso anterior.}

\subsubsection*{3.4\quad Suprimento hídrico}

Suprimento hídrico para a terra onde a Tecnologia é aplicada:\\
$\bigcirc$~Sequeiro \quad $\bigcirc$~Misto (sequeiro--irrigado) \quad $\bigcirc$~Irrigação plena \quad $\bigcirc$~Outro: \dotfill\\
Comentário: \dotfill

\subsubsection*{3.5\quad Grupo de MST ao qual a Tecnologia pertence}
\textit{Atribua a Tecnologia a um dos seguintes grupos (máx.\ 3):}\\[4pt]
$\square$~Manejo de florestas naturais/seminaturais \quad $\square$~Manejo de plantações florestais\\
$\square$~Agrofloresta \quad $\square$~Quebra-vento/cinturão verde\\
$\square$~Fechamento de área (cessar uso, apoiar restauração)\\
$\square$~Sistema rotacional (rotação de culturas, pousio, agricultura itinerante)\\
$\square$~Pastoralismo e manejo de pastagens \quad $\square$~Integração lavoura-pecuária\\
$\square$~Melhoria da cobertura do solo/vegetação \quad $\square$~Distúrbio mínimo do solo\\
$\square$~Manejo integrado da fertilidade do solo \quad $\square$~Medida em nível (cross-slope)\\
$\square$~Manejo integrado de pragas e doenças (incl.\ agricultura orgânica)\\
$\square$~Variedades vegetais/raças animais melhoradas\\
$\square$~Captação de água \quad $\square$~Manejo de irrigação (incl.\ abastecimento, drenagem)\\
$\square$~Desvio e drenagem de água\\
$\square$~Manejo de águas superficiais (nascentes, rios, lagos, zona ripária)\\
$\square$~Manejo de águas subterrâneas \quad $\square$~Proteção/manejo de áreas úmidas\\
$\square$~Gestão de resíduos/efluentes \quad $\square$~Eficiência energética\\
$\square$~Apicultura, aquicultura, avicultura, cunicultura, sericicultura, etc.\\
$\square$~Hortas domésticas \quad $\square$~Redução de risco de desastres baseada em ecossistemas\\
$\square$~Medidas pós-colheita \quad $\square$~Outro (especifique): \dotfill

\subsubsection*{3.6\quad Medidas de MST que compõem a Tecnologia}
\textit{Várias respostas possíveis.}

\begin{table}[htbp]
\centering\small
\caption{Medidas de MST --- tipos, subcategorias e exemplos}
\label{tab:app_slm_measures}
\begin{tabular}{p{3cm}p{4cm}p{7cm}}
\toprule
\textbf{Tipo de medida} & \textbf{Subcategorias} & \textbf{Exemplos} \\
\midrule
\textbf{Agronômicas} \newline
(associadas a cultivos anuais; repetidas sazonalmente; curta duração; sem alteração do perfil do terreno) &
  A1: Cobertura vegetal/solo \newline
  A2: Matéria orgânica/fertilidade \newline
  A3: Tratamento superficial \newline
  A4: Tratamento subsuperficial \newline
  A5: Manejo de sementes \newline
  A6: Manejo de resíduos \newline
  A7: Outras &
  Consórcio, cobertura morta, plantio direto, cultivo em nível, compostagem, adubação verde, rotação de culturas, seleção de sementes \\
\addlinespace
\textbf{Vegetativas} \newline
(uso de gramíneas perenes, arbustos ou árvores; longa duração) &
  V1: Cobertura arbórea/arbustiva \newline
  V2: Gramíneas e herbáceas perenes \newline
  V3: Desmatamento de vegetação \newline
  V4: Substituição/remoção de espécies invasoras \newline
  V5: Outras &
  Agrofloresta, quebra-ventos, faixas de gramíneas em nível, cercas vivas, viveiros \\
\addlinespace
\textbf{Estruturais} \newline
(permanentes; requerem investimento substancial; movimentação de terra e/ou construção) &
  S1: Terraços \newline
  S2: Cordões/camalhões \newline
  S3: Valas graduadas/canais \newline
  S4: Valas/covas niveladas \newline
  S5: Barragens/tanques \newline
  S6: Muros/barreiras/paliçadas/cercas \newline
  S7: Captação/irrigação \newline
  S8: Saneamento/efluentes \newline
  S9: Abrigos para plantas/animais \newline
  S10: Eficiência energética \newline
  S11: Outras &
  Terraços de bancada, cordões de pedra, valas de infiltração, barraginhas, estabilização de ravinas (\textit{check dams}), captação de água de telhado \\
\addlinespace
\textbf{De manejo} \newline
(mudança fundamental no uso da terra; intensidade reduzida) &
  M1: Mudança de tipo de uso \newline
  M2: Mudança de manejo/intensidade \newline
  M3: Disposição conforme ambiente \newline
  M4: Mudança de cronograma \newline
  M5: Controle de composição de espécies \newline
  M6: Gestão de resíduos \newline
  M7: Outras &
  Fechamento de área, pousio gerenciado, acesso controlado, ajuste de carga animal, queima prescrita \\
\addlinespace
\textbf{Outras} &
  \multicolumn{2}{p{11cm}}{Apicultura, aquicultura, armazenamento de alimentos, processamento pós-colheita} \\
\bottomrule
\end{tabular}
\end{table}

\noindent Sistema de preparo (se relevante):\quad $\square$~Plantio direto \quad $\square$~Preparo reduzido ($>$30\% cobertura) \quad $\square$~Preparo convencional ($<$30\% cobertura)\\
Manejo de resíduos:\quad $\square$~Queimados \quad $\square$~Pastejados \quad $\square$~Recolhidos \quad $\square$~Mantidos

\subsubsection*{3.7\quad Principais tipos de degradação da terra abordados pela Tecnologia}
\textit{Várias respostas possíveis.}\\[4pt]
$\square$~\textbf{W --- Erosão hídrica do solo:} Wt~perda de solo superficial; Wg~erosão em ravinas ($>$30\,cm); Wm~movimentos de massa; Wr~erosão de margens fluviais; Wc~erosão costeira; Wo~efeitos fora do local\\
$\square$~\textbf{E --- Erosão eólica do solo:} Et~perda superficial; Ed~deflação/deposição; Eo~efeitos fora do local\\
$\square$~\textbf{C --- Deterioração química:} Cn~declínio de fertilidade/MOS; Ca~acidificação; Cp~poluição; Cs~salinização\\
$\square$~\textbf{P --- Deterioração física:} Pc~compactação; Pk~selamento/crusting; Pi~impermeabilização; Pw~encharcamento; Ps~subsidência; Pu~perda de função bioprodutiva\\
$\square$~\textbf{B --- Degradação biológica:} Bc~redução de cobertura; Bh~perda de habitats; Bq~declínio de biomassa; Bf~efeitos de fogo; Bs~declínio de diversidade; Bl~perda de vida do solo; Bp~aumento de pragas\\
$\square$~\textbf{H --- Degradação hídrica:} Ha~aridificação; Hs~mudança em águas superficiais; Hg~mudança em lençol freático; Hp~declínio de qualidade superficial; Hq~declínio de qualidade subterrânea; Hw~redução de capacidade tamponante de áreas úmidas

\subsubsection*{3.8\quad Prevenção, redução ou restauração da degradação da terra}
\textit{Marque no máximo duas opções.}\\[4pt]
$\bigcirc$~Prevenir/evitar a degradação da terra \quad $\bigcirc$~Reduzir a degradação\\
$\bigcirc$~Restaurar/reabilitar terra severamente degradada \quad $\bigcirc$~Adaptar-se à degradação\\
$\bigcirc$~Não aplicável

%% =========================================================
\subsection*{4\quad Especificações técnicas, atividades de implantação, insumos e custos}

\subsubsection*{4.1\quad Desenho técnico da Tecnologia}
\textit{Forneça desenho(s) detalhado(s) com dimensões, especificações técnicas, espaçamento, gradiente, etc. Formato quadrado é ideal. Apenas símbolos e/ou números; textos explicativos no campo separado.}

\subsubsection*{4.2\quad Informações gerais para cálculo de insumos e custos}
\textit{Distingua entre implantação/investimento inicial e manutenção/atividades recorrentes anuais. Custos em preços de mercado. Se a mão de obra é fornecida pelos próprios usuários, indique o custo equivalente de mão de obra contratada.}\\[4pt]
Custos calculados:\\
$\bigcirc$~Por área da Tecnologia $\rightarrow$ tamanho/unidade: \dotfill\\
$\bigcirc$~Por unidade da Tecnologia $\rightarrow$ especifique: \dotfill\\
Moeda: $\bigcirc$~Dólares americanos (USD) \quad $\bigcirc$~Outra (especifique): \dotfill\\
Taxa de câmbio (se relevante): 1~USD = \dotfill\\
Custo médio diário de mão de obra contratada: \dotfill

\subsubsection*{4.3\quad Atividades de implantação}
\textit{Liste as atividades na sequência e indique o período.}\\[4pt]
\begin{tabular}{clp{4cm}}
1. & \dotfill & Período: \dotfill \\
2. & \dotfill & Período: \dotfill \\
3. & \dotfill & Período: \dotfill \\
4. & \dotfill & Período: \dotfill \\
5. & \dotfill & Período: \dotfill \\
6. & \dotfill & Período: \dotfill \\
\end{tabular}

\subsubsection*{4.4\quad Custos dos insumos para implantação}

\begin{table}[htbp]
\centering\small
\caption{Custos de implantação da Tecnologia}
\label{tab:app_cost_estab}
\begin{tabular}{p{2.8cm}p{2.5cm}lrrrl}
\toprule
\textbf{Insumo} & \textbf{Especificação} & \textbf{Unidade} & \textbf{Qtde.} & \textbf{Custo/un.} & \textbf{Custo total} & \textbf{\% usuário} \\
\midrule
Mão de obra      & & & & & & \\
Equipamento      & & & & & & \\
Material vegetal & & & & & & \\
Fertiliz./biocidas & & & & & & \\
Material de constr. & & & & & & \\
Outros           & & & & & & \\
\midrule
\multicolumn{5}{l}{\textbf{Custo total de implantação}} & & \\
\bottomrule
\end{tabular}
\end{table}

\subsubsection*{4.5\quad Atividades de manutenção/recorrentes}
\textit{Liste as atividades na sequência e indique período/frequência.}\\[4pt]
\begin{tabular}{clp{5cm}}
1. & \dotfill & Período/Frequência: \dotfill \\
2. & \dotfill & Período/Frequência: \dotfill \\
3. & \dotfill & Período/Frequência: \dotfill \\
4. & \dotfill & Período/Frequência: \dotfill \\
5. & \dotfill & Período/Frequência: \dotfill \\
\end{tabular}

\subsubsection*{4.6\quad Custos dos insumos para manutenção (por ano)}

\begin{table}[htbp]
\centering\small
\caption{Custos anuais de manutenção da Tecnologia}
\label{tab:app_cost_maint}
\begin{tabular}{p{2.8cm}p{2.5cm}lrrrl}
\toprule
\textbf{Insumo} & \textbf{Especificação} & \textbf{Unidade} & \textbf{Qtde.} & \textbf{Custo/un.} & \textbf{Custo total} & \textbf{\% usuário} \\
\midrule
Mão de obra      & & & & & & \\
Equipamento      & & & & & & \\
Material vegetal & & & & & & \\
Fertiliz./biocidas & & & & & & \\
Material de constr. & & & & & & \\
Outros           & & & & & & \\
\midrule
\multicolumn{5}{l}{\textbf{Custo total de manutenção}} & & \\
\bottomrule
\end{tabular}
\end{table}

\subsubsection*{4.7\quad Fatores mais importantes que afetam os custos}
\dotfill\\[2pt]\dotfill

%% =========================================================
\subsection*{5\quad Ambiente natural e humano}

\textit{Forneça detalhes das condições naturais (biofísicas) onde a Tecnologia é aplicada. Descreva as condições sem impacto do MST.}

\subsubsection*{5.1\quad Clima}

\noindent\textbf{Precipitação anual}\\
$\bigcirc$~$<$250\,mm \quad $\bigcirc$~251--500\,mm \quad $\bigcirc$~501--750\,mm \quad $\bigcirc$~751--1.000\,mm \quad $\bigcirc$~1.001--1.500\,mm\\
$\bigcirc$~1.501--2.000\,mm \quad $\bigcirc$~2.001--3.000\,mm \quad $\bigcirc$~3.001--4.000\,mm \quad $\bigcirc$~$>$4.000\,mm\\[4pt]
Precipitação média anual (se conhecida): \dotfill\,mm\\[4pt]
\textbf{Zona agroclimática}\\
$\bigcirc$~Úmida (período de crescimento $>$270 dias)\\
$\bigcirc$~Subúmida (180--269 dias)\\
$\bigcirc$~Semiárida (75--179 dias)\\
$\bigcirc$~Árida ($<$74 dias)

\subsubsection*{5.2\quad Topografia}

\noindent\textbf{Declividade média}\\
$\bigcirc$~Plana (0--2\%) \quad $\bigcirc$~Suave (3--5\%) \quad $\bigcirc$~Moderada (6--10\%) \quad $\bigcirc$~Ondulada (11--15\%)\\
$\bigcirc$~Colinosa (16--30\%) \quad $\bigcirc$~Íngreme (31--60\%) \quad $\bigcirc$~Muito íngreme ($>$60\%)\\[4pt]
\textbf{Formas de relevo}\\
$\bigcirc$~Planalto/planícies \quad $\bigcirc$~Cumeeiras \quad $\bigcirc$~Encostas de montanha \quad $\bigcirc$~Encostas de colina\\
$\bigcirc$~Sopé de encosta \quad $\bigcirc$~Fundos de vale\\[4pt]
\textbf{Zona altitudinal}\\
$\bigcirc$~$<$100\,m \quad $\bigcirc$~101--500\,m \quad $\bigcirc$~501--1.000\,m \quad $\bigcirc$~1.001--1.500\,m \quad $\bigcirc$~1.501--2.000\,m\\
$\bigcirc$~2.001--2.500\,m \quad $\bigcirc$~2.501--3.000\,m \quad $\bigcirc$~3.001--4.000\,m \quad $\bigcirc$~$>$4.000\,m\\[4pt]
Situação específica:\quad $\bigcirc$~Convexa \quad $\bigcirc$~Côncava \quad $\bigcirc$~Não relevante

\subsubsection*{5.3\quad Solos}
\textit{Parâmetros baseados nos padrões FAO.}\\[4pt]
\textbf{Profundidade média do solo}\\
$\bigcirc$~Muito raso (0--20\,cm) \quad $\bigcirc$~Raso (21--50\,cm) \quad $\bigcirc$~Moderadamente profundo (51--80\,cm)\\
$\bigcirc$~Profundo (81--120\,cm) \quad $\bigcirc$~Muito profundo ($>$120\,cm)\\[4pt]
\textbf{Textura do solo (superficial)}\\
$\bigcirc$~Grossa/leve (arenosa) \quad $\bigcirc$~Média (franca, siltosa) \quad $\bigcirc$~Fina/pesada (argilosa)\\[4pt]
\textbf{Matéria orgânica no horizonte superficial}\\
$\bigcirc$~Alta ($>$3\%) \quad $\bigcirc$~Média (1--3\%) \quad $\bigcirc$~Baixa ($<$1\%)

\subsubsection*{5.4\quad Disponibilidade e qualidade da água}

\noindent\textbf{Nível do lençol freático}\quad $\bigcirc$~Superficial \quad $\bigcirc$~$<$5\,m \quad $\bigcirc$~5--50\,m \quad $\bigcirc$~$>$50\,m\\[4pt]
\textbf{Disponibilidade de águas superficiais}\\
$\bigcirc$~Excesso \quad $\bigcirc$~Boa (disponível o ano todo) \quad $\bigcirc$~Média (não disponível o ano todo) \quad $\bigcirc$~Pouca/nenhuma\\[4pt]
\textbf{Qualidade da água (não tratada)}\\
$\bigcirc$~Boa (potável) \quad $\bigcirc$~Pobre (requer tratamento) \quad $\bigcirc$~Apenas para uso agrícola \quad $\bigcirc$~Inutilizável\\[4pt]
Salinidade é um problema? $\bigcirc$~Sim \quad $\bigcirc$~Não\\
Inundação da área ocorre? $\bigcirc$~Sim ($\bigcirc$ frequente / $\bigcirc$ episódica) \quad $\bigcirc$~Não

\subsubsection*{5.5\quad Biodiversidade}

\textbf{Diversidade de espécies}\quad $\bigcirc$~Alta \quad $\bigcirc$~Média \quad $\bigcirc$~Baixa\\
\textbf{Diversidade de habitats}\quad $\bigcirc$~Alta \quad $\bigcirc$~Média \quad $\bigcirc$~Baixa

\subsubsection*{5.6\quad Características dos usuários da terra que aplicam a Tecnologia}

\noindent\textbf{Sedentário ou nômade}\quad $\bigcirc$~Sedentário \quad $\bigcirc$~Seminômade \quad $\bigcirc$~Nômade \quad $\bigcirc$~Outro\\[4pt]
\textbf{Orientação de mercado}\quad $\bigcirc$~Subsistência \quad $\bigcirc$~Misto \quad $\bigcirc$~Comercial\\[4pt]
\textbf{Nível relativo de riqueza}\quad $\bigcirc$~Muito pobre \quad $\bigcirc$~Pobre \quad $\bigcirc$~Médio \quad $\bigcirc$~Rico \quad $\bigcirc$~Muito rico\\[4pt]
\textbf{Individual ou grupo}\quad $\bigcirc$~Individual/domicílio \quad $\bigcirc$~Grupos/comunidade \quad $\bigcirc$~Cooperativa \quad $\bigcirc$~Empregado\\[4pt]
\textbf{Gênero}\quad $\square$~Mulheres \quad $\square$~Homens\\[4pt]
\textbf{Faixa etária}\quad $\square$~Crianças \quad $\square$~Jovens \quad $\square$~Meia-idade \quad $\square$~Idosos\\[4pt]
\textbf{Renda extra-propriedade}\quad $\bigcirc$~$<$10\% \quad $\bigcirc$~10--50\% \quad $\bigcirc$~$>$50\% da renda total\\[4pt]
\textbf{Mecanização}\quad $\bigcirc$~Manual \quad $\bigcirc$~Tração animal \quad $\bigcirc$~Mecanizada/motorizada

\subsubsection*{5.7\quad Área média de terra possuída/arrendada/utilizada pelos usuários}

$\bigcirc$~$<$0,5\,ha \quad $\bigcirc$~0,5--1\,ha \quad $\bigcirc$~1--2\,ha \quad $\bigcirc$~2--5\,ha \quad $\bigcirc$~5--15\,ha \quad $\bigcirc$~15--50\,ha\\
$\bigcirc$~50--100\,ha \quad $\bigcirc$~100--500\,ha \quad $\bigcirc$~500--1.000\,ha \quad $\bigcirc$~1.000--10.000\,ha \quad $\bigcirc$~$>$10.000\,ha\\[4pt]
Escala local:\quad $\bigcirc$~Pequena \quad $\bigcirc$~Média \quad $\bigcirc$~Grande

\subsubsection*{5.8\quad Propriedade da terra, direitos de uso da terra e direitos de uso da água}

\noindent\textbf{Propriedade da terra}\\
$\bigcirc$~Estado \quad $\bigcirc$~Empresa \quad $\bigcirc$~Comunal/vilarejo \quad $\bigcirc$~Grupo \quad $\bigcirc$~Individual sem título \quad $\bigcirc$~Individual com título \quad $\bigcirc$~Outro\\[4pt]
\textbf{Direitos de uso da terra}\\
$\bigcirc$~Acesso livre \quad $\bigcirc$~Comunal (organizado) \quad $\bigcirc$~Arrendado \quad $\bigcirc$~Individual \quad $\bigcirc$~Outro\\[4pt]
\textbf{Direitos de uso da água}\\
$\bigcirc$~Acesso livre \quad $\bigcirc$~Comunal (organizado) \quad $\bigcirc$~Arrendado \quad $\bigcirc$~Individual \quad $\bigcirc$~Outro\\[4pt]
Direitos de uso baseados em sistema jurídico tradicional? $\bigcirc$~Sim (especifique: \dotfill) $\bigcirc$~Não

\subsubsection*{5.9\quad Acesso a serviços e infraestrutura}

\begin{table}[htbp]
\centering\small
\caption{Acesso a serviços e infraestrutura}
\label{tab:app_services}
\begin{tabular}{lccc}
\toprule
\textbf{Serviço} & \textbf{Precário} & \textbf{Moderado} & \textbf{Bom} \\
\midrule
Saúde                         & $\bigcirc$ & $\bigcirc$ & $\bigcirc$ \\
Educação                      & $\bigcirc$ & $\bigcirc$ & $\bigcirc$ \\
Assistência técnica            & $\bigcirc$ & $\bigcirc$ & $\bigcirc$ \\
Emprego (p.\,ex.\@ extra-propriedade) & $\bigcirc$ & $\bigcirc$ & $\bigcirc$ \\
Mercados                      & $\bigcirc$ & $\bigcirc$ & $\bigcirc$ \\
Energia                       & $\bigcirc$ & $\bigcirc$ & $\bigcirc$ \\
Estradas e transporte          & $\bigcirc$ & $\bigcirc$ & $\bigcirc$ \\
Água potável e saneamento      & $\bigcirc$ & $\bigcirc$ & $\bigcirc$ \\
Serviços financeiros           & $\bigcirc$ & $\bigcirc$ & $\bigcirc$ \\
\bottomrule
\end{tabular}
\end{table}

%% =========================================================
\subsection*{6\quad Impactos e declarações conclusivas}

\textit{Avalie os impactos relevantes. Se dados medidos não estiverem disponíveis, forneça sua melhor estimativa. Utilize as colunas ``Quantifique antes/depois do MST'' e ``Comentários'' para demonstrar evidências.}

\subsubsection*{6.1\quad Impactos no local (\textit{on-site})}

\noindent\textbf{Impactos socioeconômicos}\\[2pt]
\textit{Produção:} produção agrícola, qualidade das culturas, produção de forragem, produção animal, produção de madeira, qualidade florestal, produtos florestais não madeireiros, risco de falha da produção, diversidade de produtos, área de produção, manejo da terra, geração de energia.\\[4pt]
\textit{Disponibilidade e qualidade da água:} disponibilidade/qualidade de água potável, água para pecuária, água para irrigação, demanda de irrigação.\\[4pt]
\textit{Renda e custos:} despesas com insumos agrícolas, renda, diversidade de fontes de renda, disparidades econômicas, carga de trabalho.\\[6pt]
\noindent\textbf{Impactos socioculturais}\\[2pt]
Segurança alimentar/autossuficiência, situação de saúde, direitos de uso da terra/água, oportunidades culturais (espirituais, religiosas, estéticas), oportunidades de lazer, instituições comunitárias, instituições nacionais, conhecimento em MST/degradação, mitigação de conflitos, situação de grupos socioeconomicamente desfavorecidos.\\[6pt]
\noindent\textbf{Impactos ecológicos}\\[2pt]
\textit{Ciclo hidrológico/escoamento:} quantidade e qualidade da água, captação/coleta de água, escoamento superficial, drenagem, nível do lençol freático, evaporação.\\[4pt]
\textit{Solo:} umidade do solo, cobertura do solo, perda de solo, acúmulo de solo, selamento/crusting, compactação, ciclagem de nutrientes, salinidade, matéria orgânica/carbono subsuperficial, acidez.\\[4pt]
\textit{Biodiversidade:} cobertura vegetal, biomassa/carbono aéreo, diversidade vegetal, espécies invasoras, diversidade animal, espécies benéficas, espécies nocivas, diversidade de habitats, pragas/doenças.\\[4pt]
\textit{Mudanças climáticas e redução de risco de desastres:} impactos de enchentes, deslizamentos, secas, ciclones/tempestades, emissões de GEE, risco de incêndio, velocidade do vento, microclima.

\medskip\noindent
\textit{Para cada impacto selecionado, classifique a magnitude em escala de 7 pontos (muito negativo a muito positivo) e quantifique antes/depois do MST quando possível.}

\subsubsection*{6.2\quad Impactos fora do local (\textit{off-site})}

Disponibilidade hídrica (lençol freático, nascentes), estabilidade de vazões, enchentes a jusante, assoreamento a jusante, poluição de águas subterrâneas/rios, capacidade de filtragem/tamponamento (solo, vegetação, áreas úmidas), sedimentos transportados pelo vento, danos a campos vizinhos, danos a infraestrutura pública/privada, impacto de gases de efeito estufa.

\subsubsection*{6.3\quad Exposição e sensibilidade da Tecnologia a mudanças climáticas graduais e extremos/desastres climáticos}

\textit{Indique mudanças graduais no clima e extremos/desastres observados pelos usuários nos últimos 10 anos.}

\medskip\noindent
\textbf{Mudanças climáticas graduais:} temperatura anual, temperatura sazonal, precipitação anual, precipitação sazonal.\\[4pt]
\textbf{Extremos climáticos (desastres):}\\
\textit{Meteorológicos:} tempestade tropical, ciclone extratropical, chuva intensa local, trovoada, granizo, tempestade de neve, tempestade de areia/poeira, vendaval, tornado.\\
\textit{Climatológicos:} onda de calor, onda de frio, condições extremas de inverno, seca, incêndio florestal, incêndio de campo.\\
\textit{Hidrológicos:} enchente fluvial, enxurrada, maré de tempestade, deslizamento/fluxo de detritos, avalanche.\\
\textit{Biológicos:} epidemias, infestações de insetos/vermes.\\[4pt]
\textit{Para cada evento, avalie a capacidade da Tecnologia de lidar com ele (muito precariamente a muito bem).}

\subsubsection*{6.4\quad Análise custo-benefício}

\textbf{Como os benefícios se comparam aos custos de implantação (perspectiva do usuário)?}\\
Retorno de curto prazo (1--3 anos):\quad $\bigcirc$ Muito negativo \quad $\bigcirc$ Negativo \quad $\bigcirc$ Levemente negativo \quad $\bigcirc$ Neutro \quad $\bigcirc$ Levemente positivo \quad $\bigcirc$ Positivo \quad $\bigcirc$ Muito positivo\\
Retorno de longo prazo ($>$10 anos):\quad $\bigcirc$ Muito negativo \quad $\bigcirc$ Negativo \quad $\bigcirc$ Levemente negativo \quad $\bigcirc$ Neutro \quad $\bigcirc$ Levemente positivo \quad $\bigcirc$ Positivo \quad $\bigcirc$ Muito positivo

\medskip\noindent
\textbf{Como os benefícios se comparam aos custos de manutenção (perspectiva do usuário)?}\\
Retorno de curto prazo:\quad $\bigcirc$ Muito negativo \quad ... \quad $\bigcirc$ Muito positivo\\
Retorno de longo prazo:\quad $\bigcirc$ Muito negativo \quad ... \quad $\bigcirc$ Muito positivo

\subsubsection*{6.5\quad Adoção da Tecnologia}

Quantos usuários da terra na área adotaram/implementaram a Tecnologia?\\
$\bigcirc$~Casos isolados/experimentais \quad $\bigcirc$~1--10\% \quad $\bigcirc$~10--50\% \quad $\bigcirc$~$>$50\%\\[4pt]
Se disponível, quantifique (nº de domicílios e/ou área coberta): \dotfill\\[4pt]
Dos que adotaram, quantos o fizeram espontaneamente (sem incentivos)?\\
$\bigcirc$~0--10\% \quad $\bigcirc$~10--50\% \quad $\bigcirc$~50--90\% \quad $\bigcirc$~90--100\%

\subsubsection*{6.6\quad Adaptação}

A Tecnologia foi modificada recentemente para adaptar-se a condições em mudança? $\bigcirc$~Sim \quad $\bigcirc$~Não\\[4pt]
Se sim, adaptada a:\\
$\bigcirc$~Mudanças/extremos climáticos \quad $\bigcirc$~Mercados em transformação \quad $\bigcirc$~Disponibilidade de mão de obra \quad $\bigcirc$~Outro: \dotfill\\
Especifique a adaptação: \dotfill

\subsubsection*{6.7\quad Pontos fortes/vantagens/oportunidades da Tecnologia}

\textit{Diferencie entre perspectivas de usuários e de pessoas-recurso.}\\[4pt]
\textbf{Perspectiva do usuário da terra:}\\
1)\,\dotfill \quad 2)\,\dotfill \quad 3)\,\dotfill \quad 4)\,\dotfill\\[4pt]
\textbf{Perspectiva do compilador/pessoa-recurso:}\\
1)\,\dotfill \quad 2)\,\dotfill \quad 3)\,\dotfill \quad 4)\,\dotfill

\subsubsection*{6.8\quad Pontos fracos/desvantagens/riscos e formas de superá-los}

\textbf{Perspectiva do usuário da terra:}\\
\begin{tabular}{p{7cm}p{7cm}}
Fraqueza/risco & Como superar? \\
1)\,\dotfill & 1)\,\dotfill \\
2)\,\dotfill & 2)\,\dotfill \\
3)\,\dotfill & 3)\,\dotfill \\
\end{tabular}

\medskip\noindent
\textbf{Perspectiva do compilador/pessoa-recurso:}\\
\begin{tabular}{p{7cm}p{7cm}}
Fraqueza/risco & Como superar? \\
1)\,\dotfill & 1)\,\dotfill \\
2)\,\dotfill & 2)\,\dotfill \\
3)\,\dotfill & 3)\,\dotfill \\
\end{tabular}

%% =========================================================
\subsection*{7\quad Referências e links}

\subsubsection*{7.1\quad Métodos/fontes de informação}
\textit{Várias respostas possíveis.}\\[4pt]
$\square$~Visitas/levantamentos de campo \quad Nº de informantes: \dotfill\\
$\square$~Entrevistas com usuários da terra \quad Nº de informantes: \dotfill\\
$\square$~Entrevistas com especialistas em MST \quad Nº de informantes: \dotfill\\
$\square$~Compilação de relatórios e documentação existente\\
$\square$~Outro (especifique): \dotfill\\[4pt]
Data da coleta de dados (em campo): \dotfill

\subsubsection*{7.2\quad Referências a publicações disponíveis}
\textit{Liste publicações relevantes. Envie cópias digitais ao banco de dados.}\\[4pt]
Título, autor, ano, ISBN: \dotfill\\
Disponível em: \dotfill\quad Custo: \dotfill

\subsubsection*{7.3\quad Links para informações relevantes disponíveis online}
Título/descrição: \dotfill\quad URL: \dotfill

\subsubsection*{7.4\quad Comentários gerais}
\dotfill\\[2pt]\dotfill

%% =========================================================
\subsection*{8\quad Anexo --- Listas de referência}

\textit{As listas abaixo são empregadas em campos de seleção do questionário (seções 3.2 e 3.3). Apresenta-se versão condensada.}

\begin{table}[htbp]
\centering\small
\caption{Culturas anuais (lista parcial WOCAT--IPCC)}
\label{tab:app_annual_crops}
\begin{tabular}{ll}
\toprule
\textbf{Grupo} & \textbf{Exemplos} \\
\midrule
Cereais & milho, arroz (várzea/sequeiro), sorgo, milheto, trigo, aveia, centeio, quinoa \\
Leguminosas e oleaginosas & feijão, lentilha, soja, ervilha, amendoim, girassol, mamona \\
Raízes e tubérculos & mandioca, batata, batata-doce, inhame, beterraba \\
Hortaliças & tomate, cebola, abóbora, berinjela, folhosas, cenoura \\
Fibras e flores & algodão, linho, cânhamo, roseiras \\
Forrageiras & alfafa, trevo, gramíneas \\
Medicinais/aromáticas & diversas \\
\bottomrule
\end{tabular}
\end{table}

\begin{table}[htbp]
\centering\small
\caption{Culturas perenes e arbóreo-arbustivas (lista parcial)}
\label{tab:app_perennial_crops}
\begin{tabular}{ll}
\toprule
\textbf{Grupo} & \textbf{Exemplos} \\
\midrule
Perenes não lenhosas & cana-de-açúcar, banana, abacaxi, sisal, capins forrageiros \\
Arbóreo-arbustivas & café, cacau, coco, dendê, manga, abacate, citros, castanhas \\
& oliveira, seringueira, teca, mogno, goiaba, maracujá \\
\bottomrule
\end{tabular}
\end{table}

\begin{table}[htbp]
\centering\small
\caption{Pecuária e produtos de pastagem}
\label{tab:app_livestock}
\begin{tabular}{ll}
\toprule
\textbf{Tipo} & \textbf{Produtos/serviços} \\
\midrule
Bovinos (leite, corte, trabalho) & carne, leite, couro, tração, esterco \\
Bubalinos, equinos, muares & transporte, tração \\
Suínos & carne \\
Caprinos, ovinos & carne, leite, lã, couro \\
Aves, coelhos & carne, ovos \\
Apicultura & mel, cera, pólen \\
Piscicultura & peixes \\
Fauna silvestre & herbívoros grandes/pequenos \\
\bottomrule
\end{tabular}
\end{table}


\end{document}