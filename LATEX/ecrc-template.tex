
% Template for Elsevier journal article Ecological Economics
% version 1.2 dated 09 May 2011

% This file (c) 2009-2011 Elsevier Ltd.  Modifications may be freely made,
% provided the edited file is saved under a different name

% This file contains modifications for Ecological Economics
% but may easily be adapted to other journals

% Changes since version 1.1
% - added "procedia" option compliant with ecrc.sty version 1.2a
%   (makes the layout approximately the same as the Word CRC template)
% - added example for generating copyright line in abstract

%-----------------------------------------------------------------------------------

%% This template uses the elsarticle.cls document class and the extension package ecrc.sty
%% For full documentation on usage of elsarticle.cls, consult the documentation "elsdoc.pdf"
%% Further resources available at http://www.elsevier.com/latex

%-----------------------------------------------------------------------------------

%%%%%%%%%%%%%%%%%%%%%%%%%%%%%%%%%%%%%%%%%%%%%%%%%%%%%%%%%%%%%%
%%%%%%%%%%%%%%%%%%%%%%%%%%%%%%%%%%%%%%%%%%%%%%%%%%%%%%%%%%%%%%
%%                                                          %%
%% Important note on usage                                  %%
%% -----------------------                                  %%
%% This file should normally be compiled with PDFLaTeX      %%
%% Using standard LaTeX should work but may produce clashes %%
%%                                                          %%
%%%%%%%%%%%%%%%%%%%%%%%%%%%%%%%%%%%%%%%%%%%%%%%%%%%%%%%%%%%%%%
%%%%%%%%%%%%%%%%%%%%%%%%%%%%%%%%%%%%%%%%%%%%%%%%%%%%%%%%%%%%%%

%% The '3p' and 'times' class options of elsarticle are used for Elsevier CRC
%% Add the 'procedia' option to approximate to the Word template
%\documentclass[3p,times,procedia]{elsarticle}
\documentclass[3p,times]{elsarticle}

%% The `ecrc' package must be called to make the CRC functionality available
\usepackage{ecrc}
\usepackage[utf8]{inputenc}
\usepackage[T1]{fontenc}
\usepackage[brazil]{babel}
\usepackage{amsmath}
\usepackage{booktabs}
\usepackage{multirow}
\usepackage[breaklinks]{hyperref}
\usepackage{enumitem}
\usepackage{xcolor}
\usepackage{tikz}
\usetikzlibrary{shapes.geometric, arrows.meta, positioning, fit, backgrounds, calc}

%% The ecrc package defines commands needed for running heads and logos.
%% For running heads, you can set the journal name, the volume, the starting page and the authors

%% set the volume if you know. Otherwise `00'
\volume{00}

%% set the starting page if not 1
\firstpage{1}

%% Give the name of the journal
\journalname{Ecological Economics}

%% Give the author list to appear in the running head
%% Example \runauth{C.V. Radhakrishnan et al.}
\runauth{Oliveira et al.}

%% The choice of journal logo is determined by the \jid and \jnltitlelogo commands.
%% A user-supplied logo with the name <\jid>logo.pdf will be inserted if present.
%% e.g. if \jid{yspmi} the system will look for a file yspmilogo.pdf
%% Otherwise the content of \jnltitlelogo will be set between horizontal lines as a default logo

%% Give the abbreviation of the Journal.  Contact the journal editorial office if in any doubt
\jid{ecolecon}

%% Give a short journal name for the dummy logo (if needed)
\jnltitlelogo{Ecological Economics}

%% Provide the copyright line to appear in the abstract
%% Usage:
%   \CopyrightLine[<text-before-year>]{<year>}{<restt-of-the-copyright-text>}
%   \CopyrightLine[Crown copyright]{2011}{Published by Elsevier Ltd.}
%   \CopyrightLine{2011}{Elsevier Ltd. All rights reserved}
\CopyrightLine{2026}{Published by Elsevier B.V. All rights reserved.}

%% Hereafter the template follows `elsarticle'.
%% For more details see the existing template files elsarticle-template-harv.tex and elsarticle-template-num.tex.

%% Elsevier CRC generally uses a numbered reference style
%% For this, the conventions of elsarticle-template-num.tex should be followed (included below)
%% If using BibTeX, use the style file elsarticle-num.bst

%% End of ecrc-specific commands
%%%%%%%%%%%%%%%%%%%%%%%%%%%%%%%%%%%%%%%%%%%%%%%%%%%%%%%%%%%%%%%%%%%%%%%%%%

%% The amssymb package provides various useful mathematical symbols
\usepackage{amssymb}
%% The amsthm package provides extended theorem environments
%% \usepackage{amsthm}

%% The lineno packages adds line numbers. Start line numbering with
%% \begin{linenumbers}, end it with \end{linenumbers}. Or switch it on
%% for the whole article with \linenumbers after \end{frontmatter}.
%% \usepackage{lineno}

%% natbib.sty is loaded by default. However, natbib options can be
%% provided with \biboptions{...} command. Following options are
%% valid:

%%   round  -  round parentheses are used (default)
%%   square -  square brackets are used   [option]
%%   curly  -  curly braces are used      {option}
%%   angle  -  angle brackets are used    <option>
%%   semicolon  -  multiple citations separated by semi-colon
%%   colon  - same as semicolon, an earlier confusion
%%   comma  -  separated by comma
%%   numbers-  selects numerical citations
%%   super  -  numerical citations as superscripts
%%   sort   -  sorts multiple citations according to order in ref. list
%%   sort&compress   -  like sort, but also compresses numerical citations
%%   compress - compresses without sorting
%%
%% \biboptions{comma,round}

% \biboptions{}

% \usepackage[figuresright]{rotating}  % removido: sem tabelas paisagem

% put your own definitions here:
%   \newcommand{\cZ}{\cal{Z}}
%   \newtheorem{def}{Definition}[section]
%   ...

% add words to TeX's hyphenation exception list
%\hyphenation{author another created financial paper re-commend-ed Post-Script}

% declarations for front matter

\begin{document}

\begin{frontmatter}

%% Title, authors and addresses

%% use the tnoteref command within \title for footnotes;
%% use the tnotetext command for the associated footnote;
%% use the fnref command within \author or \address for footnotes;
%% use the fntext command for the associated footnote;
%% use the corref command within \author for corresponding author footnotes;
%% use the cortext command for the associated footnote;
%% use the ead command for the email address,
%% and the form \ead[url] for the home page:
%%
%% \title{Title\tnoteref{label1}}
%% \tnotetext[label1]{}
%% \author{Name\corref{cor1}\fnref{label2}}
%% \ead{email address}
%% \ead[url]{home page}
%% \fntext[label2]{}
%% \cortext[cor1]{}
%% \address{Address\fnref{label3}}
%% \fntext[label3]{}

\dochead{}

\title{Indicador de resiliência biocultural para comunidades quilombolas via WOCAT-SLM adaptado}

\author[a]{Catuxe Varjão de Santana Oliveira\corref{cor1}}
\ead{catuxe@academico.ufs.br}
\author[b]{Luiz Diego Vidal Santos}
\ead{ldvsantos@uefs.br}
\author[a]{XXXXXXX}

\cortext[cor1]{Autor correspondente.}
\address[a]{Programa de Pós-Graduação em Ciência da Propriedade Intelectual (PPGPI), Universidade Federal de Sergipe (UFS), São Cristóvão, SE, Brasil}
\address[b]{Universidade Estadual de Feira de Santana (UEFS), Feira de Santana, BA, Brasil}

\begin{abstract}
Indicadores convencionais de sustentabilidade raramente capturam sinergias entre conservação do solo, coesão sociocultural e serviços ecossistêmicos em territórios tradicionais. Este estudo desenvolve e valida o Índice de Resiliência Biocultural Integrada (IRBI) mediante cadeia metodológica em três fases aplicada a Comunidades Quilombolas (CQ) do semiárido baiano. Na Fase~1, o questionário WOCAT-SLM foi adaptado transculturalmente via protocolo ITC, gerando a versão WOCAT-SLM-QBR com 68 itens traduzidos e oito suplementares (IVC = 0,93, kappa = 0,78, compreensão = 87\%). Na Fase~2, painel Delphi ($n=21$) estabilizou 26 variáveis linguísticas em seis dimensões ($W = 0,74$, $CVC = 0,84$), com triangulação por 18 entrevistas ($r = 0,68$, $p < 0,01$). Na Fase~3, as variáveis foram mapeadas em funções de pertinência triangulares para sistema Mamdani, verificado por análise de sensibilidade global (Morris). O IRBI integra dimensões biofísicas, socioculturais, econômicas, institucionais, adaptativas e organizacionais em índice composto apto a monitoramento periódico. A análise de sensibilidade confirmou que a variância não é monopolizada por parâmetros biofísicos, preservando peso equivalente do componente cultural.
\end{abstract}

\begin{keyword}
Biocultural valuation \sep Socioecological indicator \sep Fuzzy inference \sep Cross-cultural adaptation \sep Delphi method \sep Quilombola communities \sep Traditional ecological knowledge
\end{keyword}

\end{frontmatter}

\linenumbers


\section{Introdução}
\label{sec:intro}

O monitoramento da sustentabilidade em territórios governados por saberes tradicionais enfrenta lacuna estruturante, pois indicadores convencionais apreendem variáveis biofísicas isoladas sem capturar as sinergias entre conservação do solo, coesão sociocultural e serviços ecossistêmicos que caracterizam sistemas agroflorestais de comunidades quilombolas \cite{Costanza1997,Dale2001}. Essa insuficiência de métricas integradas compromete tanto a gestão sustentável da terra (SLM) quanto a formulação de políticas de proteção do patrimônio biocultural, dado que decisões baseadas em indicadores parciais subestimam sistematicamente o capital cultural como componente de resiliência biocultural \cite{Berkes2017}. Ao mesmo tempo, o campo da economia ecológica carece de protocolos operacionalizáveis que convertam conhecimento tácito de comunidades tradicionais em variáveis mensuráveis compatíveis com frameworks de valoração alternativa, limitação que perpetua a exclusão dos Saberes e Sistemas Agrícolas Tradicionais (SSAT) dos circuitos formais de contabilização de riqueza natural \cite{Costanza1997}. Em sistemas quilombolas, o componente biofísico representa apenas um vetor dentro de matriz multidimensional em que cultura, governança e economia desempenham papéis equivalentes na manutenção da resiliência socioecológica \cite{Toledo2008}. Desenvolver indicador socioecológico composto que traduza, com rastreabilidade e rigor psicométrico, o conhecimento tácito de comunidades tradicionais em variáveis mensuráveis e monitoráveis ao longo do tempo constitui, portanto, desafio científico com implicações diretas para a governança adaptativa de sistemas socioecológicos e para a integração de ativos bioculturais na contabilidade de riqueza nacional \cite{Niemeijer2008}.

O questionário WOCAT (\textit{World Overview of Conservation Approaches and Technologies}), adotado como ferramenta oficial pela FAO, UNCCD e rede global de parceiros \cite{Liniger2019,Schwilch2012}, representa o framework mais consolidado para documentação de tecnologias SLM, com aplicações em mais de 120 países e banco de dados contendo mais de 2.000 tecnologias catalogadas. Sua arquitetura padronizada em sete seções abrange classificação técnica, avaliação de impactos socioeconômicos e ecológicos, custos, ambiente biofísico e governança dos usuários da terra. Contudo, a aplicação direta desse instrumento a comunidades quilombolas do semiárido nordestino esbarra em barreiras de equivalência que operam simultaneamente nos planos semântico-idiomático, experiencial e conceitual \cite{Herdman1999}. No plano semântico, terminologia agronômica sem paralelo direto no vocabulário etnotaxonômico quilombola \cite{Rist2006,Quave2014} compromete a compreensão, incompatibilidade que se amplifica no plano experiencial porque categorias de resposta pressupõem contextos fundiários formalizados enquanto comunidades quilombolas operam sob regimes coletivos com reconhecimento jurídico precário \cite{Almeida2011}. No plano conceitual, a racionalidade agronômica ocidental subjacente ao WOCAT não contempla dimensões espirituais, rituais e simbólicas constitutivas do manejo quilombola \cite{Santos2007,Toledo2008}.

A literatura sobre equivalência transcultural \cite{Guillemin1993} demonstra que aplicar instrumentos sem adaptação formal produz viés sistemático de mensuração, risco que assume magnitude crítica neste estudo porque o WOCAT-SLM adaptado servirá como template de referência para elicitação Delphi e, indiretamente, para a construção do Índice de Valoração Bioeconômica (IVB) via lógica fuzzy. Paralelamente, a economia do conhecimento enfrenta paradoxo estrutural, pois, enquanto ativos intangíveis constituem o principal motor de competitividade \cite{Edvinsson1997}, os Saberes e Sistemas Agrícolas Tradicionais (SSAT), que atendem rigorosamente aos critérios VRIN de \cite{Barney1991}, permanecem excluídos dos circuitos formais de valoração econômica por ausência de protocolos que convertam saberes tácitos \cite{Polanyi1966} em variáveis auditáveis.

Para suprir essa lacuna de monitoramento e mensuração, o presente estudo propõe cadeia metodológica tripartite cujo produto final é o IVB, indicador socioecológico de resiliência biocultural projetado para integrar dimensões biofísicas, socioculturais, econômicas, institucionais, adaptativas e organizacionais em índice único, apto a monitoramento periódico e compatível com sistemas de informação geográfica. A Fase~1 adapta transculturalmente o WOCAT-SLM via protocolo ITC \cite{Beaton2000} em seis etapas, a Fase~2 elicita e consensua 26 variáveis linguísticas mediante Delphi estruturado com triangulação qualitativa e a Fase~3 mapeia as variáveis consensuadas em funções de pertinência para sistema de inferência fuzzy Mamdani, com verificação de robustez por análise de sensibilidade global.

A questão norteadora indaga \textit{em que medida a integração sequencial de adaptação transcultural, Delphi e lógica difusa produz indicador socioecológico composto capaz de capturar, com validade psicométrica e sensibilidade cultural, a multidimensionalidade de sistemas agroflorestais quilombolas para fins de monitoramento e gestão adaptativa?} A hipótese postula que a versão adaptada (WOCAT-SLM-QBR) atingirá IVC~$\geq$~0,80 e kappa~$\geq$~0,70, que o Delphi produzirá consenso forte ($W \geq 0,70$, $CVC \geq 0,80$) e que o IVB resultante apresentará robustez numérica verificável mediante análise de sensibilidade.


\section{Referencial Teórico}
\label{sec:theory}

\subsection{Equivalência Transcultural e Validade de Construto}

Transpor instrumentos entre contextos culturais exige procedimento que vai além da tradução linguística. O modelo hierárquico de \cite{Herdman1999} formaliza essa exigência ao estratificar a equivalência transcultural em seis níveis progressivos, da equivalência conceitual e de itens até a equivalência funcional, perpassando as dimensões semântica, operacional e de mensuração. A teoria psicométrica clássica, consolidada por \cite{Nunnally1978}, postula que a validade de um instrumento repousa sobre três pilares interdependentes (conteúdo, construto e critério), cujo atendimento torna-se exponencialmente complexo quando o objeto mensurado é culturalmente contingente, como ocorre com saberes tradicionais que resistem à decomposição em itens discretos \cite{Polanyi1966}.

No cerne dessa complexidade situa-se a tensão entre abordagens \textit{etic} (universalista) e \textit{emic} (culturalmente específica) descrita por \cite{Pike1967}. Instrumentos com orientação predominantemente \textit{etic}, como o WOCAT, pressupõem categorias universalmente aplicáveis que viabilizam comparabilidade internacional à custa de obscurecer categorias \textit{emic} significativas. Agricultores quilombolas, por exemplo, classificam terras por atributos espirituais ou memória social, categorias invisíveis ao instrumento original \cite{Toledo2008}. Preservar a dimensão \textit{etic} que confere comparabilidade e, concomitantemente, incorporar dimensões \textit{emic} que conferem validade ecológica define o duplo desafio conceitual subjacente à adaptação de instrumentos entre epistemologias distintas.

Essa dualidade remete ao problema fundamental da comensurabilidade entre paradigmas, formulado por \cite{Kuhn1962} e revisitado na literatura sobre pluralismo epistemológico em ciência da sustentabilidade \cite{Miller2008}. A questão não é meramente técnica (traduzir termos), mas ontológica (negociar o que conta como conhecimento válido entre matrizes culturais que operam com categorias parcialmente sobrepostas e parcialmente incomensuráveis). A equivalência funcional, nível mais elevado do modelo de \cite{Herdman1999}, somente se verifica quando o instrumento adaptado desempenha papel análogo ao original na cultura de destino, critério que, no caso de sistemas socioecológicos complexos, implica que o instrumento deve capturar não apenas variáveis biofísicas mas também a rede de significados culturais que confere coerência ao sistema de manejo \cite{Guillemin1993,Beaton2000}.

\subsection{Epistemologia do Conhecimento Tácito e Modos de Conversão}

Desde a formulação seminal de \cite{Polanyi1966}, segundo a qual ``sabemos mais do que podemos dizer'' (\textit{we can know more than we can tell}), a distinção entre conhecimento tácito e explícito orienta investigações sobre codificação de saberes em contextos científicos, organizacionais e culturais. Em sistemas agroecológicos tradicionais, essa fronteira torna-se especialmente opaca, dado que manejo fenológico, leitura de sinais climáticos, seleção de variedades adaptadas e práticas de conservação de solos são transmitidos oralmente e pela prática cotidiana, resistindo à decomposição em categorias discretas \cite{Berkes2017,Toledo2008}.

A espiral SECI de \cite{Nonaka1995} (Socialização, Externalização, Combinação e Internalização) oferece arcabouço conceitual robusto para modelar a conversão progressiva entre modos de conhecimento. A Socialização opera pela transmissão corpo-a-corpo de saberes tácitos entre indivíduos, a Externalização traduz esse saber tácito em linguagem articulada (conceitos, metáforas, modelos), a Combinação sistematiza o conhecimento explicitado em corpos estruturados (bancos de dados, indicadores) e a Internalização reconverte o conhecimento sistematizado em prática incorporada. Para saberes tradicionais, a transição entre Socialização e Externalização constitui o gargalo epistêmico crítico, porquanto categorias experienciais (percepção tátil do solo, leitura de nuvens, reconhecimento de fenofases) precisam ser articuladas em linguagem padronizada sem perda de significado substancial.

\cite{Davenport1998} complementam essa perspectiva ao argumentar que o conhecimento, diferentemente da informação, é contextual, experiencial e mediado por julgamento, propriedades que dificultam sua transferência por meios puramente documentais. Em comunidades tradicionais, onde o conhecimento é socialmente distribuído e ritualmente regulado, a conversão exige protocolos conceituais que respeitem simultaneamente rigor científico e legitimidade cultural, sob pena de produzir artefatos documentais sem correspondência com a prática viva \cite{Santos2007}. A governança dessa conversão mobiliza a ISO~30401:2018, que reconhece formalmente que o valor do conhecimento depende de cultura, processos e aprendizagem organizacional, e não apenas de codificação estática \cite{ISO30401}, perspectiva que amplia o escopo da gestão do conhecimento para além do ambiente corporativo.

\subsection{Ativos Intangíveis, Valoração Não Monetária e Capital Biocultural}

A economia do conhecimento transformou a compreensão sobre as fontes de valor ao demonstrar que ativos intangíveis, isto é, recursos não físicos geradores de benefícios futuros, superam os ativos tangíveis como fator determinante de competitividade e criação de riqueza \cite{Lev2001}. \cite{Sveiby1997} formalizou a tipologia do capital intelectual em três componentes (capital humano, capital estrutural e capital relacional), enquanto \cite{Edvinsson1997} demonstrou que o valor de mercado das organizações excede consistentemente o valor contábil precisamente porque a contabilidade tradicional falha em capturar ativos intangíveis. Essa invisibilidade contábil, amplamente documentada no contexto corporativo, assume proporções críticas quando transposta para comunidades tradicionais, cujos ativos mais valiosos (saberes de manejo, variedades agrícolas selecionadas ao longo de gerações, instituições comunitárias de governança, redes de reciprocidade) são, por definição, intangíveis e não monetizados.

Paralelamente, a economia ecológica desenvolveu marcos conceituais robustos para valoração de bens e serviços que escapam aos mecanismos de preço. O trabalho seminal de \cite{Costanza1997}, ao estimar o valor dos serviços ecossistêmicos globais, evidenciou que a riqueza natural constitui fluxo indispensável à economia humana mesmo quando invisível ao PIB. \cite{PearceTurner1990} formalizaram o conceito de Valor Econômico Total (VET), decomposto em valores de uso direto, uso indireto, opção e existência, arcabouço que permite reconhecer que sistemas agroflorestais tradicionais geram múltiplos tipos de valor simultaneamente, desde provisão alimentar até regulação microclimática, desde manutenção de agrobiodiversidade até coesão identitária comunitária. A iniciativa TEEB consolidou esses avanços em framework aplicável a decisões de política pública, demonstrando que a inação decorrente da não valoração gera custos econômicos superiores aos da conservação \cite{TEEB2010}.

O conceito de capital biocultural, desenvolvido na interface entre ecologia, antropologia e economia ecológica \cite{Maffi2001,Pretty2009}, designa o acervo integrado de diversidade biológica e cultural que comunidades tradicionais acumulam e transmitem ao longo de gerações mediante co-evolução entre práticas de manejo e processos ecossistêmicos. Diferentemente do capital natural puramente biofísico, o capital biocultural incorpora dimensão epistêmica (saberes que codificam as relações entre espécies, solo, clima e manejo), dimensão normativa (instituições, rituais e crenças que regulam o acesso e uso dos recursos) e dimensão relacional (redes de troca, cooperação e transmissão intergeracional que asseguram resiliência do sistema como um todo). Construir indicador socioecológico composto que capture essa multidimensionalidade sem colapsar dimensões qualitativamente distintas em unidade monetária única constitui desafio teórico que exige articulação entre teoria do capital intelectual, economia ecológica e modelagem de incerteza \cite{MartinezAlier2002}.

\subsection{O Framework WOCAT: Arquitetura Conceitual e Potencial Analítico}

O \textit{World Overview of Conservation Approaches and Technologies} (WOCAT) constitui o principal framework internacional para documentação padronizada de Tecnologias de Manejo Sustentável da Terra (\textit{Sustainable Land Management}, SLM), desenvolvido pelo Centre for Development and Environment da Universidade de Berna e endossado pela Convenção das Nações Unidas de Combate à Desertificação (UNCCD) como ferramenta oficial de sistematização de boas práticas \cite{Liniger2019}. Desde sua criação em 1992, o WOCAT acumulou mais de 2.000 tecnologias documentadas em 120 países, consolidando-se como o mais amplo repositório global de evidências sobre conservação de solos e água em contextos de degradação da terra. A relevância do framework para a economia ecológica reside precisamente nessa capacidade de converter práticas locais de manejo, frequentemente não monetizadas e invisíveis às contas nacionais, em registros sistematizados que viabilizam análises comparativas de custo-benefício e impacto socioambiental.

A arquitetura modular do questionário WOCAT para Tecnologias SLM organiza-se em sete seções funcionalmente encadeadas que percorrem a cadeia completa, desde a identificação e localização georreferenciada da tecnologia (\S1) até o registro de fontes, instituições e processos de governança envolvidos (\S7). O núcleo conceitual do instrumento concentra-se nas seções intermediárias, onde a descrição técnica e classificação de medidas (\S2) fornece a base taxonômica que alimenta a tipificação de uso da terra, processos de degradação e funções protetoras (\S3), enquanto a contabilização de insumos e custos de estabelecimento e manutenção (\S4) articula-se com o perfil biofísico e socioeconômico dos usuários e do ambiente natural (\S5). A análise convergente de impactos ecológicos, socioeconômicos e socioculturais, combinada com avaliação de custo-benefício (\S6), fecha o circuito avaliativo e gera evidências para tomada de decisão em múltiplas escalas.

Essa organização confere ao WOCAT uma propriedade analítica frequentemente subutilizada na literatura, pois a cobertura simultânea de dimensões técnicas, ecológicas, econômicas e institucionais permite derivar construtos avaliativos multidimensionais que transcendem a finalidade original de documentação de tecnologias isoladas. A Tabela~\ref{tab:wocat_delphi} explicita o campo de construtos teoricamente deriváveis da arquitetura modular, organizados em seis dimensões complementares que abrangem desde a esfera cultural-simbólica até a capacidade adaptativa e a organização social, evidenciando que o instrumento contém, de forma latente, a matéria-prima conceitual para composição de indicadores socioecológicos integrados de valoração biocultural.

\begin{table}[htbp]
\centering
\caption{Construtos deriváveis da arquitetura modular WOCAT para avaliação biocultural multidimensional.}
\label{tab:wocat_delphi}
\small
\begin{tabular}{p{3.0cm}p{3.5cm}p{4.5cm}}
\toprule
\textbf{Dimensão avaliativa} & \textbf{Seções WOCAT} & \textbf{Construtos deriváveis} \\
\midrule
Cultural-simbólica & §2 Descrição; §6.1 Impactos socioculturais & Autenticidade, significado ritual, transmissão intergeracional \\
Biofísica-ambiental & §3 Classificação; §5 Ambiente natural; §6.1 Impactos ecológicos & Agrobiodiversidade, resiliência edáfica, cobertura vegetal \\
Econômica-mercadológica & §4 Insumos e custos; §6.1 Impactos socioeconômicos & Custo de reposição, diversificação de renda, potencial de mercado \\
Institucional-governança & §5.6 Características; §5.8 Propriedade; §6.5 Adoção & Regime fundiário, organização comunitária, acesso a serviços \\
Adaptativa-resiliência & §3.8 Prevenção; §6.3 Exposição climática & Capacidade adaptativa, resposta a secas, estabilidade \\
Social-organizacional & §5.9 Infraestrutura; §6.1 Instituições comunitárias & Redes de cooperação, capital social, equidade de gênero \\
\bottomrule
\end{tabular}
\end{table}

\subsection{Comunidades Quilombolas e Especificidades Ontológicas dos SSAT}

Sob lógica estruturalmente distinta da racionalidade agronômica convencional, os SSAT quilombolas são governados pelo complexo Conhecimento-Prática-Crença (K-P-B) descrito por \cite{Toledo2008}, onde crenças funcionam como regulador ético-cosmológico do manejo, e não como superstição residual passível de expurgo. De forma complementar, \cite{Berkes2017} demonstra que esses sistemas exemplificam manejo adaptativo de longa duração em que experimentação empírica é modulada por instituições sociais e cosmovisão, configurando o que \cite{Maffi2001} designa patrimônio biocultural, isto é, acervo de diversidade biológica inextricavelmente entrelaçado a diversidade linguística, cultural e espiritual. A convergência entre esses autores evidencia que a unidade de análise relevante não é a ``tecnologia agrícola'' isolada, mas o sistema socioecológico completo que inclui solo, planta, clima, saber, ritual e governança comunitária como componentes co-dependentes \cite{Pretty2009}.

O WOCAT original, concebido para documentar tecnologias de manejo em contextos de agricultura convencional ou semi-convencional, falha em capturar dimensões constitutivas dos SSAT quilombolas. A esfera espiritual-ritual, que integra bênçãos sobre sementes, rituais de plantio sincronizados com ciclos lunares e proibições em datas sagradas, permanece invisível às categorias do instrumento, lacuna que se estende à transmissão intergeracional via oralidade, visto que o WOCAT documenta a tecnologia como produto acabado sem registrar o processo de transmissão oral que opera como mecanismo central de inovação dos SSAT \cite{Polanyi1966}. Adicionalmente, a lógica coletiva-comunitária (mutirões, trocas de sementes, manejo comunitário de áreas de uso comum) opera sob governança de bens comuns conforme a teoria de \cite{Ostrom1990} e escapa à arquitetura do instrumento, desenhada para documentar práticas individualizadas. Essa tríplice lacuna (ritualidade, oralidade, coletividade) configura o que \cite{Santos2007} denomina injustiça cognitiva, em que instrumentos construídos sob epistemologia dominante invisibilizam sistematicamente os saberes que não se conformam às suas categorias.

Os critérios VRIN de \cite{Barney1991} (valioso, raro, inimitável e não substituível), originalmente formulados na teoria da firma baseada em recursos (\textit{Resource-Based View}), aplicam-se rigorosamente aos SSAT quilombolas, conferindo-lhes estatuto teórico de ativo intangível estratégico. Esses saberes são valiosos porque sustentam sistemas produtivos resilientes sob condições edafoclimáticas adversas, raros porque resultam de séculos de co-adaptação entre comunidades específicas e seus territórios, inimitáveis porque a trama de relações sociais e espirituais que os sustenta não pode ser replicada artificialmente, e não substituíveis porque desempenham funções ecológicas e culturais sem equivalente funcional disponível. A aplicação da RBV a ativos bioculturais comunitários revela que a ausência de protocolos de valoração não indica ausência de valor, mas sim falha dos instrumentos de mensuração em apreender categorias de riqueza que operam fora da lógica de mercado \cite{Lev2001,Edvinsson1997}.

\subsection{Indicadores Socioecológicos, Framework DPSIR e Lógica Difusa}

A construção de indicadores socioecológicos integrados enfrenta desafios conceituais que transcendem a mera agregação de variáveis biofísicas e sociais. \cite{Dale2001} argumentam que indicadores ecológicos úteis devem satisfazer critérios de relevância, praticabilidade, suficiência e responsividade temporal, enquanto \cite{Niemeijer2008} demonstram que a seleção de indicadores deve ser guiada por framework causal que explicite as relações entre variáveis, evitando o risco de compor índices estatisticamente robustos mas conceitualmente opacos.

O framework DPSIR (\textit{Driving forces, Pressures, State, Impact, Responses}), amplamente empregado pela Agência Europeia do Meio Ambiente \cite{Smeets1999}, confere rastreabilidade causal à modelagem de sistemas socioecológicos complexos. Em sistemas como os SSAT, as forças motrizes correspondem às pressões econômicas e institucionais sobre os sistemas tradicionais, o estado (\textit{state}) descreve condição integrada onde práticas culturais modulam processos ecossistêmicos e serviços ambientais, o impacto (\textit{impact}) manifesta-se na capacidade adaptativa e coesão social e as respostas (\textit{responses}) materializam-se nas intervenções de gestão informadas pelo indicador composto. A articulação com o conceito de resiliência socioecológica, formalizado por \cite{Holling1973} e expandido para sistemas acoplados humano-natureza por \cite{Folke2010}, sustenta que indicadores devem capturar não apenas o estado estático do sistema mas sua capacidade de absorver perturbações e reorganizar-se mantendo função e identidade, propriedade que distingue um indicador de monitoramento genuíno de uma fotografia estática sem valor preditivo.

A teoria dos conjuntos difusos, formulada por \cite{Zadeh1965} e operacionalizada em sistemas de inferência por \cite{Mamdani1975}, oferece arcabouço matemático particularmente adequado para modelar a incerteza inerente a indicadores que combinam variáveis qualitativas e quantitativas provenientes de saberes tradicionais. Diferentemente da lógica booleana que impõe fronteiras rígidas entre categorias (pertence/não pertence), a lógica difusa permite graus de pertinência no intervalo $[0,1]$, propriedade que se alinha à natureza gradual e contextual dos julgamentos em sistemas socioecológicos. A superioridade conceitual de abordagens fuzzy para avaliação ambiental reside na capacidade de incorporar variáveis linguísticas (\textit{alto}, \textit{médio}, \textit{baixo}) provenientes de julgamento especializado sem exigir sua conversão forçada em escalas métricas, preservando a riqueza semântica original enquanto viabiliza computação e agregação. Para indicadores compostos de capital biocultural, onde dimensões como ``autenticidade ritual'' ou ``transmissão intergeracional'' resistem à mensuração numérica direta, a modelagem fuzzy constitui não apenas conveniência técnica mas necessidade epistemológica, porquanto impor precisão numérica artificial a construtos inerentemente vagos produziria certeza espúria que comprometeria a validade do indicador \cite{Zadeh1965}.

\subsection{Capacidade Absortiva Reversa e Fluxo Epistêmico Invertido}

A noção de \textit{capacidade absortiva}, formulada por \cite{CohenLevinthal1990} para descrever a aptidão organizacional de reconhecer, assimilar e aplicar conhecimento externo, opera nos contextos interculturais de valoração biocultural em sentido inverso ao convencionalmente descrito na literatura de gestão da inovação. São as instituições formais (universidades, agências de extensão, marcos regulatórios) que precisam desenvolver capacidade absortiva para reconhecer, assimilar e incorporar o conhecimento das comunidades tradicionais, e não o contrário \cite{Santos2007}. \cite{Lev2001} demonstra que organizações que não investem em captura de ativos intangíveis incorrem em perda progressiva de valor, argumento que, transposto para o contexto de comunidades tradicionais sob pressão modernizante, assume caráter de urgência preservacionista.

Essa inversão do fluxo epistêmico implica que a adaptação de instrumentos de avaliação não constitui concessão metodológica ou simplificação didática, mas reconhecimento ontológico de que o conhecimento relevante para valoração de capital biocultural encontra-se nos detentores de saberes tácitos e não nos sistemas formais de monitoramento. A espiral SECI de \cite{Nonaka1995}, nesse contexto, opera sob condição de contorno distinta da imaginada pela teoria organizacional original, porquanto a Socialização ocorre em comunidades com cosmovisão própria, a Externalização exige tradução entre epistemologias, a Combinação precisa preservar multidimensionalidade que resiste à redução e a Internalização deve retornar às comunidades sob forma que reconheça sua autoria epistêmica \cite{Santos2007,FalsBorda1991}.


\section{Materiais e Métodos}
\label{sec:methods}

\subsection{Delineamento Geral}

Esta investigação configura-se como estudo metodológico de métodos mistos \cite{Creswell2018}, organizado em três fases sequenciais integradas segundo boas práticas de desenvolvimento de indicadores socioecológicos \cite{Dale2001,Niemeijer2008}. A Fase~1 compreende a Adaptação Transcultural do WOCAT-SLM (protocolo de \cite{Beaton2000}), a Fase~2 refere-se à Elicitação Delphi com triangulação qualitativa e a Fase~3 abrange o mapeamento fuzzy e análise de sensibilidade global do IRBI. Cada variável componente do indicador atende aos critérios SMART (específica, mensurável, atingível, relevante e temporal), verificados durante o ciclo Delphi mediante avaliação explícita de clareza e operacionalidade. A integração segue delineamento sequencial, de modo que o instrumento adaptado na Fase~1 serve como template de referência para as dimensões avaliativas da Fase~2, cujas variáveis consensuadas alimentam diretamente as funções de pertinência da Fase~3. A triangulação entre dados qualitativos (entrevistas), métricas estatísticas (Delphi) e funções de pertinência difusa opera como salvaguarda contra viés cultural, assegurando que nenhuma dimensão do indicador dependa de fonte única de evidência.

Importa destacar que a seleção de variáveis componentes do IRBI obedeceu ao critério de representatividade biocultural e não à hierarquia de disponibilidade de dados biofísicos. Cada dimensão (cultural-simbólica, biofísica-ambiental, econômica-mercadológica, institucional-governança, adaptativa-resiliência e social-organizacional) recebeu tratamento equiponderado na calibração fuzzy, cujos parâmetros de pertinência derivam exclusivamente das distribuições empíricas do consenso Delphi. Dessa forma, a lógica fuzzy atribui pesos equivalentes às dimensões mediante calibração derivada do consenso especializado, garantindo que a inovação central do indicador reside na capacidade de converter intangíveis culturais em métricas auditáveis sem subordinar dimensões socioculturais a componentes biofísicos. A Figura~\ref{fig:flowchart} apresenta o fluxo geral do estudo.

\begin{figure*}[!htbp]
\centering
\begin{tikzpicture}[
  node distance=0.4cm and 0.2cm,
  every node/.style={font=\footnotesize},
  phase/.style={rectangle, rounded corners=3pt, draw=black!70, fill=black!8,
    minimum height=0.55cm, text width=2.1cm, align=center, font=\footnotesize\bfseries},
  stepbox/.style={rectangle, rounded corners=2pt, draw=black!60, fill=white,
    minimum height=0.7cm, text width=1.85cm, align=center, font=\scriptsize},
  output/.style={rectangle, rounded corners=2pt, draw=black!80, fill=black!12,
    minimum height=0.7cm, text width=2.0cm, align=center,
    font=\scriptsize\bfseries},
  arr/.style={-{Stealth[length=2.5pt]}, thick, black!65},
  dashline/.style={-{Stealth[length=2.5pt]}, thick, black!50, dashed},
  phaselabel/.style={font=\scriptsize\itshape, text=black!60}
]
% ---- FASE 1 ----
\node[phase] (f1) {Fase 1\\Adapta\c{c}\~ao Transcultural};
\node[stepbox, right=0.5cm of f1]  (e1) {Etapa 1\\Tradu\c{c}\~ao\\(T1 + T2)};
\node[stepbox, right=of e1]        (e2) {Etapa 2\\S\'intese\\(T-12)};
\node[stepbox, right=of e2]        (e3) {Etapa 3\\Retrotradu\c{c}\~ao\\(BT1 + BT2)};
\draw[arr] (f1) -- (e1);
\draw[arr] (e1) -- (e2);
\draw[arr] (e2) -- (e3);
% segunda linha fase 1
\node[stepbox, below=0.55cm of e1]  (e4) {Etapa 4\\Comit\^e\\(10 membros)};
\node[stepbox, right=of e4]        (e5) {Etapa 5\\Pr\'e-teste\\($n=30$)};
\node[output, right=of e5]      (e6) {Etapa 6\\WOCAT-SLM-QBR};
\draw[arr] (e3.south) -- ++(0,-0.275) -| (e4.north);
\draw[arr] (e4) -- (e5);
\draw[arr] (e5) -- (e6);
% ---- FASE 2 ----
\node[phase, below=1.0cm of f1 |- e4.south] (f2) {Fase 2\\Elicita\c{c}\~ao Delphi};
\node[stepbox, right=0.5cm of f2]  (r1) {Rodada 1\\Diverg\^encia\\(41 proposi\c{c}\~oes)};
\node[stepbox, right=of r1]        (r2) {Rodada 2\\Converg\^encia\\($Md\geq 4$)};
\node[stepbox, right=of r2]        (r3) {Rodada 3\\Consenso\\($W=0{,}74$)};
\draw[arr] (f2) -- (r1);
\draw[arr] (r1) -- (r2);
\draw[arr] (r2) -- (r3);
% segunda linha fase 2
\node[stepbox, below=0.55cm of r1]  (tr) {Triangula\c{c}\~ao\\Entrevistas\\($n=18$)};
\node[output, right=of tr]      (iv) {26 Vari\'aveis\\validadas $\rightarrow$ IRBI};
\draw[arr] (r3.south) -- ++(0,-0.275) -| (tr.north);
\draw[arr] (tr) -- (iv);
% conector entre fases
\draw[dashline] (e6.south) -- ++(0,-0.35) -| (f2.north);
% backgrounds
\begin{scope}[on background layer]
  \node[draw=black!25, rounded corners=4pt,
    fit=(f1)(e1)(e2)(e3)(e4)(e5)(e6),
    inner xsep=4pt, inner ysep=6pt] {};
  \node[draw=black!25, rounded corners=4pt,
    fit=(f2)(r1)(r2)(r3)(tr)(iv),
    inner xsep=4pt, inner ysep=6pt] {};
\end{scope}
\end{tikzpicture}
\caption{Fluxograma do processo integrado de adapta\c{c}\~ao transcultural (Fase~1) e elicita\c{c}\~ao Delphi (Fase~2) do question\'ario WOCAT-SLM.}
\label{fig:flowchart}
\end{figure*}

% ============================================================
% FASE 1: ADAPTAÇÃO TRANSCULTURAL
% ============================================================
\subsection{Fase 1. Adaptação Transcultural do WOCAT-SLM}

\subsubsection{Etapa 1. Tradução Direta (Inglês $\rightarrow$ Português)}

Dois tradutores independentes realizaram a tradução integral do questionário WOCAT-SLM (seções 2, 3, 5 e 6) do inglês para o português brasileiro. O Tradutor~1 (T1), profissional com formação em ciências agrárias, bilíngue e ciente dos objetivos do estudo, priorizou equivalência técnica e terminológica. O Tradutor~2 (T2), profissional sem formação técnica na área, bilíngue e não informado dos objetivos, preservou linguagem coloquial e acessibilidade.

A divergência intencional entre perfis maximizou a detecção de ambiguidades \cite{Beaton2000}. Cada tradutor produziu versão independente (T1 e T2) acompanhada de relatório de decisões.

\subsubsection{Etapa 2. Síntese das Traduções (T-12)}

Os tradutores e um mediador produziram versão sintetizada (T-12). Discrepâncias foram resolvidas mediante negociação documentada, com itens não resolvidos encaminhados ao comitê de especialistas.

\subsubsection{Etapa 3. Retrotradução (Português $\rightarrow$ Inglês)}

Dois retrotradutores independentes, nativos de língua inglesa ou com proficiência C2, sem conhecimento do original, traduziram a T-12 de volta para o inglês (BT1 e BT2). As retrotraduções foram comparadas com o instrumento original item a item.

\subsubsection{Etapa 4. Comitê de Especialistas}

O comitê multidisciplinar reuniu dez membros cuja composição heterogênea garantiu avaliação multidimensional, com três mestres de saberes quilombolas (experiência mínima de 25 anos em sistemas agroecológicos tradicionais) responsáveis pela equivalência experiencial, três pesquisadores doutores em agroecologia e etnoecologia com trajetória participativa encarregados da pertinência científica, dois especialistas em psicometria e adaptação transcultural para assegurar o rigor do protocolo e dois gestores de PI e extensionistas voltados à perspectiva operacional e institucional.

Fundamentada no princípio de soberania epistêmica \cite{Santos2007} e nas diretrizes ITC \cite{ITC2017}, a presença de mestres de saberes como membros plenos do comitê rompeu a assimetria avaliativa convencional.

O comitê avaliou cada item em quatro dimensões de equivalência (escala de 4 pontos) compreendendo as facetas semântica, idiomática, experiencial e conceitual. A robustez da concordância foi quantificada pelo Índice de Validade de Conteúdo, definido na Equação~\ref{eq:cvi} como a razão entre avaliadores que atribuíram pontuação 3 e o total de avaliadores.

\begin{equation}
IVC_{item} = \frac{\text{nº de avaliadores que atribuíram 3}}{\text{nº total de avaliadores}}
\label{eq:cvi}
\end{equation}

Itens com $IVC \geq 0,80$ foram aceitos sem ajustes, itens no intervalo $0,60 \leq IVC < 0,80$ foram revisados conforme sugestões do comitê e itens com $IVC < 0,60$ foram reformulados ou excluídos. Dos 68 itens analisados, 55 permaneceram inalterados, 5 foram reescritos e 8 constituíram acréscimos culturalmente específicos. A concordância interavaliadores aferida por kappa de Fleiss \cite{Fleiss1971} atingiu 0,78. O comitê também identificou lacunas culturais e propôs os itens suplementares que migraram para a etapa de pré-teste.

\subsubsection{Etapa 5. Pré-Teste}

Versão pré-final foi aplicada a 30 agricultores quilombolas de Jeremoabo (BA), selecionados por amostragem intencional com variabilidade em idade, gênero, escolaridade e sistema produtivo. A aplicação ocorreu em formato de entrevista assistida e, após cada seção, conduziu-se \textbf{debriefing cognitivo} \cite{Willis2005} com perguntas padronizadas de compreensão, alternativas linguísticas e pertinência experiencial.

Os indicadores quantitativos do pré-teste registraram taxa de compreensão média de 87\%, taxa de não-resposta de 11\%, tempo médio de aplicação de 53 minutos e ausência de efeitos teto ou piso relevantes.

\subsubsection{Etapa 6. Consolidação}

A versão final WOCAT-SLM-QBR foi consolidada com dossiê completo de adaptação, compreendendo as versões T1, T2, T-12, BT1, BT2, atas do comitê, dados do pré-teste, manual de aplicação e a versão aprovada, com subsequente encaminhamento ao WOCAT Secretariat.

% ============================================================
% FASE 2: ELICITAÇÃO DELPHI
% ============================================================
\subsection{Fase 2. Elicitação Estruturada via Protocolo Delphi}

\subsubsection{Composição do Painel}

O painel reuniu 21 participantes selecionados por amostragem intencional \cite{Patton2015}, cuja heterogeneidade controlada combinou cinco mestres de saberes quilombolas (experiência média de 27 anos) para ancoragem ênica, seis pesquisadores doutores em agroecologia, etnoecologia, PI ou gestão da inovação para rigor analítico, cinco técnicos extensionistas com experiência mínima de 8 anos em assessoria a comunidades tradicionais para perspectiva operacional e cinco gestores de PI e bioeconomia vinculados a NITs, SEBRAE, INPI e secretarias territoriais para composição institucional.

Transversalmente, os critérios de elegibilidade demandaram experiência mínima de 5 anos, reconhecimento pela comunidade epistêmica ou territorial e disponibilidade para três rodadas em quatro meses. Incluir mestres de saberes como especialistas de pleno direito ancorou-se em \cite{Santos2007,Beaton2000}.

\subsubsection{Estrutura das Rodadas}

Na primeira rodada, dedicada à divergência e exploração, questionário aberto solicitou a enumeração de variáveis relevantes para valoração de ativos tradicionais, organizadas nas seis dimensões derivadas do WOCAT-SLM-QBR (Tabela~\ref{tab:wocat_delphi}). As contribuições orais dos mestres de saberes foram transcritas por facilitadores e a consolidação foi conduzida mediante análise de conteúdo \cite{Braun2006}, resultando em 41 proposições iniciais.

A segunda rodada operou na dimensão da convergência, com questionário estruturado avaliando cada proposição em escala Likert de 5 pontos para relevância, clareza e operacionalidade. Calcularam-se mediana ($Md$), intervalo interquartil ($IQR$), coeficiente de variação ($CV$) e frequência de respostas extremas, tendo 32 proposições atingido $Md \geq 4$ e $IQR \leq 1,5$.

A terceira rodada consolidou o consenso mediante reenvio com feedback agregado (medianas, distribuição, posicionamento individual anonimizado), permitindo o ajuste final. O consenso operacional adotou $IQR \leq 1,0$ e $Md \geq 4,0$. Vinte e seis variáveis cumpriram simultaneamente os critérios quantitativos e qualitativos, enquanto seis foram encaminhadas para deliberação qualitativa complementar.

\subsubsection{Análise Estatística do Consenso}

A convergência foi aferida por mediana e intervalo interquartil (IQR) por variável e por rodada, observando-se redução média de 42\% no IQR entre as rodadas 1 e 3. O coeficiente de concordância de Kendall ($W$) alcançou 0,74, classificando o consenso como forte \cite{Schmidt2014}. O Coeficiente de Validade de Conteúdo \cite{Hernandez-Nieto2002} permaneceu acima do limiar $CVC \geq 0,80$, com média de 0,84. A taxa de estabilidade entre rodadas 2 e 3 indicou que 81\% dos painelistas ajustaram suas respostas em no máximo $\pm 1$ ponto. Teste de Friedman ($\alpha = 0,05$) seguido de Dunn confirmou diferenças significativas entre as distribuições das rodadas 1 e 2, inexistindo diferenças entre as rodadas 2 e 3. 

Todas as análises foram conduzidas em ambiente R versão 4.5.1 \cite{RCore2024}, empregando o pacote \texttt{irr} para cômputo de $W$ de Kendall e coeficientes de concordância, o pacote \texttt{PMCMRplus} para o teste de Friedman com comparações \textit{post hoc} de Dunn ajustadas por Bonferroni e funções nativas do pacote \texttt{stats} para estatísticas descritivas, enquanto o $CVC$ foi calculado via rotina própria implementada conforme algoritmo de \cite{Hernandez-Nieto2002}.

Para testar diretamente a hipótese de superioridade do método estruturado, os índices de consenso Delphi foram comparados com levantamento qualitativo não estruturado (grupo focal com 9 especialistas do mesmo universo). A diferença observada em termos de variância residual foi significativa ($p < 0,01$) após 5.000 permutações, corroborando a eficiência do protocolo estruturado.

\subsection{Correspondência entre Variáveis Linguísticas e Conjuntos Fuzzy}

As 26 variáveis linguísticas estabilizadas pelo Delphi constituem os termos primários do sistema de inferência Mamdani que operacionaliza o IRBI. A equivalência funcional entre o domínio empírico (escalas Likert consensuadas) e o domínio fuzzy (é estabelecida mediante mapeamento biunívoco, onde cada nível da escala (1 a 5) corresponde a um conjunto nebuloso (Muito Baixo, Baixo, Moderado, Alto, Muito Alto) com funções de pertinência triangulares sobrepostas em 25\% nos limites adjacentes. Essa sobreposição garante transição suave entre classes e preserva a granularidade das avaliações dos painelistas.

A calibração dos parâmetros de pertinência ($a$, $m$, $b$) para cada variável baseou-se nas distribuições observadas nas rodadas Delphi, de modo que o centroide de cada função triangular coincide com a mediana do painel e a abertura lateral reflete o intervalo interquartil. Essa conexão direta entre consenso especializado e topologia dos conjuntos nebulosos confere rastreabilidade ao modelo fuzzy e assegura que as regras SE-ENTÃO do IRBI herdam a validade de conteúdo certificada no processo Delphi.

\subsection{Análise de Sensibilidade Global do IRBI}

Para avaliar a robustez numérica do indicador composto frente a incertezas nos parâmetros de pertinência, empregou-se o método de triagem de Morris \cite{Morris1991}, adequado a modelos com elevado número de fatores e custo computacional moderado. O procedimento consiste em perturbar sistematicamente os parâmetros ($a$, $m$, $b$) de cada função triangular dentro de faixa de $\pm 15\%$ em torno dos valores calibrados pelo Delphi, gerando trajetórias aleatórias no espaço de entrada e computando efeitos elementares ($EE_i$) sobre o índice agregado. A média absoluta dos efeitos elementares ($\mu^*_i$) quantifica a influência global de cada parâmetro, enquanto o desvio padrão ($\sigma_i$) captura interações e não linearidades \cite{Campolongo2007}. O procedimento foi implementado em R~4.5.1 \cite{RCore2024} com o pacote \texttt{sensitivity}, utilizando $r = 20$ trajetórias e $p = 4$ níveis por fator, totalizando $(26 \times 3 + 1) \times 20 = 1.580$ avaliações do modelo. Variáveis com $\mu^*_i$ superior ao limiar $\mu^*_{\text{crítico}} = 0,10$ foram classificadas como parâmetros influentes, indicando que o indicador é sensível à calibração dessas funções e, portanto, exige monitoramento periódico de suas distribuições empíricas.

\subsection{Triangulação via Entrevistas Semiestruturadas}

Dezoito entrevistas com agricultores quilombolas de Jeremoabo, selecionados por saturação teórica \cite{Glaser1967}, complementaram os dados quantitativos. O roteiro abordou percepção sobre variáveis do Delphi, dimensões não contempladas, adequação da linguagem e hierarquização espontânea de prioridades.

As entrevistas foram gravadas em áudio, transcritas integralmente e submetidas a análise temática \cite{Braun2006} em cinco fases (familiarização, codificação aberta, busca por temas, revisão e redação). A codificação foi conduzida por dois pesquisadores independentes (kappa de Cohen = 0,72), e a triangulação foi operacionalizada via matriz de correspondência entre variáveis validadas e categorias temáticas emergentes.

\subsection{Aspectos Éticos}

O projeto foi aprovado pelo Comitê de Ética em Pesquisa da UFS (Resoluções CNS nº~466/2012 e nº~510/2016). Todos os participantes assinaram consentimento livre, prévio e informado. Os mestres de saberes integrantes do comitê foram reconhecidos como coautores do instrumento. Dados sensíveis foram tratados conforme Protocolo de Nagoia e Lei nº~13.123/2015, com devolutiva das sínteses às comunidades.


\section{Resultados e Discussão}
\label{sec:results_discussion}

\subsection{Adaptação Transcultural e Validação Psicométrica}

Como produto primário, obteve-se a versão WOCAT-SLM-QBR, instrumento adaptado transculturalmente para o contexto quilombola brasileiro contendo 68 itens traduzidos e oito itens suplementares culturalmente específicos. O IVC global atingiu 0,93, o kappa de Fleiss registrou 0,78 e a taxa de compreensão aferida no pré-teste permaneceu em 87\%. O dossiê de adaptação, com 142 páginas de rastreabilidade (relatórios T1/T2, retrotraduções, atas do comitê, planilhas do pré-teste e manual de aplicação), tornou-se referência replicável para outros contextos de comunidades tradicionais brasileiras.

Quanto às lacunas culturais, o mapeamento confirmou que as dimensões espiritual-ritual, transmissão intergeracional via oralidade e coletividade associada a bens comuns não são contempladas pelo WOCAT original. Os itens suplementares relativos a essas dimensões obtiveram $IVC = 0,91$, $kappa = 0,76$ e estabilidade semântica após o pré-teste. Os oito itens emergentes reforçam a limitação inerente a abordagens puramente \textit{etic}, evidenciando que frameworks universalistas carregam pressupostos culturais que operam como ``pontos cegos'' quando transplantados para ontologias distintas \cite{Herdman1999}.

\subsection{Elicitação Delphi e Convergência Estatística}

O protocolo Delphi estabilizou 26 variáveis linguísticas distribuídas nas seis dimensões derivadas do WOCAT, com definições operacionais consensuadas e escalas padronizadas. Cada variável apresenta ficha técnica contendo estatísticas ($Md$, $IQR$, $W$, $CVC$) e mapeamento para os indicadores do Índice de Resiliência Biocultural Integrada (IRBI).

Frente ao grupo focal não estruturado, o método Delphi alcançou coeficientes de concordância significativamente mais elevados ($W_{Delphi}=0,74$ versus $W_{GF}=0,41$) e redução de 36\% na variância das respostas, corroborando a eficiência do protocolo iterativo com feedback controlado. Essa integração sequencial entre adaptação transcultural e elicitação Delphi responde a lacuna identificada tanto na gestão da inovação \cite{Tidd2005} quanto na economia de ativos intangíveis \cite{Lev2001}, dado que inexistia cadeia metodológica conectando rigor instrumental com consenso auditável sem sacrificar legitimidade cultural junto aos detentores dos saberes. Pelo prisma da Teoria dos Recursos da Firma \cite{Barney1991}, as variáveis elicitadas operacionalizam os atributos VRIN em dimensões mensuráveis, viabilizando que comunidades quilombolas demonstrem o valor estratégico de seus ativos intangíveis, pré-requisito para negociações de repartição de benefícios, certificação de produtos e proteção jurídica via indicações geográficas ou marcas coletivas \cite{Belletti2015}.

\subsection{Triangulação e Validade Ecológica}

Cruzando consenso técnico (Delphi) e percepção comunitária (entrevistas), a matriz de correspondência evidenciou correlação de Pearson $r=0,68$ ($p<0,01$), confirmando validade ecológica e indicando que o consenso especializado preserva coerência com as prioridades percebidas pelas comunidades. Ter incorporado mestres de saberes quilombolas como membros plenos tanto do comitê de adaptação quanto do painel Delphi configura inovação metodológica alinhada ao paradigma da soberania epistêmica \cite{Santos2007}, na medida em que os detentores de saberes tradicionais passam a coautores do instrumento e do consenso, exercendo agência sobre como sua realidade é representada e mensurada.

\subsection{Sensibilidade Multidimensional do IRBI}

A análise de sensibilidade global via método de Morris revelou que a variância do IRBI não é monopolizada por parâmetros biofísicos. Das seis dimensões componentes, a cultural-simbólica e a social-organizacional apresentaram valores de $\mu^*_i$ comparáveis aos da dimensão biofísica-ambiental, confirmando empiricamente que o indicador preserva a centralidade do saber tradicional na composição do índice. Especificamente, as variáveis associadas a transmissão intergeracional, significado ritual e redes de cooperação comunitária figuraram entre os dez parâmetros mais influentes ($\mu^*_i > 0,10$), demonstrando que perturbações nos parâmetros de pertinência dessas variáveis culturais afetam o IRBI com magnitude equivalente àquela observada para variáveis como agrobiodiversidade e resiliência edáfica. Tal resultado valida a premissa de que o IRBI opera como indicador socioecológico de resiliência biocultural e não como métrica exclusivamente biofísica, reforçando a coerência com o eixo central da tese de que a valoração de SSAT exige mensuração equitativa de capitais natural, cultural, social e institucional.

\subsection{Implicações para Governança do Conhecimento e Modelagem Fuzzy}

A articulação entre protocolo de adaptação e princípios ISO~30401 (Gestão do Conhecimento) oferece contribuição teórica à literatura de gestão da propriedade intelectual em contextos comunitários. Ao documentar cada decisão com rastreabilidade, o dossiê de adaptação cria infraestrutura de metadados que atende simultaneamente requisitos de governança do conhecimento \cite{ISO30401} e demandas de proteção de conhecimentos tradicionais associados à biodiversidade, funcionalidade dual que posiciona o estudo na interface entre psicometria transcultural e gestão estratégica de PI \cite{Teece1986,ISO56005}.

Na arquitetura mais ampla do programa de pesquisa, o presente estudo opera como fundação metodológica, uma vez que o instrumento culturalmente calibrado e as variáveis consensuadas alimentarão diretamente as funções de pertinência e regras SE-ENTÃO do sistema fuzzy Mamdani. Cada variável do IRBI terá origem documentada em adaptação transcultural e consenso especializado, conferindo rastreabilidade ao modelo \cite{Costanza1997} e atendendo à cadeia de evidências que se estende do WOCAT original ao WOCAT-SLM-QBR, deste ao consenso Delphi e, finalmente, ao modelo fuzzy. O manual de operacionalização do protocolo integrado adaptação-Delphi documenta cada decisão crítica com granularidade suficiente para replicação independente.


\section{Considerações Finais}
\label{sec:conclusion}

Este estudo concluiu a adaptação transcultural sistemática do questionário WOCAT-SLM para comunidades quilombolas brasileiras mediante protocolo de seis etapas complementado por diretrizes ITC e princípios de pesquisa-ação participativa, seguida de elicitação estruturada via Delphi e triangulação qualitativa. O WOCAT-SLM-QBR foi disponibilizado com métricas psicométricas robustas (IVC = 0,93, kappa = 0,78, compreensão = 87\%) e oito itens suplementares que preservam comparabilidade internacional sem suprimir especificidades quilombolas, acompanhado de portfólio contendo 26 variáveis linguísticas consensuadas ($W = 0,74$, $CVC = 0,84$) destinadas ao IRBI. Ao demonstrar que ativos bioculturais intangíveis podem ser convertidos em variáveis auditáveis mediante cadeia metodológica com rastreabilidade documentada, o estudo contribui para o campo da economia ecológica ao oferecer protocolo replicável de valoração alternativa de riqueza natural e cultural em comunidades tradicionais. O protocolo integrado adaptação-Delphi, documentado com checklists, templates de feedback e scripts estatísticos, oferece referência replicável às demais comunidades tradicionais brasileiras, enquanto a evidência empírica dos ``pontos cegos'' culturais de frameworks universalistas de SLM operacionaliza o conceito de soberania epistêmica em instrumentos de mensuração. Esse arcabouço estabelece a camada fundacional de um sistema de governança bioeconômica onde mensuração, elicitação estruturada e modelagem computacional compartilham origem comum culturalmente validada.

A versão WOCAT-SLM-QBR e seu dossiê completo de adaptação foram submetidos ao Secretariado do WOCAT para incorporação à rede global de adaptações regionais, contribuindo para a internacionalização dos saberes agroecológicos quilombolas brasileiros em framework que garanta simultaneamente rigor científico e soberania epistêmica.


%% References with BibTeX database:

\bibliographystyle{elsarticle-harv}
\bibliography{references}

\end{document}